\chapter{Galoistheorie und Anwendungen}





\section{Hauptsatz der Galois-Theorie}

\begin{convention}
  Es sei~$L/K$ eine Körpererweiterung.
\end{convention}


\begin{definition}
  Für jeden Untergruppe~$H$ von~$\Gal(L/K)$ ist
  \[
    L^G
    \defined
    \{
      x \in L
    \suchthat
      \text{$\varphi(x) = x$ für alle~$\varphi \in H$}
    \}
  \]
  der \defemph{Fixkörper} von~$H$.
\end{definition}

\begin{proposition}
  Für jede Untergruppe~$H$ von~$\Gal(L/K)$ ist~$L^H$ ein Zwischenkörper der Erweiterung~$L/K$.
\end{proposition}

Ist~$Z$ ein Zwischenkörper der Erweiterugn~$L/K$, ist ist die Galois-Gruppe~$\Gal(L/Z)$ eine Untergruppe von~$\Gal(L/K)$.
Wir haben also die folgenden Abbildungen:
