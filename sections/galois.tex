\chapter{Galoistheorie und Anwendungen}





\section{Hauptsatz der Galois-Theorie}

\begin{convention}
  Es sei~$L/K$ eine Körpererweiterung.
\end{convention}


\begin{definition}
  Für jeden Untergruppe~$H$ von~$\Gal(L/K)$ ist
  \[
    L^G
    \defined
    \{
      \alpha \in L
    \suchthat
      \text{$\varphi(\alpha) = \alpha$ für alle~$\varphi \in H$}
    \}
  \]
  der \defemph{Fixkörper} von~$H$.
\end{definition}

\begin{lemma}
  Es sei~$G \defined \Gal(L/K)$.
  \begin{enumerate}
    \item
      Für jede Untergruppe~$H$ von~$G$ ist~$L^H$ ein Zwischenkörper der Erweiterung~$L/K$.
    \item
      Für jeden Zwischenkörper~$Z$ der Erweiterung~$L/K$ ist~$\Gal(L/Z)$ eine Untergruppe von~$G$.
  \end{enumerate}
\end{lemma}

\begin{proposition}
  \label{general galois}
  Es sei~$L/K$ ein Körpererweiterung mit Galois-Gruppe~$G \defined \Gal(L/K)$.
  Wir betrachten die beiden Abbildungen
  \begin{align*}
    \Phi
    \colon
    \{ \text{Untergruppen von~$\Gal(L/K)$} \}
    &\onetoone
    \{ \text{Zwischenkörper von~$L/K$} \}
    \cocolon
    \Psi
    \\
    H
    &\mapsto
    L^H \,,
    \\
    \Gal(L/Z)
    &\mapsfrom
    Z \,.
  \end{align*}
  \begin{enumerate}
    \item
      Sind~$H$,~$H'$ zwei Untergruppen von~$G$ mit~$H \subseteq H'$ so gilt~$L^H \supseteq L^{H'}$.
      Die Abbildung~$\Phi$ ist also ordnungsumkehrend.
    \item
      Sind~$Z$,~$Z'$ zwei Zwischenkörper von~$L/K$ mit~$Z \subseteq Z'$ so gilt~$\Gal(L/Z) \supseteq \Gal(L/Z')$.
      Die Abbildung~$\Psi$ ist also ordnungsumkehrend.
    \item
      Für jede Untergruppe~$H$ von~$\Gal(L/K)$ und jeden Zwischenkörper~$Z$ von~$L/K$ gilt
      \[
        L^H \supseteq Z
        \iff
        \text{$\varphi(\alpha) = \alpha$ für alle~$\varphi \in H$,~$\alpha \in Z$}
        \iff
        H \subseteq \Gal(L/Z) \,.
      \]
      Es ist also genau dann~$\Phi(H) \supseteq Z$, wenn~$H \subseteq \Psi(Z)$.
    \item
      Für jede Untergruppe~$H$ von~$G$ und jeden Zwischenkörper~$Z$ von~$L/K$ gilt
      \[
        H \subseteq \Gal(L / L^H) \,,
        \quad
        Z \subseteq L^{\Gal(L/Z)} \,.
      \]
      Die Abbildungen~$\Psi \Phi$ und~$\Phi \Psi$ sind also ordnungserhaltend.
    \item
      Für jede Untergruppe~$H$ von~$G$ und jeden Zwischenkörper~$Z$ von~$L/K$ gilt
      \[
        L^{\Gal(L / L^H)} = L^H \,,
        \quad
        \Gal( L / L^{\Gal(L / Z)} ) = \Gal(L / Z) \,.
      \]
      Es gilt also~$\Phi \Psi \Phi = \Phi$ und~$\Psi \Phi \Psi = \Psi$.
    \item
      Die Abbildungen~$\Phi$ und~$\Psi$ schränken sich zu zueinander inversen Bijektionen zwischen den Bildern von~$\Psi$ und~$\Phi$ ein.
  \end{enumerate}
\end{proposition}

\begin{proposition}
  \label{galois galois}
  Es sei~$L/K$ eine endliche Körpererweiterung, und~$\Phi$ und~$\Psi$ seien wie in \cref{general galois}.
  \begin{enumerate}
    \item
      Die Abbildung~$\Psi$ ist surjektiv.
    \item
      Es gilt~$\Psi \Phi = \id$.
    \item
      Die Abbildung~$\Phi$ ist injektiv.
  \end{enumerate}
\end{proposition}

\begin{proposition}
  \label{characterizations of galois extensions}
  Die Erweiterung~$L/K$ sei endlich mit Galois-Gruppe~$G \defined \Gal(L/K)$.
  Die folgenden Aussagen sind äquivalent:
  \begin{equivlist}
    \item
      Die Erweiterung~$L/K$ ist normal und separabel.
    \item
      Es gilt~$\card{ \Gal(L/K) } = [L : K]$.
    \item
      Für jedes Element~$\alpha \in L$ ist das Minimalpolynom von~$\alpha$ über~$K$ gegeben durch
      \[
        \mu_\alpha
        =
        \prod_{\beta \in G.\alpha}
        (X - \beta) \,.
      \]
    \item
      Es gilt~$K = L^G$.
  \end{equivlist}
\end{proposition}

\begin{definition}
  Eine endliche Körpererweiterung~$L/K$, welche die äquivalenten Bedingungen aus \cref{characterizations of galois extensions} erfüllt, ist eine \defemph{Galois-Erweiterung}.
\end{definition}

\begin{remark}
  Nach dem Satz vom primitven Element ist jede Galois-Erweiterung bereits einfach.
\end{remark}

\begin{theorem}[Galois-Korrespondenz, Hauptsatz der Galois-Theorie]
  Es sei~$L/K$ eine Galois-\hspace{0pt}Erweiterung mit Galois-Gruppe~$L/K$.
  Es seien~$\Phi$ und~$\Psi$ wie in \cref{general galois}.
  \begin{enumerate}
    \item
      Die Abbildungen~$\Phi$ und~$\Psi$ sind zueinander inverse Bijektionen.
  \end{enumerate}
  Es sei nun~$H$ eine Untergruppe von~$G$ und~$Z$ der zugehörige Zwischenkörper von~$L/K$, d.\,h. es gelten~$H = \Gal(L/Z)$ und~$Z = L^H$.
  \begin{enumerate}[resume*]
    \item
      Es gilt~$\card{H} = [L : Z]$ und~$[G : H] = [Z : K]$.
    \item
      Für jedes~$\varphi \in G$ gilt~$\Gal(L / \varphi(Z)) = \varphi H \varphi^{-1}$.
    \item
      Die Erweiterung~$L/Z$ ist ebenfalls eine Galois-Erweiterung, mit Galois-Gruppe~$H$.
    \item
      Die Erweiterung~$Z/K$ ist genau dann normal, wenn~$H$ ein Normalteiler in~$G$ ist.
      Es ist dann
      \[
        G / H
        \to
        \Gal(Z/K) \,,
        \quad
        \class{\varphi}
        \mapsto
        \restrict{\varphi}{Z}
      \]
      ein wohldefinierter Gruppenisomorphismus.
  \end{enumerate}
\end{theorem}





\section{Lösbarkeit durch Radikale}



\subsection{Auflösbarkeit von Gruppen}

\begin{convention}
  Im Folgenden sei~$G$ eine Gruppe.
\end{convention}

\begin{definition}
  \leavevmode
  \begin{enumerate}
    \item
      Der \defemph{Kommutator} zweier Elemente~$g, h \in G$ ist~$[g,h] \defined g h g^{-1} h^{-1}$.
    \item
      Die \defemph{Kommutator-Untergruppe} von~$G$ ist~$[G,G] \defined \gen{ [g,h] \suchthat g, h \in G }$.
    \item
      Die \defemph{abgeleiteten Untergruppen} von~$G$ sind rekursiv gegeben durch
      \[
        \Derived^0(G) \defined G \,,
        \quad
        \Derived^{i+1}(G) \defined [\Derived^i(G), \Derived^i(G)] \,.
      \]
  \end{enumerate}
\end{definition}

\begin{warning}
  Im Allgemeinen gilt~$[G,G] \neq \{ [g,h] \suchthat g, h \in G \}$.
  Die kleinsten Gegenspiele sind von Ordnung~$96$.
\end{warning}

\begin{proposition}
  \leavevmode
  \begin{enumerate}
    \item
      Die Kommutatoruntergruppe~$[G,G]$ ist normal in~$G$.
    \item
      Ist~$N$ ein Normalteiler in~$G$, so ist die Faktorgruppe~$G/N$ genau dann abelsch, wenn die Inklusion~$N \supseteq [G,G]$ gilt.
  \end{enumerate}
\end{proposition}

\begin{proposition}
  \label{characterization of solvable groups}
  Die folgenden Bedingungen sind äquivalent.
  \begin{enumerate}
    \item
      Es gibt eine Normalenreihe
      \[
        1
        =
        N_0
        \normgroupeq
        N_1
        \normgroupeq
        \dotsb
        \normgroupeq
        N_n
        =
        G \,,
      \]
      so dass die Faktorgruppen~$N_i / N_{i-1}$ für alle~$i = 1, \dotsc, n$ abelsch sind.
    \item
      Es gilt~$\Derived^i(G) = 1$ für genügend großes~$i$.
  \end{enumerate}
  Ist die Gruppe~$G$ endlich, so ergeben sich außerdem die folgenden äquivalenten Bedingungen.
  \begin{enumerate}[resume*]
    \item
      Es gibt eine Normalenreihe
      \[
        0
        =
        N_0
        \normgroupeq
        N_1
        \normgroupeq
        \dotsb
        \normgroupeq
        N_n
        =
        G \,,
      \]
      so dass die Faktorgruppen~$N_i / N_{i-1}$ für alle~$i = 1, \dotsc, n$ zyklisch und endlich sind.
    \item
      Es gibt eine Normalenreihe
      \[
        0
        =
        N_0
        \normgroupeq
        N_1
        \normgroupeq
        \dotsb
        \normgroupeq
        N_n
        =
        G \,,
      \]
      so dass die Faktorgruppen~$N_i / N_{i-1}$ für alle~$i = 1, \dotsc, n$ Primordnung haben.
  \end{enumerate}
\end{proposition}

\begin{definition}
  \leavevmode
  \begin{enumerate}
    \item
      Die Gruppe~$G$ ist \defemph{auflösbar}, falls sie die äquivalenten Bedingungen aus~\cref{characterization of solvable groups} erfüllt.
    \item
      Die Gruppe~$G$ ist \defemph{perfekt}, falls~$[G, G] = G$ gilt.
  \end{enumerate}
\end{definition}

\begin{lemma}
  \label{solveable via ses}
  Ist~$N$ ein Normalteiler in~$G$, so ist~$G$ genau dann auflösbar, wenn~$N$ und~$G/N$ beide auflösbar sind.
\end{lemma}

\begin{example}
  \leavevmode
  \begin{enumerate}
    \item
      Jede abelsche Gruppe ist auflösbar, da~$\Derived^1(G) = [G,G] = 1$.
    \item
      Ist~$p$ eine Primzahl, so ist jede~\Gruppe{$p$} auflösbar.
    \item
      Für alle~$n \in \Natural$ gilt~$[\Sym_n, \Sym_n] = \Alt_n$.
      Die alternierende Gruppe~$\Alt_n$ ist für alle~$n \geq 5$ perfekt.
      Für~$n = 4$ gilt~$[\Alt_4, \Alt_4] \cong \Integer_2 \times \Integer_2$, und für~$n = 1, 2, 3$ ist~$\Alt_n$ abelsch, also~$[\Alt_n, \Alt_n] = 0$.
      Wir erhalten somit, dass~$\Sym_n$ und~$\Alt_n$ für~$n \leq 4$ auflösbar sind, und für~$n \geq 5$ nicht.
  \end{enumerate}
\end{example}

\begin{remark}
  \leavevmode
  \begin{enumerate}
    \item
      Jede Gruppe von Ordnung~$< 60$ ist auflösbar (die kleinste nicht-auflösbare Gruppe ist~$\Alt_5$);
      dies zeigt man mithilfe der Sylowsätze.
    \item
      Jede Gruppe der Ordnung~$p^n q^m$ mit~$p$,~$q$ prim ist auflösbar;
      dies zeigt man mithilfe von Darstellungstheorie.
    \item
      Jede Gruppe ungerader Ordnung ist auflösbar;
      dies zeigt man nicht selber, sondern zitiert den entsprechenden Beweis von Feit-Thompson.
  \end{enumerate}
\end{remark}



\subsection{Radikalerweiterungen}

\begin{convention}
  Im Folgenden ist~$L/K$ eine Körpererweiterung, sofern nicht anders angegeben.
\end{convention}

Im Allgemeinen können Galois-Gruppen sehr viele mögliche Werte annehmen.

\begin{theorem}
  Für jede endliche Gruppe~$G$ gibt es für eine Galois-Erweiterung~$L/K$ mit
  \[
    \Gal(L/K) \cong G \,.
  \]
  Dies gilt sowohl in Charakteristik~$0$ als auch in primer Charakteristik.
\end{theorem}

\begin{definition}
  Die Erweiterung~$L/K$ ist \defemph{zyklisch}, \defemph{abelsch}, \defemph{auflösber}, etc., falls die folgenden Bedingungen gelten.
  \begin{itemize}
    \item
      Die Erweiterung~$L/K$ ist eine Galois-Erweiterung.
    \item
      Die Galois-Gruppe~$\Gal(L/K)$ hat die entsprechende Eigenschaft.
  \end{itemize}
\end{definition}

\begin{definition}
  \leavevmode
  \begin{enumerate}
    \item
      Die Erweiterung~$L/K$ entsteht durch \defemph{Adjunktion einer Wurzel}, fall es ein Element~$\alpha \in L$ mit~$L = K(\alpha)$ gibt, so dass~$\alpha^n \in K$ für einen passenden Exponenten~$n \in \Natural_1$.
    \item
      Die Körperweiterung~$L/K$ ist eine \defemph{Radikalerweiterung}, falls es eine aufsteigende Kette von Körperweiterungen
      \[
        K = Z_0 \subseteq Z_1 \subseteq Z_2 \subseteq \dotsb \subseteq Z_n = L
      \]
      gibt, so dass es für jede Erweiterung~$Z_i / Z_{i-1}$ durch Adjunktion einer Wurzel entsteht.
  \end{enumerate}
\end{definition}

\begin{proposition}
  \label{cyclic iff simple}
  Es sei~$n \in \Natural_1$, es gelte~$\ringchar(K) \ndivides n$ (etwa~$\ringchar(K) = 0$) und~$K$ enthalte alle~\ten{$n$} Einheitswurzeln (aus~$\closure{K}$).
  Dann sind für eine Körperweiterung~$L/K$ die folgenden beiden Bedingungen äquivalent.
  \begin{equivlist}
    \item
      Es gibt ein Element~$\alpha \in L$ mit~$L = K(\alpha)$ und~$\alpha^n \in K$.
    \item
      Die Erweiterung~$L/K$ ist zyklisch vom Grad~$n$, d.\,h.~$L/K$ ist galoisch mit~$\Gal(L/K) \cong \Integer_n$.
  \end{equivlist}
\end{proposition}

Bis auf Probleme mit Einheitswurzeln und folgt aus \cref{cyclic iff simple} für~$\ringchar(K) = 0$ mithilfe der Galois-Korrespondenz, dass eine Körpererweiterung~$L/K$ genau dann durch Radikale auflösbar ist, wenn~$L/K$ auflösbar ist.


\subsection{Auflösbarkeit durch Radikale}

\begin{definition}
  Ein Polynom~$f \in K[X]$ ist \defemph{durch Radikale auflösbar}, falls es eine Radikalerweiterung~$L/K$ gibt, so dass~$f$ über~$L$ in Linearfaktoren zerfällt. 
\end{definition}

Anschaulich gesehen ist~$f$ durch Radikale auflösbar, wenn sich die Nullstellen von~$f$ mithilfe der üblichen Körperoperationen, sowie Wurzeln, ausdrücken lassen. 

\begin{definition}
  Es sei~$f \in K[X]$ und es sei~$L$ ein Zerfällungskörper von~$f$ über~$K$.
  Die Gruppe~$\Gal(f) \defined \Gal(L/K)$ die \defemph{Galois-Gruppe} von~$f$.
\end{definition}

\begin{theorem}
  Es sei~$f \in K[X]$ und es gelte~$\ringchar(K) = 0$.
  Dann ist das Polynom~$f$ genau dann durch Radikale auflösbar, falls die Gruppe~$\Gal(f)$ auflösbar ist.
\end{theorem}

\begin{lemma}
  \leavevmode
  \begin{enumerate}
    \item
      Es gibt für jede endliche Gruppe~$G$ einen Körper~$K$ von Charakteristik~$0$ und ein Polynom~$f \in K[X]$ mit~$\Gal(f) \cong G$.
    \item
      Für jedes~$n \geq 1$ gibt es ein Polynom~$f \in \Rational[X]$ vom Grad~$n$ mit~$\Gal(f) \cong \Sym_5$.
  \end{enumerate}
\end{lemma}

\begin{theorem}[Abel, Ruffini]
  Nicht jedes Polynom~$f \in \Rational[X]$ vom Grad~$\deg(f) \geq 5$ ist durch Radikale auflösbar.
\end{theorem}





\section{Konstruierbarkeit}



\subsection{Konstruierbare Punkte}

\begin{convention}
  Im folgenden ist~$E$ die euklidische Ebene, sofern nicht anders angegeben.
\end{convention}

\begin{definition}
  Es sei~$M$ eine Teilmenge von~$E$.
  \begin{enumerate}
    \item
      Es ist~$G(M)$ die Menge aller Geraden in~$E$, die durch zwei verschieden Punkte von~$M$ gehen.
    \item
      Es ist~$C(M)$ die Menge aller Kreise in~$E$, deren Mittelpunkt in~$M$ liegt, und deren Radius der Abstand zweier Punkte von~$M$ ist.
    \item
      Es ist~$\ZL(M)$ die Menge aller Schnittpunkte zweier verschiedener Elemente aus~$G(M) \cup C(M)$.
    \item
      Es seien induktiv~$M_0 \defined M$ und~$M_{i+1} \defined \ZL(M_i)$ für alle~$i \geq 0$.
    \item
      Es ist~$M_{\infty} \defined \bigcup_{i=0}^\infty M_i$ die Menge aller aus~$M$ \defemph{(durch Zirkel und Lineal) konstruierbarer Punkte}.
  \end{enumerate}
\end{definition}

Wir wollen verstehen, welche Punkte der Ebene~$E$ sich ausgehend von zwei verschiedenen Punkten~$x \neq y$ durch Zirkel und Lineal konstruieren lassen.
Hierfür identifzieren wir~$E$ mit~$\Complex$, wobei wir annehmen können, dass~$M = \{ x, y \} = \{ 0, 1 \}$.

\begin{theorem}
  \leavevmode
  \begin{enumerate}
    \item
      Die Menge der konstruierbaren Punkte,~$M_{\infty}$, ist ein Zwischenkörper der Erweiterung~$\Complex / \Rational$.
    \item
      Die Erweiterung~$M_{\infty} / \Rational$ ist algebraisch.
    \item
      Der Körper~$M_{\infty}$ ist quadratisch abgeschlossen, d.\,h. für jedes Element~$z \in Z$ und jede komplexe Zahl~$w \in \Complex$ mit~$z = w^2$ gilt auch~$w \in M_{\infty}$.
    \item
      Es gilt~$i \in M_{\infty}$.
    \item
      Es gilt~$\conjugate{M_{\infty}} = M_{\infty}$, wobei~$\conjugate{(-)}$ die komplexe Konjugation bezeichnet.
    \item
      Für~$z \in \Complex$ gilt genau dann dann~$z \in M_{\infty}$, wenn~$\Re(z), \Im(z) \in M_{\infty}$.
    \item
      Für~$z \in Z$ mit~$z \neq 0$ gilt für~$z = r e^{i \varphi}$ mit~$r \in \Real_{>0}$ und~$\varphi \in \Real$ genau dann~$z \in M_{\infty}$, wenn~$r, e^{i \varphi} \in M_{\infty}$.
  \end{enumerate}
\end{theorem}

\begin{theorem}
  Für~$z \in \Complex$ sind die folgenden Bedingungen äquivalent:
  \begin{equivlist}
    \item
      Es gilt $z \in M_{\infty}$.
    \item
      Es gibt einen Zwischenkörper~$L$ von~$\Complex/\Rational$, der~$z$ enthält, so das~$L$ aus~$\Rational$ durch sukzessive Adjunktion von Quadratwurzeln ensteht.
    \item
      Es gibt eine aufsteigende Folge
      \[
        \Rational = K_0 \subseteq K_1 \subseteq K_2 \subseteq \dotsb \subseteq K_n = L
      \]
      von~$\Complex$, so dass~$[K_i : K_{i-1}] = 2$ für alle~$i = 1, \dotsc, n$ gilt, sowie~$z \in L$.
  \end{equivlist}
\end{theorem}

\begin{corollary}
  Ist~$z \in \Complex$ konstruierbar, so gilt
  \[
    [\Rational(z) : \Rational] = 2^n
  \] 
  für einen passenden Exponenten~$n \in \Natural$.
\end{corollary}



\subsection{Quadratur des Kreises}

Ist es möglich, aus einen Kreis und seinem Mittelpunkt mithilfe von Zirkel und Lineal ein Quadrat mit gleichen Flächeninhalt zu konstruieren?
Algebraisch übersetzt sich dieses Problem in die Konstruierbarkeit von~$\pi \in \Complex$, welche nach Lindemann nicht möglich ist.

\begin{theorem}[Lindemann]
  Die reelle Zahl~$\pi$ is transzendent über~$\Rational$.
\end{theorem}

\begin{corollary}
  Die Zahl~$\pi$ ist nicht konstruierbar.
\end{corollary}



\subsection{Würfelverdoppelung}

Ist es möglich, aus einem Würfel mithilfe von Zirkel und Lineal einen Würfel vom doppelten Volumen zu konstruieren?
Algebraisch übersetzt sich dieses Problem in die Konstruierbarkeit von~$\sqrt[3]{2}$.
Da~$[\Rational( \sqrt[3]{2} ) : \Rational] = 3$ gilt, ist~$\sqrt[3]{2}$ allerdings nicht konstruierbar, die Würfeldoppelung also nicht möglich.



\subsection{Konstruierbarkeit des Regelmäßigen~\texorpdfstring{$n$}-Ecks}

Ist es möglich, ein regelmäßigen~\Eck{$n$} zu konstruieren?
Algebraisch übersetzt sich dies in dieses Problem in die Konstruierbarkeit von~$e^{2 \pi i / n}$, und hängt somit von~$n$ ab.

\begin{definition}
  Eine \defemph{Fermat-Primzahl} ist eine Primzahl der Form~$2^{2^n} + 1$ für~$n \in \Natural$.
\end{definition}

\begin{example}
  Die ersten fünf, und einzig bekannten, Fermat-Primzahlen sind
  \[
    3 \,,
    \quad
    5 \,,
    \quad
    17 \,,
    \quad
    257 \,,
    \quad
    65537 \,.
  \]
\end{example}

\begin{theorem}[Gauß]
  Für jede Anzahl an Ecken~$n \geq 3$ sind die folgenden Aussagen äquivalent.
  \begin{equivlist}
    \item
      Das regelmäßige~\Eck{$n$} ist mit Zirkel und Lineal konstruierbar.
    \item
      Es gilt
      \[
        n = 2^m p_1 p_2 \dotsm p_r
      \]
      für einen Exponenten~$m \in \Natural$ und paarweise verschiedene Fermat-Primzahlen~$p_1, \dotsc, p_r$.
  \end{equivlist}
\end{theorem}



\subsection{Dreiteilung eines Winkels}

Für alle~$z_1, z_2 \in Z$ sei~$\sphericalangle(z_1, z_2) \in [0, \alpha)$ der (unorientierte) Winkzel zwischen der Strecke~$\overline{0,z_1}$ und der Strecke~$\overline{0,z_2}$.
Es stellt sich nun die Frage, wann sich ein gegebener Winkel durch Zirkel und Lineal dritteln lässt.

\begin{proposition}
  Die Dreiteilung eines Winkels~$\alpha \in [0, \pi)$ ist genau dann möglich, wenn
  \[
    [ \Rational( e^{i \alpha / 3}  ) : \Rational( e^{i \alpha} ) ]
    \leq 2
    \,.
  \]
\end{proposition}





\section{Kreisteilungskörper und Kreisteilungspolynome}



\subsection{Eulersche~\texorpdfstring{$\varphi$}{phi}-Funktion}

\begin{definition}
  Die \defemph{Eulersche~\Funktion{$\varphi$}}~$\varphi \colon \Natural_1 \to \Natural_1$ ist gegeben durch
  \[
    \varphi(n)
    \defined
    \{
      0 \leq d \leq n-1
    \suchthat
      \text{$d$ und~$n$ sind teilerfremd}
    \} \,.
  \]
\end{definition}

\begin{proposition}
  Es sei~$n \in \Natural_1$.
  \begin{enumerate}
    \item
      Es gilt~$\varphi(n) = \card{ \Integer_n^\times }$.
    \item
      Es ist~$\varphi(n)$ die Anzahl der Erzeuger der Gruppe~$(\Integer_n, +)$.
  \end{enumerate}
\end{proposition}

\begin{lemma}
  Sind~$n, m \in \Natural_1$ teilerfremd, so gilt~$\Integer_{nm} \cong \Integer_n \times \Integer_m$ als Ringe.
\end{lemma}

\begin{proposition}
  \leavevmode
  \begin{enumerate}
    \item
      Ist~$\varphi$ prim und~$k \in \Natural$, so gilt~$\varphi(p^k) = p^k - p^{k-1} = (p-1) p^{k-1}$.
    \item
      Sind~$n, m \in \Natural_1$ teilerfremd, so gilt~$\varphi(nm) = \varphi(n) \varphi(m)$.
  \end{enumerate}
\end{proposition}

\begin{example}
  Es gilt~$\varphi(6125) = \varphi(5^3 \cdot 7^2) = \varphi(5^3) \cdot \varphi(7^2) = 4 \cdot 5^2 \cdot 6 \cdot 7 = 4200$.
\end{example}



\subsection{Kreisteilung}

\begin{convention}
  Im Folgenden sei~$n \in \Natural_1$, und es gelte~$\ringchar(K) \ndivides n$ (etwa~$\ringchar(K) = 0$).
\end{convention}

\begin{definition}
  \leavevmode
  \begin{enumerate}
    \item
      Es ist~$\Ein_n \defined \{ x \in \closure{K} \suchthat x^n = 1 \}$ die Menge der~\ten{$n$} Einheitswurzeln über~$K$, eine multiplikative Untergruppe von~$\closure{K}^\times$.
    \item
      Der~\te{$n$} \defemph{Kreisteilungskörper} ist~$K(\Ein_n(K))$.
    \item
      Eine Einheitswurzel~$\zeta \in \Ein_n$ ist \defemph{primitiv} falls sie die Gruppe~$\Ein_n$ erzeugt.
    \item
      Das~\te{$n$} \defemph{Kreisteilungspolynom} ist~$\Phi_n \defined \prod_{\zeta} (X - \zeta) \in K[X]$, wobei~$\zeta$ die primitiven~\ten{$n$} Einheitswurzeln durchläuft.
  \end{enumerate}
\end{definition}

\begin{proposition}
  Die folgenden Körper sind gleich.
  \begin{equivlist}
    \item
      $K( \Ein_n )$, der~\te{$n$} Kreisteilungskörper, also der Zerfällungskörper von~$X^n - 1$.
    \item
      $K( \zeta )$, wobei~$\zeta \in \Ein_n$ eine primitive~\te{$n$} Einheitswurzel ist.
    \item
      $K( \Phi_n )$, der Zerfällungskörper des Kreisteilungspolynomes~$\Phi_n$.
  \end{equivlist}
\end{proposition}

\begin{proposition}
  \leavevmode
  \begin{enumerate}
    \item
      Es gilt~$\card{ \Ein_n } = n$.
    \item
      Es gilt~$X^n - 1 = \prod_{d \divides n} \Phi_d$.
    \item
      Es gilt~$\deg( \Phi_n ) = \varphi(n)$.
    \item
      Ist~$\zeta \in \Ein_n(K)$ primitiv, so gilt~$K( \Ein_n ) = K( \zeta )$.
    \item
      Es gilt~$[ K(\Ein_n) : K ] = \varphi(n)$.
  \end{enumerate}
\end{proposition}

\begin{proposition}
  Es sei~$K = \Rational$.
  \begin{enumerate}
    \item
      Das Kreisteilungspolynom~$\Phi$ ist normiert, und es gilt bereits~$\Phi \in \Integer[X]$.
    \item
      Das Kreisteilungspolnom~$\Phi$ ist irreduzibel.
    \item
      Ist~$p$ prim, so gilt~$\Phi_p = X^{p-1} + X^{p-2} + \dotsb + X + 1$.
    \item
      Ist~$L$ ein beliebiger Körper, so ergibt sich das~\te{$n$} Kreisteilungspolynom über~$L$ aus dem~\ten{$n$} Kreisteilungspolynom über~$\Rational$ durch Anwenden des Ringhomomorphismus~$\Integer[X] \to K[X]$, der durch den eindeutigen Ringhommorphismus~$\Integer \to K$ induziert wird.

      Gilt insbesondere~$\ringchar(L) = 0$ (ist also~$\Rational$ ein Unterkörper von~$L$), so sind die~\ten{$n$} Kreisteilungspolynome über~$L$ und über~$\Rational$ bereits gleich.
  \end{enumerate}
\end{proposition}

\begin{remark}
  Für~$K = \Rational$ lassen sich die Kreisteilungspolynome induktiv durch Polynomdivision berechnen.
\end{remark}

\begin{corollary}
  Es sei~$K = \Rational$.
  \begin{enumerate}
    \item
      Für jede zwei primitive~\te{$n$} Einheitswurzeln~$\zeta, \xi \in \Ein_n$ gibt es ein eindeutiges Element~$\varphi$ der Galois-Gruppe~$\Gal( \Rational(\Ein_n) / \Rational )$ mit~$\varphi(\zeta) = \xi$.
    \item
      Es gilt~$\Gal( \Phi_n ) = \Gal( \Rational(\Ein_n) / \Rational ) \cong (\Integer/n)^\times$.
  \end{enumerate}
\end{corollary}




