\chapter*{Wichtige Sätze und Definitionen}
\chead{\upshape Wichtige Sätze und Definitionen}

Für die Klausur sollte man die folgenden Sätze auswendig können, und zwar in genau der Formulierung, wie sie im offiziellen Vorlesungs-Skript erscheinen.
So wurden etwa die makierten Einträge in früheren Klausuren abgefragt.
\begin{itemize}
  \item
    \textbf{Artinscher Fortsetzungssatz}~(S.~134/135)
  \item
    Bahnenformel
  \item
    \textbf{Dritter Sylow-Satz}
  \item
    Eisenstein-Kriterium (S.~126)
  \item
    Erster Noetherscher Isomorphiesatz für Gruppen
  \item
    \textbf{Erster Sylow-Satz}
  \item
    Fundamentalsatz der Algebra
  \item
    Homomorphiesatz für Gruppen
  \item
    Homomorphiesatz für Ringe
  \item
    Lemma von Gauß (S.~124)
  \item
    Noetherscher Isomorphiesatz für Ringe.
  \item
    \textbf{Satz vom primitiven Element}
  \item
    Satz von Abel-Ruffini
  \item
    Satz von Cauchy (über endliche Gruppen)
  \item
    Satz von Cayley
  \item
    \textbf{Satz von Gauß über die Konstruierbarkeit des regelmäßigen~\Ecks{$n$}}
  \item
    \textbf{Satz von Gauß über faktorielle Ringe}~(S.~124)
  \item
    Satz von Lagrange
  \item
    Reduktionskriterium (S.~128)
  \item
    Universelle Eigenschaft der Lokalisierung
  \item
    Universelle Eigenschaft des Polynomrings
  \item
    Zweiter Noetherscher Isomorphiesatz für Gruppen.
  \item
    \textbf{Zweiter Sylow-Satz}
\end{itemize}

Selbiges gilt für die Definitionen der folgenden Begriffe.
Auch hier wurden die makierten Einträge in früheren Klausuren schon einmal abgefragt.
\begin{itemize}
  \item
    abgeleitete Untergruppen
  \item
    \textbf{algebraisch abgeschlossen}
  \item
    algebraisches Element
  \item
    algebraische Körpererweiterung
  \item
    \textbf{algebraischer Abschluss eines Körpers}
  \item
    \textbf{auflösbare Gruppe}
  \item
    Bahn einer Gruppenaktion
  \item
    einfache Gruppe
  \item
    Einheitengruppe
  \item
    euklidischer Ring
  \item
    Eulersche~\Funktion{$\varphi$}
  \item
    \textbf{faktorieller Ring}
  \item
    Frobenius-Homomorphismus (S.~146/147)
  \item
    Galois-Gruppe einer Körpererweiterung
  \item
    Galois-Gruppe eines Polynoms (S.~141)
  \item
    Gruppenaktion
  \item
    Hauptidealring
  \item
    Integritätsring
  \item
    irreduzibles Element
  \item
    Kommutatoruntergruppe
  \item
    Körper der rationalen Funktionen
  \item
    Lokalisierung
  \item
    maximales Ideal
  \item
    multiplikativ abgeschlossene Teilmenge eines kommutativen Rings
  \item
    Nebenklasse
  \item
    Normalisator
  \item
    normale Hülle (S.~143)
  \item
    \textbf{normale Körpererweiterung}
  \item
    Primelement
  \item
    \textbf{Primideal}
  \item
    \textbf{primitives Element (einer Körpererweiterung)}
  \item
    primitives Polynom
  \item
    Primkörper
  \item
    separabler Abschluss
  \item
    separables Element
  \item
    separables Polynom
  \item
    separable Körpererweiterung
  \item
    Stabilisator
  \item
    \textbf{Zerfällungskörper eines Polynoms} (S.~139)
  \item
    Zentralisator
  \item
    Zentrum
\end{itemize}
