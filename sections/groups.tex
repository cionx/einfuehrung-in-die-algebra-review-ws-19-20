\chapter{Gruppentheorie}

\begin{convention}
  Im Folgenden ist~$G$ eine Gruppe, sofern nicht anders angegeben.
\end{convention}

\section{Allgemeines}

\begin{definition}
  Es seien~$G$,~$H$ zwei Gruppen.
  Eine Abbildung~$\varphi \colon G \to H$ ist ein \defemph{Gruppenhomomorphismus} falls
  \[
    \varphi(g_1 g_2) = \varphi(g_1) \spacing \varphi(g_2)
    \qquad
    \text{für alle~$g_1, g_2 \in G$} \,.
  \]
\end{definition}

\begin{definition}
  Eine Teilmenge~$H$ von~$G$ ist eine \defemph{Untergruppe} falls die folgenden drei Bedingungen erfüllt sind:
  \begin{itemize}
    \item
      Es gilt~$1 \in H$.
    \item
      Für je zwei Elemente~$h_1, h_2 \in H$ gilt auch~$h_1 h_2 \in H$.
    \item
      Für jedes Element~$h \in H$ gilt auch~$h^{-1} \in H$.
  \end{itemize}
  Dass~$H$ eine Untergruppe von~$G$ ist, wird mit~$H \subgroupeq G$ notiert.
\end{definition}

Ist~$H$ eine Untergruppe von~$G$, so lässt sich die Multiplikation von~$G$ zu einer Multiplikation auf~$H$ einschränken.
Hierdurch wird~$H$ ebenfalls zu einer Gruppe.

\begin{example}
  \leavevmode
  \begin{enumerate}
    \item
      Sowohl~$G$ also auch~$1$ sind Untergruppen von~$G$.%
      \footnote{
        Wir bezeichnen mit~$1$ die triviale Gruppe, bzw. die triviale Untergruppe.
      }
    \item
      Die Untergruppen von~$\Integer$ sind genau die Teilmengen~$n \Integer = \{na \suchthat a \in \Integer\}$ für~$n \in \Integer$.
    \item
      Es sei~$\varphi \colon G \to H$ ein Gruppenhomomorphismus.
      Der~\defemph{Kern} von~$\varphi$ ist
      \[
        \ker(\varphi)
        \defined
        \{
          g \in G
        \suchthat
          \varphi(g) = 1
        \} \,,
      \]
      und das \defemph{Bild} von~$\varphi$ ist
      \[
        \im(\varphi)
        \defined
        \{
          \varphi(g)
        \suchthat
          g \in G
        \} \,.
      \]
      Es ist~$\im(\varphi)$ eine Untergruppe von~$H$ und~$\ker(\varphi)$ eine Untergruppe von~$G$.
      Der Homomorphismus~$\varphi$ ist genau dann surjektiv, wenn~$\im(\varphi) = H$, und genau dann injektiv, wenn~$\ker(\varphi) = 1$.
  \end{enumerate}
\end{example}

\begin{convention}
  Im Folgenden ist~$H$ eine Untergruppe von~$G$, sofern nicht anders angegeben.
\end{convention}





\section{Ordnungen und Indizes}

\begin{definition}
  Die \defemph{Ordnung} von~$G$ ist~$\card{G} \in \Natural_{1} \cup \{ \infty \}$.
\end{definition}



\subsection{Nebenklassen und der Satz von Lagrange}

\begin{definition}
  Es sei~$g$ ein Element von~$G$.
  \begin{enumerate}
    \item
      Die \defemph{Linksnebenklasse} von~$g$ bzgl.~$H$ ist die Teilmenge~$gH = \{ gh \suchthat h \in H \}$ von~$G$.
    \item
      Die \defemph{Rechtsnebenklasse} von~$g$ bzgl.~$H$ ist die Teilmenge~$Hg = \{ gh \suchthat h \in H \}$ von~$G$.
    \item
      Die Menge der Links- bzw. Rechtsnebenklassen in~$G$ bzgl.~$H$ sind
      \[
        G/H \defined \{ gH \suchthat g \in G \}
        \quad\text{und}\quad
        H \backslash G \defined \{ Hg \suchthat g \in G \} \,.
      \]
  \end{enumerate}
\end{definition}

\begin{example}
  Für~$G = \Integer$ und~$H = 2 \Integer$ sind die Nebenklassen von~$G$ bzgl.~$H$ die Menge der geraden Zahlen und die Menge der ungeraden Zahlen.
\end{example}

\begin{proposition}
  Die Abbildung
  \[
    G/H
    \to
    H \backslash G \,,
    \quad
    gH
    \mapsto
    (gH)^{-1}
    =
    Hg^{-1}
  \]
  ist eine wohldefinierte Bijektion.
\end{proposition}

\begin{corollary}
  Die beiden Mengen~$G/H$ und~$H \backslash G$ sind gleichmächtig.
\end{corollary}

\begin{definition}
  Der \defemph{Index} von~$H$ in~$G$ ist~$[G : H] \defined \card{ G/H } = \card{ H \backslash G }$.
\end{definition}

\begin{proposition}
  \leavevmode
  \begin{enumerate}
    \item
      Die Gruppe~$G$ ist die disjunkte Vereinigung der Nebenklassen bzgl.~$H$.
    \item
      Für je zwei Elemente~$g_1, g_2 \in G$ gilt genau dann~$g_1 H = g_2 H$ wenn~$g_1^{-1} g_2 \in H$.
    \item
      Für jedes Element~$g \in G$ ist die Abbildung~$H \to gH$,~$h \mapsto gh$ bijektiv, weshalb~$\card{gH} = \card{H}$ gilt.
  \end{enumerate}
\end{proposition}

\begin{corollary}[Satz von Lagrange]
  Es gilt~$\card{G} = \card{H} \cdot [G : H]$.
\end{corollary}

\begin{corollary}
  Ist~$G$ endlich, so sind~$\card{H}$ und~$[G : H]$ jeweils Teiler von~$\card{G}$.
\end{corollary}



\subsection{Erzeugte Untergruppen}

\begin{convention}
  Im Folgenden ist~$S$, sofern nicht anders angegeben, eine Teilmenge von~$G$.
\end{convention}

\begin{definition}
  \leavevmode
  \begin{enumerate}
    \item
      Die von~$S$ \defemph{erzeugte Untergruppe} von~$G$ ist die Menge
      \[
        \gen{S}
        \defined
        \bigl\{
          s_1^{\varepsilon_1} \dotsm s_n^{\varepsilon_n}
        \suchthat[\big]
          n \in \Natural,
          s_1, \dotsc, s_n \in S,
          \varepsilon_1, \dotsc, \varepsilon_n \in \{1, -1\}
        \bigr\}
      \]
    \item
      Die Menge~$S$ ist ein \defemph{Erzeugendensystem} von~$G$ falls~$G = \gen{S}$.
  \end{enumerate}
\end{definition}

\begin{proposition}
  \leavevmode
  \begin{enumerate}
    \item
      Die von~$S$ erzeugte Untergruppe~$\gen{S}$ ist eine Untergruppe von~$G$.
    \item
      Es ist~$\gen{S}$ die kleinste Untergruppe von~$G$, die~$S$ enthält:
      Es gilt~$S \subseteq \gen{S}$, und für jede Untergruppe~$H$ von~$G$ mit~$S \subseteq H$ gilt auch~$\gen{S} \subseteq H$.
    \item
      Es gilt~$\gen{S} = \bigcap \{ H \suchthat H \subgroupeq G, S \subseteq H \}$.
  \end{enumerate}
\end{proposition}


\begin{example}
  \leavevmode
  \begin{enumerate}
    \item
      Die symmetrische Gruppe~$\Sym_n$ wird von der Menge der Transpositionen erzeugt, also von
      \[
        S_1
        \defined
        \{
          (i, j)
        \suchthat
          1 \leq i < j \leq n
        \}
      \]
      Die Menge der einfachen Transpositionen ist ebenfalls ein Erzeugendensystem, also
      \[
        S_2
        \defined
        \{
          (i, i+1)
        \suchthat
          i = 1, \dotsc, n-1
        \} \,.
      \]
    \item
      Ist~$K$ ein Körper und~$n \in \Natural$, so wird die Gruppe~$\GL(n,K)$ von der Menge der Elementarmatrizen erzeugt.
  \end{enumerate}
\end{example}


\subsection{Zyklische Gruppen}

\begin{definition}
  Die Gruppe~$G$ ist \defemph{zyklisch} wenn sie von einem einzelnen Element erzeugt wird, d.h. falls es ein Element~$g \in G$ mit~$G = \gen{g}$ gibt.
\end{definition}

\begin{example}
  \leavevmode
  \begin{enumerate}
    \item
      Für jedes Gruppenelement~$g \in G$ ist die erzeugte Untergruppe~$\gen{g}$ zyklisch.
    \item
      Die Gruppe~$\Integer$ ist zyklisch.
      Die möglichen Erzeuger sind~$1$ und~$-1$.
    \item
      Für jedes~$n \in \Natural_1$ ist die Gruppe~$\Integer_n$ zyklisch.
      Das Element~$\class{1}$ ist ein zyklischer Erzeuger.
      Allgemeiner ist~$\class{k}$ genau dann ein zyklischer Erzeuger von~$\Integer_n$, wenn~$n$ und~$k$ teilerfremd sind.
    \item
      Ist~$K$ ein Körper, so ist jede endliche Untergruppe der multiplikativen Gruppe~$K^\times$ zyklisch.
      Inbesondere ist~$K^\times$ zyklisch, falls~$K$ endlich ist.
  \end{enumerate}
\end{example}

\begin{theorem}[Klassifikation zyklischer Gruppen]
  Ist~$G$ eine zyklische Gruppe, so gilt
  \[
    G
    \cong
    \begin{cases*}
        \Integer
        &
        falls~$\card{G} = \infty$,
        \\
        \Integer_n
        &
        falls~$\card{G} = n < \infty$.
    \end{cases*}
  \]
\end{theorem}

\begin{corollary}
  Ist~$K$ ein endlicher Körper mit~$\card{K} = q$, so gilt~$K^\times \cong \Integer_{q-1}$.
\end{corollary}

\begin{proposition}
  Die Gruppe~$G$ sei endlich und~$\card{G}$ sei prim.
  Dann ist~$G$ zyklisch, und jedes Element~$g \in G$ mit~$g \neq 1$ ist ein zyklischer Erzeuger von~$G$.
\end{proposition}

\begin{corollary}
  Für jede Primzahl~$p$ ist~$\Integer_p$ bis auf Isomorphie die einzige Gruppe von Ordnung~$p$.
\end{corollary}



\subsection{Die Ordnung eines Elements}

\begin{definition}
  Die \defemph{Ordnung} eines Elements~$g$ von~$G$ ist
  \[
    \ord(g)
    \defined
    \inf
    {}
    \{
      n \in \Natural
    \suchthat
      n \geq 1,
      g^n = 1
    \}
    \in
    \Natural_1 \cup \{ \infty \} \,.
  \]
\end{definition}

\begin{remark}
  Es gilt also~$\ord(g) = \infty$ falls es kein~$n \in \Natural_1$ mit~$g^n = 1$ gibt, und ansonsten ist~$\ord(g)$ das minimale solche~$n$.
\end{remark}

\begin{proposition}
  Für jedes Gruppenelement~$g \in G$ gilt~$\ord(g) = \card{\gen{g}}$.
\end{proposition}

\begin{corollary}
  Ist~$G$ endlich, so ist für jedes Gruppenelement~$g \in G$ auch die Ordnung~$\ord(g)$ endlich, und es gilt~$\ord(g) \divides \card{G}$.
\end{corollary}

\begin{corollary}
  Ist~$G$ endlich, so gilt für die Ordnung~$n \defined \card{G}$, dass~$g^n = 1$ für jedes Element~$g \in G$.
\end{corollary}





\section{Gruppenaktionen}



\subsection{Grundlegendes}

\begin{definition}
  Es sei~$G$ eine Gruppe.
  \begin{enumerate}
    \item
      Eine \defemph{Gruppenaktion}, bzw. \defemph{Aktion} von~$G$ auf einer Menge~$X$  ist eine Abbildung
      \[
        G \times X \to X \,,
        \quad
        (g,x)
        \mapsto
        g.x
      \]
      welche die folgenden beiden Eigenschaften erfüllt:
      \begin{itemize}
        \item
          Es gilt~$1.x = x$ für jedes Element~$x \in X$.
        \item
          Es gilt~$g.(h.x) = (gh).x$ für alle Gruppenelemente~$g, h \in G$ und jedes Element~$x \in X$. 
      \end{itemize}
    \item
      Eine~\defemph{\Menge{$G$}} ist eine Menge~$X$ zusammen mit einer Aktion von~$G$ auf~$X$.
  \end{enumerate}
\end{definition}

\begin{definition}
  Es seien~$X$,~$Y$ zwei~\Mengen{$G$}.
  Eine Abbildung~$\varphi \colon X \to Y$ ist ein~\defemph{\Homomorphismus{$G$}} falls
  \[
    \varphi(g.x) = g.\varphi(x)
    \qquad
    \text{für alle~$g \in G$,~$x \in X$.}
  \]
\end{definition}



\subsection{Gruppenaktionen als Gruppenhomomorphismen}

Ist~$X$ eine~\Menge{$G$}, so ist für jedes Gruppenelement~$g \in G$ die Abbildung
\[
  \lambda_g
  \colon
  X \to X \,,
  \quad
  x \mapsto g.x
\]
eine Bijektion (mit~$\lambda_g^{-1} = \lambda_{g^{-1}}$), und die entstehende Abbildung
\[
  \lambda
  \colon
  G \to \Sym_X \,,
  \quad
  g \mapsto \lambda_g
\]
ist ein Gruppenhomomorphismus.
Ist umgekehrt~$\varphi \colon G \to \Sym(X)$ ein Gruppenhomomorphismus, so wird durch
\[
  g.x \defined \varphi(g)(x)
  \qquad
  \text{für alle~$g \in G$,~$x \in X$}
\]
eine Aktion von~$G$ auf~$X$ definiert.
Diese beiden Konstruktionen sind inverse zueinander, und wir erhalten das folgende Resultat.

\begin{proposition}
  Es sei~$G$ eine Gruppe und~$X$ eine Menge.
  Dann gibt es eine Eins-zu-eins-Korrespondenz gegeben durch
  \begin{align*}
    \left\{
      \begin{tabular}{@{}c@{}}
        Gruppenaktionen\\
        von~$G$ auf~$X$
      \end{tabular}
    \right\}
    &\onetoone
    \left\{
      \begin{tabular}{@{}c@{}}
        Gruppenhomomorphismen\\
        $G \to \Sym_X$
      \end{tabular}
    \right\} \,,
    \\
    *
    &\mapsto
    [g \mapsto [x \mapsto g*x]]
    \\
    [(g,x) \mapsto \varphi(g)(x)]
    &\mapsfrom
    \varphi \,.
  \end{align*}
\end{proposition}



\subsection{Bahnen und Stabilisatoren}

\begin{convention}
  Im Folgenden ist~$X$ eine~\Menge{$G$}, sofern nicht anders angegeben.
\end{convention}

\begin{definition}
  Es sei~$x \in X$.
  \begin{enumerate}
    \item
      Die \defemph{Bahn} von~$x$ ist die Teilmenge~$G.x$ von~$X$ gegeben durch~$G.x \defined \{ g.x \suchthat g \in G \}$.
    \item
      Der \defemph{Stabilisator} von~$x$ ist die Teilmenge~$G_x$ von~$G$ gegeben durch~$G_x \defined \{ g \in G \suchthat g.x = x \}$.
  \end{enumerate}
\end{definition}

\begin{proposition}
  Für jedes Element~$x \in X$ ist der Stabilisator~$G_x$ eine Untergruppe von~$G$.
\end{proposition}

\begin{example}
  \leavevmode
  \begin{enumerate}
    \item
      Die Gruppe~$G$ operiert auf einer beliebigen Menge~$X$ vermöge~$g.x = x$ für alle~$g \in G$,~$x \in X$.
      (Diese Gruppenaktion entspricht dem trivialen Gruppenhomomorphismus~$G \to \Sym_X$.)
      Für diese Aktion sind Bahnen sind einelementig, und alle Stabilisatoren sind ganz~$G$.
    \item
      Die symmetrische Gruppe~$\Sym_n$ operiert auf der Menge~$X = \{1, \dotsc, n\}$ vermöge
      \[
        \sigma . x \defined \sigma(x)
        \qquad
        \text{für alle~$\sigma \in \Sym_n$,~$x \in X$.}
      \]
      Für jedes Element~$x \in X$ gelten~$G.x = X$ und~$G.x = \{ \sigma \in \Sym_n \suchthat \sigma(x) = x \} \cong \Sym_{n-1}$.
    \item
      Die Gruppe~$G \defined \SOrth(n)$ der~$(n \times n)$\nobreakdash-\hspace{0pt}Rotationsmatrizen operiert auf~$\Real^n$ durch
      \[
        A.x \defined Ax
        \qquad
        \text{für alle~$A \in \SOrth(n)$,~$x \in \Real^n$.}
      \]
      Für jedes~$x \in \Real^n$ gilt~$G.x = \{ y \in \Real^n \suchthat \abs{y} = \abs{x} \}$.
      Es ist~$G_x$ die Gruppe der Rotationen der zu~$x$ orthogonalen Hyperebene, also~$G_x \cong \SOrth(n-1)$.
    \item
      Die Untergruppe~$H$ von~$G$ operiert auf~$G$ wahlweise durch
      \[
        h.g \defined hg \,,
        \quad
        h.g \defined g h^{-1} \,.
      \]
      Die Bahnen dieser Gruppenaktionen sind die Rechts- bzw. Linksnebenklassen in~$G$ bzgl.~$H$.
      Die Stabilisatoren sind jeweils trivial.
  \end{enumerate}
\end{example}

Für je zwei Gruppenelement~$g_1, g_2 \in G$ und jedes Element~$x \in X$ gelten die Äquivalenzen
\[
  g_1.x = g_2.x
  \iff
  g_2^{-1} g_1.x = x
  \iff
  g_2^{-1} g_1 \in G_x
  \iff
  g_1 G_x = g_2 G_x \,.
\]
Hieraus erhalten wir das folgende Resultat.

\begin{corollary}
  \label{isomorphism theorem for G-sets}
  Für jedes Element~$x \in X$ ist die Abbildung
  \[
    G/G_x \to G.x \,,
    \quad
    g G_x \mapsto g.x
  \]
  ein wohldefinierter, bijektiver~\Homomorphismus{$G$}.
\end{corollary}

\begin{corollary}
  Für jedes Element~$x \in X$ gilt~$\card{G.x} = [G : G_x]$.
\end{corollary}

\begin{corollary}
  Ist~$G$ endlich, so ist für jedes Element~$x \in X$ die Kardinalität~$\card{G.x}$ ein Teiler der Gruppenordnung~$\card{G}$.
\end{corollary}

% TODO: Beispiele.

\begin{example}
  Gilt~$\card{G} = 35$ und ist~$X$ eine~\Menge{$G$}, so haven die~\Bahnen{$G$} von~$X$ die möglichen Kardinalitäten
  \[
    1 \,,
    \quad
    5 \,,
    \quad
    7 \,,
    \quad
    35 \,.
  \]
  Gilt~$\card{X} = 9$, so besteht~$X$ aus mindestens drei~\Bahnen{$G$}.
\end{example}


\subsection{Transitivität}

\begin{proposition}
  \label{characterizations of transitive actions}
  Es sei~$X \neq \emptyset$.
  Die folgenden Bedingungen für~$X$ sind äquivalent.
  \begin{equivlist}
    \item
      Es gibt für alle Elemente~$x_1, x_2 \in X$ ein Gruppenelement~$g \in G$ mit~$g.x_1 = x_2$.
    \item
      Für jedes Element~$x \in X$ gilt~$G.x = X$.
    \item
      Es gibt ein Element~$x \in X$ mit~$G.x = X$.
    \item
      Die Menge~$X$ besteht aus genau einer~\Bahn{$G$}.
  \end{equivlist}
\end{proposition}

\begin{definition}
  Eine Gruppenaktion von~$G$ auf~$X$ ist \defemph{transitiv}, falls sie die äquivalenten Bedingungen aus \cref{characterizations of transitive actions} erfüllt.
\end{definition}

\begin{warning}
  Ob die eindeutige Gruppenaktion von~$G$ auf der leeren Menge~$\emptyset$ transitiv ist, ist Geschmackssache.
  In dieser Vorlesung ist diese Wirkung \emph{nicht} transitiv.
\end{warning}

\begin{example}
  \leavevmode
  \begin{enumerate}
    \item
      Die Wirkung der symmetrischen Gruppe~$\Sym_n$ auf der Menge~$\{1, \dotsc, n\}$ durch~$\sigma.x \defined \sigma(x)$ ist transitiv.
    \item
      Es gelte~$\card{G} = 40$ und~$\card{X} = 25$.
      Es sei~$x \in X$.
      Wäre die Aktion von~$G$ auf~$X$ transitiv, so würde
      \[
        25
        =
        \card{X}
        =
        \card{G.x}
        \divides
        \card{G}
        =
        40
      \]
      gelten.
      Dies ist nicht der Fall.
  \end{enumerate}
\end{example}



\subsection{Fixpunkte}

\begin{definition}
  \leavevmode
  \begin{enumerate}
    \item
      Ein Element~$x \in X$ ist ein \defemph{Fixpunkt}, falls~$g.x = x$ für jedes Gruppenelement~$g \in G$ gilt.
    \item  
      Die Menge der Fixpunkte von~$X$ wird mit~$X^G$ bezeichnet.
  \end{enumerate}
\end{definition}

\begin{proposition}
  Für ein Element~$x \in X$ sind die folgenden Bedingungen äquivalent.
  \begin{equivlist}
    \item
      Das Element~$x$ ist ein Fixpunkt.
    \item
      Die Bahn~$G.x$ ist gegeben durch~$G.x = \{ x \}$.
    \item
      Der Stabilisator~$G_x$ ist gegeben durch~$G_x = G$.
  \end{equivlist}
  Insbesondere sind Fixpunkte durch ihre Bahnen, und auch durch ihre Stabilisatoren charakterisiert.
\end{proposition}



\subsection{Die Bahnenformel}

\begin{proposition}
  Die~\Menge{$G$}~$X$ ist die disjunkte Vereinigung ihrer~\Bahnen{$G$}.
\end{proposition}

\begin{corollary}
  Die~\Menge{$G$}~$X$ sei endlich, und es seien~$x_1, \dotsc, x_n$ ein Repräsentantensystem der~\Bahnen{$G$} von~$X$.
  Es gilt
  \[
    \card{X}
    =
    \sum_{i=1}^n \card{G.x_i}
    =
    \card*{X^G}
    +
    \sum_{\substack{i=1, \dotsc, n \\ x_i \notin X^G}}
    \card{G.x_i} \,.
  \]
\end{corollary}

\begin{theorem}[Bahnenformel]
  Die~\Menge{$G$}~$X$ sei endlich, und es seien~$x_1, \dotsc, x_n$ ein Repräsentatensystem der~\Bahnen{$G$} von~$X$.
  Es gilt
  \[
    \card{X}
    =
    \sum_{i=1}^n [G : G_{x_i}]
    =
    \card*{X^G}
    +
    \sum_{\substack{i=1, \dotsc, n \\ x_i \notin X^G}}
    [G : G_{x_i}] \,.
  \]
\end{theorem}



\section{Konjugation}

\begin{definition}
  Es sei~$g \in G$.
  \begin{enumerate}
    \item
      Die Abbildung~$G \to G$,~$x \mapsto g x g^{-1}$ ist die \defemph{Konjugation} mit~$g$.
    \item
      Die \defemph{Konjugationsklasse} eines Gruppenelements~$x \in G$ ist die Menge~$x^G \defined \{ g x g^{-1} \suchthat g \in G \}$.
%    \item
%      Die \defemph{Konjugationsklasse} einer Untergruppe~$H$ von~$G$ ist die Menge~$\{ g H g^{-1} \suchthat g \in G \}$.
  \end{enumerate}
\end{definition}

\begin{proposition}
  Für jedes Grupenelement~$g \in G$ ist die Konjugationsabbildung
  \[
    c_g
    \colon
    G \to G \,
    \quad
    x \mapsto g x g^{-1}
  \]
  ein Gruppenautomorphismus.
  Es gilt~$c_g^{-1} = c_{g^{-1}}$.
\end{proposition}

\begin{proposition}
  Die Gruppe~$G$ ist die disjunkte Vereinigung ihrer Konjugationsklassen.
\end{proposition}

\begin{definition}
  Zwei Gruppenelemente~$x_1$,~$x_2$ von~$G$ sind \defemph{konjugiert} zueinander, falls sie in der gleichen Konjugationsklasse liegen, d.h. falls es ein Gruppenelement~$g \in G$ mit~$x_2 = g x_1 g^{-1}$ gibt.
\end{definition}



\subsection{Zentrum und Zentralisatioren}

\begin{definition}
  \leavevmode
  \begin{enumerate}
    \item
      Das \defemph{Zentrum} von~$G$ ist die Menge~$\zenter(G) \defined \{ g \in G \suchthat \text{$g h = h g$ für alle~$h \in G$} \}$.
    \item
      Der \defemph{Zentralisator} eines Gruppenenelements~$g \in G$ ist die Menge~$\zentralizer_G(g) \defined \{ h \in G \suchthat gh = hg \}$.
  \end{enumerate}
\end{definition}

\begin{proposition}
  \leavevmode
  \begin{enumerate}
    \item
      Das Zentrum~$\zenter(G)$ ist eine Untergruppe von~$G$.
    \item
      Für jedes Gruppenelement~$g \in G$ ist der Zentralisator~$\zentralizer_G(g)$ eine Untergruppe von~$G$.
    \item
      Es gilt~$\zenter(G) = \bigcap_{g \in G} \zentralizer_G(g)$.
  \end{enumerate}
\end{proposition}



\subsection{Anwendung der Bahnenformel}

Die Gruppe~$G$ operiert auf sich selbst durch Konjugation, d.h. vermöge
\[
  g.x = g x g^{-1}
  \qquad
  \text{für alle~$g \in G$,~$x \in G$.}
\]
Die Bahnen dieser Gruppenaktion sind genau die Konjugationsklassen von~$G$.
Der Stabilisator eines Elements~$x \in G$ ist genau der Zentralisator~$\zentralizer_G(x)$.
Die Fixpunkte sind genau die Element des Zentrums.
Aus der Bahnenformel ergibt sich somit das folgende Resultat.

\begin{theorem}[Klassenformel]
  Die Gruppe~$G$ sei endlich, und seien~$x_1, \dotsc, x_n$ ein Repräsentantensystem der Konjugationsklassen von~$G$.
  Es gilt
  \[
    \card{G}
    =
    \sum_{i=1}^n [G : \zentralizer_G(x_i)]
    =
    \card{ \zenter(G) }
    +
    \sum_{\substack{i = 1, \dotsc, n \\ x_i \notin \zenter(G)}}
    [G : \zentralizer_G(x_i)] \,.
  \]
\end{theorem}



\section{Normalteiler}

\begin{proposition}
  \label{characterizations of normal subgroups}
  Für eine Untergruppe~$N$ von~$G$ sind die folgenden Bedingungen äquivalent:
  \begin{equivlist}
    \item
      Es gilt~$g N g^{-1} = N$ für jedes Gruppenelement~$g \in G$.
    \item
      Es gilt~$g N g^{-1} \subseteq N$ für jedes Gruppenelement~$g \in G$.
    \item
      Es gilt~$gN = Ng$ für jedes Gruppenelement~$g \in G$.
  \end{equivlist}
\end{proposition}

\begin{definition}
  Eine Untergruppe~$N$ von~$G$ ist \defemph{normal}, bzw. ein \defemph{Normalteiler}, falls sie die äquivalenten Bedingungen aus \cref{characterizations of normal subgroups} erfüllt.
  Dass~$N$ ein Normalteiler ist, wird mit~$N \normgroupeq G$ notiert.
\end{definition}

\begin{example}
  \leavevmode
  \begin{enumerate}
    \item
      Ist die Gruppe~$G$ abelsch, so ist jede Untergruppe von~$G$ normal.
    \item
      Das Zentrum~$\zenter(G)$ ist ein Normalteiler in~$G$.
    \item
      Ist~$\varphi \colon G \to G'$ ein Gruppenhomomorphsimus, so ist~$\ker(\varphi)$ ein Normalteiler in~$G$.
      Hierdurch ergeben sich die folgenden Sonderfälle, wobei~$K$ jeweils ein Körper ist.
      \begin{enumerate}
        \item
          Die alternierende Gruppe~$\Alt_n \defined \{ \sigma \in S_n \suchthat \sgn(\sigma) = 1 \}$ ist eine normale Untergruppe der symmetrischen Gruppe~$\Sym_n$.
        \item
          Die spezielle lineare Gruppe~$\SL(n,K) \defined \{ A \in \GL(n,K) \suchthat \det(A) = 1 \}$ ist eine normale Untergruppe der allgemeinen linearen Gruppe~$\GL(n,K)$.
        \item
          Die spezielle orthogonale Gruppe~$\SOrth(n,K) \defined \{ A \in \Orth(n,K) \suchthat \det(A) = 1 \}$ ist eine normale Untergruppe der orthogonalen Gruppe~$\Orth(n,K)$.
      \end{enumerate}
    \item
      Ist~$N$ eine Untergruppe von~$G$ mit~$[G : N] = 2$, so ist~$N$ ein Normalteiler in~$G$.%
      \footnote{
        Es sei allgemeiner~$G$ eine endliche Gruppe und~$N$ eine Untergruppe von~$G$, so dass der Index~$[G : N]$ der kleinste Primfaktor von~$\card{G}$ ist.
        Dann ist~$N$ normal in~$G$.
      }
  \end{enumerate}
\end{example}

\begin{convention}
  Im Folgenden ist~$N$ ein Normalteiler in~$G$, sofern nicht anders angegeben.
\end{convention}




%\subsection{Normalisatoren}
%
%\begin{definition}
%  Der \defemph{Normalisator} von~$H$ (in~$G$) ist die Menge~$\normalizer_G(H) \defined \{ g \in G \suchthat g H g^{-1} = H \}$.
%\end{definition}
%
%\begin{proposition}
%  Es seien~$H$,~$K$ zwei Untergruppen von~$G$.
%  Es ist~$H$ genau dann einen normale Untergruppe von~$K$, wenn~$H \subgroupeq K \subgroupeq \normalizer_G(H)$.
%\end{proposition}
%
%\begin{corollary}
%  
%\end{corollary}
%
%Es sei~$X$ die Menge der zu $H$ konjugierten Untergruppen.
%Die Gruppe~$G$ operiert auf der Menge~$X$ vermöge
%\[
%  g.H' \defined g H g^{-1} \,.
%\]
%Es gilt~$H \in X$, und die Bahn von~$H$ ist ganz~$X$.
%Der Stabilisator von~$H$ ist genau der Normalisator~$\normalizer_G(H)$.
%Wir erhalten somit aus \cref{isomorphism theorem for G-sets} das folgende Resultat.
%
%\begin{corollary}
%  Die Abbildung
%  \[
%    G / {\normalizer_G(H)}
%    \to
%    \{
%      \text{zu~$H$ konjugierte Untergruppen von~$G$}
%    \} \,,
%    \quad
%    g {\normalizer_G(H)}
%    \mapsto
%    g H g^{-1}
%  \]
%  ist eine wohldefinierte Bijektion.
%\end{corollary}
%
%\begin{corollary}
%  Ist~$G$ endlich, so gilt die Anzahl der zu~$H$ konjugierten Untergruppen von~$G$ durch den Index~$[G : \normalizer_G(H)]$ gegeben.
%\end{corollary}





\section{Faktorgruppen}



\subsection{Konstruktion}

Für jedes Gruppenelement~$g \in G$ sei~$\class{g} \defined gN$.
Die Multiplikation
\[
  \class{g_1} \cdot \class{g_2}
  \defined
  \class{g_1 g_2}
\]
ist eine wohldefinierte Gruppenstruktur auf~$G/N$, und die Abbildung
\[
  \pi
  \colon
  G \to G/N \,,
  \quad
  g \mapsto \class{g}
\]
ist ein Gruppenhomomorphismus mit~$\ker(\pi) = N$.

\begin{definition}
  In der obigen Situation ist~$G/N$ die \defemph{Faktorgruppe} von~$G$ nach~$N$, und die Abbildung~$\pi$ ist die \defemph{kanonische Projektion}.
\end{definition}



\subsection{Korrespondenz von Untergruppen}

\begin{theorem}
  Es sei~$\pi \colon G \to G/N$ die kanonische Projektion.
  \begin{enumerate}
    \item
      Es gibt eine Eins-zu-eins-Korrespondenz von Untergruppen, gegeben durch
      \begin{align*}
        \left\{
          \begin{tabular}{@{}c@{}}
            Untergruppen von~$G$,\\
            die~$N$ enthalten
          \end{tabular}
        \right\}
        &\onetoone
        \{ \text{Untergruppen von~$G/N$} \}
        \\
        H
        &\mapsto
        H/N \,,
        \\
        \pi^{-1}(H')
        &\mapsfrom
        H' \,.
      \end{align*}
    \item
      In der obigen Korrespondenz ist eine Untergruppe~$H$ von~$G$ genau dann ein Normalteiler in~$G$, wenn~$H/N$ ein Normalteiler von~$G/N$ ist.
      Durch Einschränkung der obigen Korrespondenz ergibt sich daher die Eins-zu-eins-Korrespondenz
      \[
        \left\{
          \begin{tabular}{@{}c@{}}
            normale Untergruppen von~$G$,\\
            die~$N$ enthalten
          \end{tabular}
        \right\}
        \onetoone
        \{ \text{normale Untergruppen von~$G/N$} \} \,.
      \]
  \end{enumerate}
\end{theorem}



\subsection{Homomorphiesatz und Isomorphiesätze}

\begin{theorem}[Homomorphiesatz]
  Es sei~$\pi \colon G \to G/N$ die kanonische Projektion, und es sei~$H$ eine weitere Gruppe.
  Ein Gruppenhomomorphismus~$\varphi \colon G \to H$ induziert genau dann einen Gruppenhomorphismus~$\induced{\varphi} \colon G/N \to H$, der das Diagramm
  \[
    \begin{tikzcd}
      G
      \arrow{r}[above]{\varphi}
      \arrow{d}[left]{\pi}
      &
      H
      \\
      G/N
      \arrow[dashed]{ur}[below right]{\induced{\varphi}}
      &
      {}
    \end{tikzcd}
  \]
  zum kommutieren bringt, wenn~$N \subseteq \ker(\varphi)$ gilt.
  Es gilt dann
  \[
    \induced{\varphi}( \class{g} )
    =
    \varphi(g)
    \qquad
    \text{für alle~$g \in G$,}
  \]
  sowie~$\im(\induced{\varphi}) = \im(\varphi)$ und~$\ker(\induced{\varphi}) = \ker(\varphi)/N$.
  Insbesondere ist der Gruppenhomomorphismus~$\induced{\varphi}$ eindeutig.
%  \begin{enumerate}
%    \item
%      Es gilt~$\ker(\pi) = N$.
%    \item
%      Ist~$\psi \colon G/N \to H$ ein Gruppenhomomorphismus, so gilt für den Gruppenhomomorphismus~$\psi \circ \pi \colon G \to H$, dass~$N \subseteq \ker(\psi \circ \pi)$.
%    \item
%      Ist andererseits~$\varphi \colon G \to H$ ein Gruppenhomomorphismus mit~$N \subseteq \ker(\varphi)$, so ist
%      \[
%        \induced{\varphi}
%        \colon
%        G/N \to H \,,
%        \quad
%        \class{g} \mapsto \varphi(g)
%      \]
%      ein wohldefinierter Gruppenhomorphisms.
%      Dies ist der eindeutige Gruppenhomomorphismus, der das folgende Diagramm zum kommutieren bringt:
%      \[
%        \begin{tikzcd}
%          G
%          \arrow{r}[above]{\varphi}
%          \arrow{d}[left]{\pi}
%          &
%          H
%          \\
%          G/N
%          \arrow[dashed]{ur}[below right]{\induced{\varphi}}
%          &
%          {}
%        \end{tikzcd}
%      \]
%      Dabei gilt~$\im(\induced{\varphi}) = \im(\varphi)$ sowie~$\ker(\induced{\varphi}) = \ker(\varphi)/N$.
%    \item
%      Die beiden obigen Konstruktionen sind invers zueinander, und liefern somit eine Eins-zu-eins-Korrespondenz gegeben durch
%      \begin{align*}
%        \left\{
%          \begin{tabular}{@{}c@{}}
%            Gruppenhomomorphismen \\
%            $\varphi \colon G \to H$ mit~$N \subseteq \ker(\varphi)$
%          \end{tabular}
%        \right\}
%        &\onetoone
%        \left\{
%          \begin{tabular}{@{}c@{}}
%            Gruppenhomomorphismen \\
%            $\psi \colon G/N \to H$
%          \end{tabular}
%        \right\}
%        \\
%        \varphi
%        &\mapsto
%        \induced{\varphi} \,,
%        \\
%        \psi \circ \pi
%        &\mapsfrom
%        \psi \,.
%      \end{align*}
%  \end{enumerate}
\end{theorem}

\begin{corollary}
  Es sei~$H$ eine weiter Gruppe und~$\varphi \colon G \to H$ ein Gruppenhomomorphismus.
  Dann induziert~$\varphi$ einen Gruppenisomorphismus
  \[
    \induced{\varphi}
    \colon
    G/{\ker(\varphi)} \to \im(\varphi) \,,
    \quad
    \class{g}
    \mapsto
    \varphi(g) \,.
  \]
\end{corollary}

\begin{corollary}[Noethersche Isomorphiesätze]
  \leavevmode
  \begin{enumerate}
    \item
      Es seien~$H$,~$N$ Untergruppen von~$G$, wobei~$N$ ein Normalteiler in~$G$ ist.
      \begin{enumerate}
        \item
          Das Produkt~$HN = \{ hn \suchthat h \in H, n \in N \}$ ist eine Untergruppe von~$G$.
        \item
          Es ist~$N$ ein Normalteiler in~$HN$ und~$H \cap N$ ein Normalteiler in~$H$.
        \item
          Die Inklusion~$H \to HN$ induziert einen Gruppenisomorphismus
          \[
            H / (H \cap N)
            \mapsto
            HN / N \,,
            \quad
            \class{h}
            \mapsto
            \class{h} \,.
          \]
%          Die beiden Gruppen beschreiben gegenüberliegende Seiten des folgenden Diagramms:
%          \[
%            \begin{tikzcd}
%              {}
%              &
%              HN
%              &
%              {}
%              \\
%              N
%              \arrow[very thick, dash]{ur}
%              &
%              {}
%              &
%              H
%              \arrow[dash]{ul}
%              \\
%              {}
%              &
%              H \cap N
%              \arrow[dash]{ul}
%              \arrow[very thick, dash]{ur}
%              &
%              {}
%            \end{tikzcd}
%          \]
      \end{enumerate}
    \item
      Es seien~$N$,~$K$ zwei Normalteiler in~$G$ mit~$K \subseteq N \subseteq G$.
      \begin{enumerate}
        \item
          Es ist~$N$ normal in~$K$, und es ist~$K/N$ normal in~$G/N$.
        \item
          Die kanonische Projektion~$G \to G/K$ induziert einen wohldefinierten Gruppenisomorphismus
          \[
            (G/N) / (K/N)
            \to
            G / K \,,
            \quad
            \class{ \class{g} }
            \mapsto
            \class{g} \,.
          \]
          Beide Gruppen beschreiben im folgenden Diagramm den obersten Teil:
          \[
            \begin{tikzcd}
              G
              \\
              K
              \arrow[very thick, dash]{u}
              \\
              N
              \arrow[dash]{u}
              \\
              1
              \arrow[dash]{u}
            \end{tikzcd}
          \]
      \end{enumerate}
  \end{enumerate}
\end{corollary}





\clearpage





\section{\texorpdfstring{$p$}{p}-Gruppen und Sylowsätze}

\begin{convention}
  Im Folgenden ist~$p$ eine Primzahl, sofern nicht anders angegeben.
\end{convention}

\begin{definition}
  Eine~\defemph{\Gruppe{$p$}} ist eine endliche Gruppe~$G$, so dass~$\card{G} = p^n$ für ein~$n \in \Natural$ gilt.
\end{definition}

\begin{example}
  \leavevmode
  \begin{enumerate}
    \item
      Die triviale Gruppe~$1$ ist eine~\Gruppe{$p$}.
    \item
      Die Gruppen~$\Integer_{p^2}$ und~$\Integer_p \times \Integer_p$ sind~\Gruppen{$p$}.%
      \footnote{
        Dies sind tätsächlich schon die einzigen Gruppen von Ordnung~$p^2$.
      }
    \item
      Die Heisenberg-Gruppe
      \[
        G
        \defined
        \left\{
          \begin{bsmallmatrix}
            1 & a & b \\
            0 & 1 & c \\
            0 & 0 & 1
          \end{bsmallmatrix}
        \suchthat*
          a, b, c \in \Finite_p
        \right\}
      \]
      ist eine nicht-abelsche~\Gruppe{$p$} von Ordnung~$p^3$.
  \end{enumerate}
\end{example}

\begin{proposition}
  Es sei~$G$ eine~\Gruppe{$p$} und~$X$ eine endliche~\Menge{$G$}.
  Es gilt
  \[
    \card{X} \equiv \card{X^G} \pmod{p} \,.  
  \]
\end{proposition}

\begin{corollary}
  Ist~$G$ eine nicht-triviale~\Gruppe{$p$}, so gilt~$\zenter(G) \neq 1$.
\end{corollary}

\begin{definition}
  Es sei~$G$ eine endliche Gruppe, und es sei~$\card{G} = p^r m$ mit~$p \ndivides m$.
  Eine Untergruppe~$P$ von~$G$ der Ordnung~$\card{P} = p^r$ ist eine~\defemph{\Sylow{$p$}-Untergruppe}.
\end{definition}

\begin{theorem}[Erster Sylowsatz]
  Es sei~$G$ eine endliche Gruppe, und es sei~$\card{G} = p^r m$ mit~$p \ndivides m$.
  \begin{enumerate}
    \item
      Für jede Potenz~$s = 0, \dotsc, r$ enthält~$G$ eine Untergruppe der Ordnung~$p^s$.
      Insbesondere (für~$s = r$) enthält~$G$ eine~\Sylow{$p$}-Untergruppe.
    \item
      Für jede Untergruppe~$H$ von~$G$ der Ordnung~$p^s$ mit~$0 \leq s \leq r-1$ gibt es eine Untergruppe~$K$ von~$G$ der Ordnung~$p^{s+1}$, so dass~$H$ eine normale Untergruppe von~$K$ ist.
  \end{enumerate}
\end{theorem}

\begin{theorem}[Zweiter Sylowsatz]
  Es sei~$G$ eine endliche Gruppe.
  \begin{enumerate}
    \item
      Es sei~$H$ eine~\Untergruppe{$p$} von~$G$ und~$P$ eine~\Sylow{$p$}-Untergruppe von~$G$.
      Dann gibt es ein Gruppenelement~$g \in G$, so dass~$g H g^{-1} \subseteq P$ gilt.
    \item
      Je zwei~\Sylow{$p$}-Untergruppen von~$G$ sind konjugiert zueinander.
  \end{enumerate}
\end{theorem}

\begin{corollary}
  Eine~\Sylow{$p$}-Untergruppe~$P$ von~$G$ ist genau dann eine normale Untergruppe von~$G$, wenn~$P$ die einzige~\Sylow{$p$}-Untergruppe von~$G$ ist.
\end{corollary}

\begin{theorem}[Dritter Sylowsatz]
  Es sei~$G$ eine endliche Gruppe, und es sei~$\card{G} = p^r m$ mit~$p \ndivides m$.
  Es sei~$n_p$ die Anzahl der~\Sylow{$p$}-Untergruppen von~$G$.
  \begin{enumerate}
    \item
      Es gilt~$n_p \divides \card{G}$.
    \item
      Es gilt~$n_p \equiv 1 \Mod{p}$.
  \end{enumerate}
\end{theorem}

\begin{example}[Erstklausur 19/20]
  Es sei~$G$ eine endliche Gruppe mit~$\card{G} = 30$.
  Dann gibt es für eine passende Primzahl~$p$ eine nicht-triviale~\Sylow{$p$}-Untergruppe~$N$ von~$G$, die normal in~$G$ ist:

  Es gilt~$30 = 2 \cdot 3 \cdot 5$.
  Für jede Primzahl~$p \notin \{ 2, 3, 5 \}$ ist deshalb die triviale Untergruppe~$1$ die eindeutige~\Sylow{$p$}-Untergruppe von~$G$.
  Für alle Primzahlen~$p \in \{2, 3, 5\}$ gilt~$n_p \divides \card{G} = 30$, und somit
  \[
    n_p \in \{ 1, 2, 3, 5, 6, 10, 15, 30 \} \,.
  \]
  Außerdem gilt~$n_p \equiv 1 \Mod{p}$.
  Hieraus ergibt sich, dass
  \[
    n_2 \in \{ 1, 3, 5, 15 \} \,,
    \quad
    n_3 \in \{ 1, 10 \} \,,
    \quad
    n_5 \in \{ 1, 6 \} \,.
  \]
  Wir wollen zeigen, dass~$n_3 = 1$ oder~$n_5 = 1$ gilt.
  Dann gibt es eindeutige~\Sylow{$3$}-Untergruppe, bzw.~\Sylow{$5$}-Untergruppe in~$G$, und diese Sylow-Untergruppe ist dann normal.
  (Denn Konjugate von~\Sylow{$p$}-Untergruppen sind wieder~\Sylow{$p$}-Untergruppen.)
  
  Wir nehmen an, dass~$n_3, n_5 \neq 1$ gilt.
  Dann gilt~$n_3 = 10$ und~$n_5 = 6$.

  Da~$n_3 = 10$ gilt, gibt es zehn~\Sylow{$3$}-Untegruppen~$P_1, \dotsc, P_{10}$ in~$G$.
  Für alle~$i = 1, \dotsc, n$ gilt~$\card{P_i} = 3$ und somit~$P_i \cong \Integer_3$.
  Somit enthält jedes~$P_i$ genau zwei Element von Ordnung~$3$.
  Für~$i \neq j$ gilt~$P_i \nsubseteq P_j$, weshalb~$P_i \cap P_j$ eine echte Untergruppe von~$P_i$ ist.
  Da~$P_i$ prim ist, ist dann~$P_i \cap P_j = 1$.
  Deshalb haben die~$P_i$ keine gemeinsamen nicht-trivialen Element.
  Wir somit deshalb, dass die Gruppe~$G$ mindestens~$20$ Element von Ordnung~$3$ enthält.

  Wir erhalten auf gleiche Weise aus der Annahme~$n_5 = 6$, dass die Gruppe~$G$ mindestens~$24$ Elemente von Ordnung~$5$ enthält.

  Es gilt aber~$20 + 24 = 44 > 30 = \card{G}$, weshalb wir einen Widerspruch erhalten.
  Es muss somit~$n_3 = 1$ oder~$n_5 = 1$ gelten.

\end{example}


