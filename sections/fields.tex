\chapter{Körpertheorie}

\begin{definition}
  Es sei~$L$ ein Körper.
  \begin{enumerate}
    \item
      Eine Teilmenge~$K$ von~$L$ ist ein \defemph{Unterkörper} falls~$K$ ein Unterring von~$L$ ist, der ebenfalls ein Körper ist.
      Es ist dann~$L$ eine \defemph{Körpererweiterung} von~$K$.
      Diese Körpererweiterung wird mit~$L/K$ notiert.
    \item
      Ist~$Z$ ein Unterkörper von~$L$ mit~$K \subseteq Z$, so ist~$Z$ ein \defemph{Zwischenkörper} der Erweiterung~$L/K$.
  \end{enumerate}
\end{definition}

\begin{convention}
  Im Folgenden ist~$L/K$ eine Körpererweiterung, sofern nicht anders angegeben.
\end{convention}





\section{Allgemeines}



\subsection{Der Grad einer Körpererweiterung}

\begin{definition}
  \leavevmode
  \begin{enumerate}
    \item
      Der \defemph{Grad} der Körpererweiterung~$L/K$ ist~$[L : K] \defined \dim_K(L) \in \Natural_1 \cup \{ \infty \}$.
    \item
      Die Körpererweiterung~$L/K$ ist \defemph{endlich} wenn ihr Grad~$[L : K]$ endlich ist, wenn also~$L$ endlichdimensional als~\Vektorraum{$K$} ist.
  \end{enumerate}
\end{definition}

\begin{example}
  Es gelten
  \[
    [\Complex : \Real] = 2 \,,
    \quad
    [\Real : \Rational] = \infty \,,
    \quad
    [\Complex : \Rational] = \infty \,.
  \]
\end{example}

\begin{theorem}[Gradformel]
  Für Körpererweiterungen~$M/L/K$ gilt~$[M : K] = [M : L] [L : K]$.
\end{theorem}

\begin{corollary}
  Die Erweiterung~$L/K$ sei endlich und~$Z$ sei ein Zwischenkörper dieser Erweiterung.
  Dann sind die beiden Grade~$[L : Z]$ und~$[Z : K]$ jeweils Teiler des Grades~$[L : K]$.
\end{corollary}

\begin{example}
  Ist~$L/K$ endlich und~$[L : K]$ prim ist, so gilt für jeden Zwischenkörper~$Z$ dieser Körpererweiterung bereits~$Z = K$ oder~$Z = L$.
\end{example}



\subsection{Erzeugte Zwischenkörper}

\begin{definition}
  \leavevmode
  \begin{enumerate}
    \item
      Es sei~$A$ eine Teilmenge von~$L$.
      Dann ist
      \[
        K(A)
        \defined
        \bigcap
        \left\{
          Z
        \suchthat*
          \begin{tabular}{@{}c@{}}
             $Z$ ist ein Zwischenkörper\\
             der Erweiterung~$L/K$ mit~$A \subseteq Z$
          \end{tabular}
        \right\}
      \]
      der von~$A$ \defemph{erzeugte Zwischenkörper} der Körpererweiterung~$L/K$.
    \item
      Die Körpererweiterung~$L/K$ wird von der Menge~$A$ \defemph{erzeugt}, und~$A$ ist ein \defemph{Erzeugendensystem} der Erweiterung~$L/K$, falls~$L = K(A)$ gilt.
  \end{enumerate}
\end{definition}

\begin{proposition}
  Es sei~$A$ eine Teilmenge von~$L$.
  \begin{enumerate}
    \item
      Die Menge~$K(A)$ ist ein Zwischenkörper der Erweiterung~$L/K$.
    \item
      Der Körper~$K(A)$ der kleinste Zwischenkörper der Erweiterung~$L/K$, der die Menge~$A$ enthält:
      Es gilt~$A \subseteq K(A)$, und ist~$Z$ ein Zwischenkörper von~$L/K$ mit~$A \subseteq Z$, so gilt bereits~$K(A) \subseteq Z$.
  \end{enumerate}
\end{proposition}

\begin{definition}
  Die Körperweiterung~$L/K$ ist \defemph{einfach}, falls es ein Element~$\alpha \in L$ gibt, so dass~$L = K(\alpha)$ gilt.
\end{definition}




\section{Algebraizität}

Für eine Familie~$(\alpha_i)_{i \in I}$ von Elementen~$\alpha_i \in K$ ist
\[
  K[\alpha_i \suchthat \in I]
  \defined
  \im( \ev_{(\alpha_i)_{i \in I}} )
\]
die Menge aller Polynomausdrücke in den~$\alpha_i$, mit Koeffienten in~$K$.

\begin{example}
  \leavevmode
  \begin{enumerate}
    \item
      Für die imaginäre Einheit~$i \in \Complex$ gilt~$i^2 = -1$ und somit
      \[
        \Rational[i]
        =
        \{
          f(i)
        \suchthat
          f \in \Rational[X]
        \}
        =
        \{
          a + b i
        \suchthat
          a, b \in \Rational
        \} \,.
      \]
    \item
      Für die Elemente~$i, \sqrt{2} \in \Complex$ gilt
      \[
        \Rational[i, \sqrt{2}]
        =
        \{
          f(i, \sqrt{2})
        \suchthat
          f \in \Rational[X, Y]
        \}
        =
        \{
          a + i b + \sqrt{2} c + i \sqrt{2} d
        \suchthat
          a, b, c, d \in \Rational
        \} \,.
      \]
  \end{enumerate}
\end{example}


\subsection{Algebraische Elemente}

\begin{proposition}
  \label{characterizations of algebraic elements}
  Für jedes Element~$\alpha$ von~$L$ sind die folgenden Bedingungen äquivalent.
  \begin{equivlist}
    \item
      Die Körpereweiterung~$K(\alpha)/K$ ist endlich.
    \item
      Es gilt~$K[\alpha] = K(\alpha)$.
    \item
      Es ist~$K[\alpha]$ ein Körper.
    \item
      Es gilt~$\alpha^{-1} \in K[\alpha]$.
    \item
      Es gibt ein Polynom~$f \in K[X]$ mit~$f \neq 0$, so dass~$f(\alpha) = 0$.
  \end{equivlist}
\end{proposition}

\begin{definition}
  Es sei~$\alpha$ ein Element von~$K$.
  \begin{enumerate}
    \item
      Das Element~$\alpha$ ist \defemph{algebraisch} über~$K$, falls es die Bedingungen von \cref{characterizations of algebraic elements} erfüllt.
    \item
      Ist das Element~$\alpha$ nicht algebraisch über~$K$, so ist es \defemph{transzendent} über~$K$.
  \end{enumerate}
\end{definition}

\newpage

\begin{example}
  \leavevmode
  \begin{enumerate} 
    \item
      Ist~$\alpha \in \Complex$ eine Wurzel einer rationalen Zahl, d.\,h. gilt~$\alpha^n \in \Rational$ für eine passende Potenz~$n \in \Natural_1$, so ist~$\alpha$ algebraisch über~$\Rational$.
    \item
      Für die Körpererweiterung~$\Real/\Rational$ enthält~$\Real$ abzählbar viele algebraische Elemente, und überabzählbar viele transzendente Elemente.
    \item
      Für die Erweiterung~$K(t)/K$ (wobei~$K(t) = \Quot(K[t])$) ist~$t$ transzendent über~$\Rational$.
      Tatsächlich ist bereits jedes Element~$f \in K(t)$ mit~$f \notin K$ transzendent über~$K$.
  \end{enumerate}
\end{example}

Es sei~$\alpha$ ein Element von~$L$, das algebraisch über~$K$ ist.
Dann ist der Einsetzhomomorphisums
\[
  \ev_\alpha
  \colon
  K[X] \to K(\alpha) \,,
  \quad
  f \mapsto f(\alpha)
\]
ein surjektiver Ringhomomorphismus.
Aus der Endlichkeit der Körpererweiterung~$K(\alpha)/K$ folgt, dass der Homomorphismus~$\ev_\alpha$ nicht injektiv ist, und somit~$\ker(\ev_\alpha) \neq 0$ gilt.
Da~$K[X]$ ein Hauptidealring ist, gibt es ein eindeutiges normiertes Polynom~$\mu_\alpha \in K[X]$ mit~$\ker(\ev_\alpha) = \genideal{ \mu_\alpha }$.
Dabei gilt insbesondere~$\mu_\alpha \neq 0$.

\begin{definition}
  In der obigen Situation ist~$\mu_\alpha$ das \defemph{Minimalpolynom} von~$\alpha$ über~$K$.
\end{definition}

\begin{proposition}
  Es sei~$\alpha$ ein Element von~$\alpha$, das algebraisch über~$K$ ist.
  Es sei~$n \defined \deg(\mu_\alpha)$.
  \begin{enumerate}
    \item
      Der Einsetzhomomorphismus~$\ev_\alpha \colon K[X] \to K(\alpha)$ induziert einen wohldefinierten Ringisomorphismus~$K[X] / \genideal{\mu_\alpha} \to K(\alpha)$ gegeben durch~$\class{f} \mapsto f(\alpha)$.
    \item
      Das Minimalpolynom~$\mu_\alpha$ ist irreduzibel.
    \item
      Es ist~$\mu_\alpha$ das eindeutige normierte, irreduzible Polynom in~$K[X]$, das~$\alpha$ als Nullstelle hat.
    \item
      Die Elemente~$1, \alpha, \dotsc, \alpha^{n-1}$ sind eine~\Basis{$K$} von~$K(\alpha)$.
    \item
      Es gilt~$[K(\alpha) : K] = \deg(\mu_\alpha)$.
  \end{enumerate}
\end{proposition}

\begin{example}
  Für das Element~$\alpha \defined \sqrt[5]{3} \in \Real$ gilt~$f(\alpha) = 0$ für das Polynom~$f = X^5 - 3 \in \Rational[X]$.
  Nach dem Eisenstein-Kriterium (mit~$p = 3$) ist das Polynom~$f$ irreduzibel.
  Es ist außerdem normiert.
  Also ist~$f$ das Minimalpolynom von~$\alpha$ über~$\Rational$.
  Es folgt, dass~$[\Rational(\alpha) : \Rational] = \deg(f) = 5$, und dass die Elemente
  \[
    1 \,,
    \quad
    \sqrt[5]{3} \,,
    \quad
    \sqrt[5]{3}^2 \,,
    \quad
    \sqrt[5]{3}^3 \,,
    \quad
    \sqrt[5]{3}^4
  \]
  eine~\Basis{$\Rational$} von~$\Rational(\alpha)$ bilden.
\end{example}




\clearpage





\subsection{Algebraische Körpererweiterungen}
\newday{4}

\begin{definition}
  Die Menge~$\Alg_{L/K} \defined \{ \alpha \in L \suchthat \text{$\alpha$ ist algebraisch über~$K$} \}$ ist der \defemph{algebraische Abschluss} von~$K$ in~$L$.
\end{definition}

\begin{definition}
  \leavevmode
  \begin{enumerate}
    \item
      Die Erweiterung~$L/K$ ist \defemph{algebraisch} falls jedes Element von~$L$ algebraisch über~$K$ ist.
    \item
      Ist die Erweiterung~$L/K$ nicht algebraisch, so ist sie \defemph{transzendent}.
  \end{enumerate}
\end{definition}

\begin{warning}
  Dass die Erweiterung~$L/K$ transzendent ist, bedeutet, dass~$L$ ein Element enthält, das transzendent über~$K$ ist.
  Das schließt nicht aus, dass~$L$ auch Elemente enthält, die algebraisch über~$K$ sind.
  So sind etwa alle Elemente~$\alpha \in K$ algebraisch über~$K$;
  es kann aber auch Elemente~$\alpha \in L$ mit~$\alpha \notin K$ geben, die algebraisch über~$K$ sind.
\end{warning}

\begin{remark}
  Es lässt sich zeigen, dass es für jede Körpererweiterung~$L/K$ einen Zwischenkörper~$Z$ gibt, so dass~$Z \cong K(X_i \suchthat i \in I)$ für eine passende Indexmenge~$I$ gilt, und die Erweiterung~$L/Z$ nur noch algebraisch ist.
  Die Kardinalität der Menge~$I$ ist dabei eindeutig, und wird der \defemph{Transzendenzgrad} von~$L/K$ genannt.
\end{remark}

\begin{lemma}
  Es seien~$M/L/K$ Körpererweiterungen und es sei~$\alpha$ ein Element von~$M$.
  Ist das Element~$\alpha$ algebraisch über~$L$ und die Erweiterung~$L/K$ algebraisch, so ist das Element~$\alpha$ auch algebraisch über~$K$.
\end{lemma}

\begin{proposition}
  Für Körpererweiterungen~$M/L/K$ ist die Erweiterung~$M/K$ genau dann algebraisch, wenn die Erweiterungen~$M/L$ und~$L/K$ beide algebraisch sind.
\end{proposition}

\begin{proposition}
  Für die Körpererweiterung~$L/K$ sind die folgenden Bedingungen äquivalent.
  \begin{enumerate}
    \item
      Die Erweiterung~$L/K$ ist endlich.
    \item
      Die Erweiterung ist algebraisch und wird von endlich vielen Elementen erzeugt.
    \item
      Die Erweiterung~$L/K$ wird von endlich vielen algebraischen Elementen erzeugt.
  \end{enumerate}
  Insbesondere ist jede endliche Körpererweiterung algebraisch.
\end{proposition}

\begin{example}
  Die Elemente~$\sqrt[5]{2}, \sqrt[4]{3} \in \Complex$ sind beide algebraisch über~$\Rational$.
  Also ist die Erweiterung~$\Rational(\sqrt[5]{2}, \sqrt[4]{3}) / \Rational$ endlich, und jedes Element von~$\Rational(\sqrt[5]{2}, \sqrt[4]{3})$ ist algebraisch über~$\Rational$.
  Da diese Erweiterung von endlich vielen algebraischen Elementen erzeugt wird, ist sie bereits endlich;
  tatsächlich gilt~$[\Rational(\sqrt[5]{2}) : \Rational(\sqrt[4]{3})] = 20$.
\end{example}

\begin{corollary}
  Es ist~$\Alg_{L/K}$ ein Zwischenkörper der Erweiterung~$L/K$.
\end{corollary}

\begin{example}
  Der \defemph{Körper der algebraischen Zahlen} ist
  \[
    \{
      z \in \Complex
    \suchthat
      \text{$z$ ist Nullstelle eines rationalen Polynoms~$f \neq 0$}
    \} \,.
  \]
  Dies ist genau~$\Alg_{\Complex / \Rational}$, also ein Zwischenkörper der Erweiterung~$\Complex / \Rational$.
\end{example}

\begin{corollary}
  Die Körpererweiterung~$L/K$ ist genau dann algebraisch, wenn sie von algebraischen Elementen erzeugt wird.
\end{corollary}

\begin{example}
  Es sei~$L \defined \Rational( \sqrt[n]{2} \suchthat n \in \Natural_1)$.
  Die Erweiterung~$L/\Rational$ ist unendich, da
  \[
    [L : \Rational]
    \geq
    [\Rational(\sqrt[n]{2}) : \Rational]
    =
    n
  \]
  für alle~$n \in \Natural_1$ gilt.
  Diese Erweiterung ist auch algebraisch, da~$\sqrt[n]{2}$ für jedes~$n \in \Natural_1$ algebraisch über~$\Rational$ ist.
\end{example}



\subsection{Algebraische Abschlüsse}

\begin{proposition}
  \label{characterizations of algebraically closed field}
  Für einen Körper~$K$ sind die folgenden Bedingungen äquivalent.
  \begin{equivlist}
    \item
      Jedes nicht-konstante Polynom~$f \in K[X]$ hat eine Nullstelle in~$K$.
    \item
      Die irreduziblen Polynome in~$K[X]$ sind genau die Polynome vom Grad~$1$.
    \item
      Jedes normierte Polynom~$f \in K[X]$ zerfällt in Linearfaktoren.
    \item
      Für jede algebraische Körpererweiterung~$L/K$ gilt~$L = K$.
  \end{equivlist}
\end{proposition}

\begin{definition}
  Es sei~$K$ ein Körper.
  \begin{enumerate}
    \item
      Ein Körper ist \defemph{algebraisch abgeschlossen} falls er die äquivalenten Bedingungen aus \cref{characterizations of algebraically closed field} erfüllt.
    \item
      Ein \defemph{algebraischer Abschluss} von~$K$ ist ein algebraisch abgeschlossener Körper~$\closure{K}$, so dass~$\closure{K}$ eine algebraische Körpererweiterung von~$K$ ist.
  \end{enumerate}
\end{definition}

\begin{example}
  \leavevmode
  \begin{enumerate}
    \item
      Der \defemph{Fundamentalsatz der Algebra besagt}, dass der Körper der komplexen Zahlen,~$\Complex$, algebraisch abgeschlossen ist.
      Es handelt such um einen algebraischen Abschluss von~$\Real$.
    \item
      Ein algebraischer Abschluss von~$\Rational$ ist gegeben durch~$\closure{\Rational} \defined \Alg_{\Complex/\Rational}$.
  \end{enumerate}
\end{example}

\begin{lemma}
  Ist~$L/K$ algebraisch und~$K$ abzählbar, so ist auch~$L$ abzählbar.
\end{lemma}

\begin{corollary}
  Ist~$K$ abzählbar und~$\closure{K}$ ein algebraischer Abschluss von~$K$, so ist auch~$\closure{K}$ abzählbar.
\end{corollary}

Ist~$f \in K[X]$ ein irreduzibles Polynom, so ist~$L \defined K[X]/\genideal{f}$ eine Körpererweiterung, in der das Polynom~$f$ eine Nullstelle hat (nämlich die Restklasse~$\class{X}$).
Auf diese Weise lassen sich zu dem Körper~$K$ Nullstelllen von (irreduziblen) Polynomen \defemph{dazuadjungieren}.

Durch ein geschicktes, gleichzeitiges Dazuadjungieren von Nullstellen für alle nicht-\hspace{0pt}konstanten Polynome in~$K[X]$ ergibt sich das folgende Resultat.

\begin{theorem}[Steinitz]
  Für jeden Körper~$K$ gibt einen algebraischen Abschluss~$\closure{K}$ von~$K$.
\end{theorem}





\clearpage





\section{\texorpdfstring{$K$}{K}-Homomorphismen}

\begin{proposition}
  Jeder Körperhomomorphismus ist injektiv.
\end{proposition}

\begin{remark}
  Ist~$\varphi \colon K \to L$ ein Körperhomomorphismus, so können wir also~$K$ als Unterkörper von~$L$ auffassen, indem wir~$K$ mit seinem Bild~$\varphi(K)$ identifizieren.
\end{remark}

\begin{lemma}
  Ist~$L/K$ eine Körpererweiterung, so gilt~$\ringchar(K) = \ringchar(L)$.
\end{lemma}

\begin{proposition}
  Zwischen Körpern von unterschiedlicher Charakteristik gibt es keine Körperhomomorphismen.
\end{proposition}



\subsection{Definitionen}

\begin{convention}
  Im Folgende sind~$L$,~$L_1$,~$L_2$ jeweils Körpererweiterungen von~$K$, sofern nicht anders angegeben.
\end{convention}

\begin{definition}
  Ein~\linearer{$K$} Körperhomomorphismus~$\varphi \colon L_1 \to L_2$ ist ein~\defemph{\Homomorphismus{$K$}}.
\end{definition}

\begin{remark}
  \leavevmode
  \begin{enumerate}
    \item
      Ein Körperhomomorphismus~$\varphi \colon L_1 \to L_2$ ist genau dann ein~\Homomorphismus{$K$} (also~\linear{$K$}), wenn~$\restrict{\varphi}{K} = \id_K$ gilt.
      Dies ist außerdem äquivalent dazu, dass~$\varphi$ das folgende Diagramm zum kommutieren bringt:
      \[
        \begin{tikzcd}[column sep = small]
          {}
          &
          {}
          &
          L_2
          \\
          L_1
          \arrow[dashed]{urr}[above]{\varphi}
          &
          {}
          &
          {}
          \\
          {}
          &
          K
          \arrow{ul}
          \arrow{uur}
          &
          {}
        \end{tikzcd}
      \]
    \item
      Wenn es einen~\Homomorphismus{$K$}~$L_1 \to L_2$ gibt, so können wir~$L_1$ als einen Unterkörper von~$L_2$ auffassen, und erhalten somit Körpererweiterungen~$L_2/L_1/K$.
  \end{enumerate}
\end{remark}

\begin{proposition}
  Es sei~$\varphi \colon L_1 \to L_2$ ein~\Homomorphismus{$K$}.
  Ist~$\alpha \in L_1$ eine Nullstelle eines Polynoms~$f \in K[X]$, so ist auch~$\varphi(\alpha)$ eine Nullstelle von~$f$.
\end{proposition}

\begin{corollary}
  \label{galois on elements}
  Es sei~$\alpha \in L_1$ algebraisch über~$K$ mit Minimalpolynom~$\mu_\alpha$.
  Ist~$\varphi \colon L_1 \to L_2$ ein~\Homomorphismus{$K$}, so ist~$\varphi(\alpha)$ eine Nullstelle on~$\mu_\alpha$.
\end{corollary}

\begin{proposition}
  \label{galois uniqueness}
  Es gelte~$L = (\alpha_i \suchthat i \in I)$.
  Dann ist jeder~\Homomorphismus{$K$}~$\varphi \colon L_1 \to L_2$ bereits eindeutig durch die Funktionswerte~$\varphi(\alpha_i)$ für~$i \in I$ bestimmt.
\end{proposition}

\begin{corollary}
  Ist~$L_1/K$ endlich, so gibt es nur endlich viele~\Homomorphismen{$K$}~$L_1 \to L_2$.
\end{corollary}



\subsection{Fortsetzungssätze}

\begin{theorem}[Artinscher Fortsetzungssatz]
  Es sei~$\varphi \colon K \to L$ ein Körperhomomorphismus, und es sei~$\varphi_* \colon K[X] \to L[X]$ der induzierte Ringhomomorphismus, der durch koeffizientenweises Anwenden von~$\varphi$ entsteht.

  Es sei~$K(\alpha)/K$ eine einfache, algebraische Körpererweiterung, und es~$\mu_\alpha$ das Minimalpolynom von~$\alpha$ über~$K$.
  Dann gibt es für jede Nullstelle~$\beta$ von~$\varphi_*(\mu_\alpha)$ in~$L$ eine eindeutige Fortsetzung von~$\varphi$ zu einem Körperhomomorphismus~$\psi \colon K(\alpha) \to L$ mit~$\psi(\alpha) = \beta$.
  \[
    \begin{tikzcd}[column sep = huge]
      K(\alpha)
      \arrow[dashed]{r}[above]{\varphi' \colon \alpha \mapsto \beta}
      &
      L
      \\
      K
      \arrow{u}
      \arrow{ur}[below right]{\varphi}
      &
      {}
    \end{tikzcd}
  \]
\end{theorem}

\begin{corollary}
  \label{artin extension theorem version 2}
  Es sei~$K(\alpha)/K$ eine einfache, algebraische Körpererweiterung und~$L/K$ eine beliebige Körpererweiterung.
  Dann gibt es für jede Nullstelle~$\beta$ von~$\mu_\alpha$ in~$L$ einen eindeutigen~\Homomorphismus{$K$}~$\psi \colon K(\alpha) \to L$ mit~$\psi(\alpha) = \beta$.
  \[
    \begin{tikzcd}[column sep = huge]
      K(\alpha)
      \arrow[dashed]{r}[above]{\psi \colon \alpha \mapsto \beta}
      &
      L
      \\
      K
      \arrow{u}
      \arrow{ur}
      &
      {}
    \end{tikzcd}
  \]
\end{corollary}

\begin{corollary}
  \label{galois existence}
  In der Situation von \cref{artin extension theorem version 2} erhalten wir eine Bijektion
  \begin{align*}
    \left\{
      \begin{tabular}{c}
        \Homomorphismen{$K$}\\
        $\varphi \colon K(\alpha) \to L$
      \end{tabular}
    \right\}
    &\to
    \{
      \text{Nullstellen von~$\mu_\alpha$ in~$L$}
    \} \,,
    \\
    \varphi
    &\mapsto
    \varphi(\alpha) \,.
  \end{align*}
\end{corollary}

\begin{theorem}
  Die Körpererweiterung~$L/K$ sei algebraisch, und es sei~$\closure{K}$ ein algebraischer Abschluss von~$K$.
  Ist~$Z$ ein Zwischenkörper von~$L/K$ und ist~$\varphi \colon Z \to \overline{K}$ ein~\Homomorphismus{$K$}, so gibt es eine Fortsetzung von~$\varphi$ zu einem~\Homomorphismus{$K$}~$\varphi' \colon L \to \overline{K}$.
  \[
    \begin{tikzcd}
      L
      \arrow[dashed]{r}[above]{\varphi'}
      &
      \closure{K}
      \\
      Z
      \arrow{u}
      \arrow{ur}[below right]{\varphi}
      &
      {}
      \\
      K
      \arrow{u}
      \arrow[bend right]{uur}
      &
      {}
    \end{tikzcd}
  \]
\end{theorem}

\begin{corollary}
  Ist~$\closure{K}$ ein algebraischer Abchluss von~$K$ und~$L/K$ eine algebraische Körpererweiterung, so gibt es einen~\Homomorphismus{$K$}~$L \to \closure{K}$.
  Insbesondere können wir~$L$ als einen Zwischenkörper der Erweiterung~$\closure{K} / K$ auffassen.
\end{corollary}

\begin{corollary}
  Sind~$L_1$ und~$L_2$ zwei algebraische Abschlusse von~$K$, so gibt es einen~\Isomorphismus{$K$}~$L_1 \to L_2$.
\end{corollary}



\subsection{Galois-Gruppen}

\begin{definition}
  Die \defemph{Galois-Gruppe} der Körpererweiterung~$L/K$ ist die Gruppe der~\Automorphismen{$K$} von~$L$ und sie wird mit~$\Gal(L/K)$ notiert.
\end{definition}

\begin{proposition}
  \label{homos are autos for finite extension}
  Ist die Erweiterung~$L/K$ endlich, so ist jeder~\Homomorphismus{$K$}~$L \to L$ bereits ein~\Automorphismus{$K$}.
\end{proposition}

\begin{remark}
  \Cref{homos are autos for finite extension} gilt auch dann noch, wenn~$L/K$ nur algebraisch ist.
\end{remark}

\begin{corollary}
  \label{galois consists of all homomorphisms}
  Ist die Erweiterung~$L/K$ endlich (oder allgemeiner algebraisch), so gilt
  \[
    \Gal(L/K)
    =
    \{
      \text{\Homomorphismen{$K$}~$L \to L$}
    \} \,.
  \]
\end{corollary}

Dank \cref{galois consists of all homomorphisms} lassen sich Galois-Gruppen in der Praxis durch Kombination von \cref{galois uniqueness} und wiederholten Anwenden von \cref{galois existence} bestimmen.

\begin{example}
  \leavevmode
  \begin{enumerate}
    \item
      Für die Körpererweiterung~$\Rational(i) / \Rational$ ist~$X^2 + 1 \in \Rational[X]$ das Minimalpolynom von~$i$.
      Die Nullstellen dieses Polynoms in~$\Rational(i)$ sind genau~$i$ und~$-i$.
      Wir erhalten somit aus \cref{galois existence}, dass die Galois-Gruppe~$\Gal(\Rational(i) / \Rational)$ aus genau zwei Elementen~$\varphi_1$,~$\varphi_2$ besteht, die durch~$\varphi_1(i) = i$ und~$\varphi_2(i) = -i$ gegeben sind.
      Es ist~$\varphi_1 = \id$ und~$\varphi_2$ die komplexe Konjugation.
      Es gilt insbesondere~$\Gal( \Rational(i) / \Rational ) \cong \Integer_2$.
    \item
      Für die Körperweiterung~$\Rational(\sqrt{2}, i)$ sind~$X^2 - 2, X^2 + 1 \in \Rational[X]$ die Minimalpolynome von~$\sqrt{2}$ und~$i$.
      Die Nullstellen dieser Polynome in~$\Rational(i, \sqrt{2})$ sind gegeben durch~$\sqrt{2}$ und~$\sqrt{-2}$, sowie~$i$ und~$-i$.
      Aus \cref{galois uniqueness} und~\cref{galois existence} erhalten wir, dass~$\Gal( \Rational(i, \sqrt{2}) / \Rational )$ aus höchstens vier Elementen bestehen, die gegeben sind durch
      \begin{alignat*}{2}
        \varphi_{00}
        &\colon
        \left\{
          \begin{array}{rcr}
            \sqrt{2}  &\mapsto& \sqrt{2} \,,
            \\
            i         &\mapsto& i \,,
          \end{array}
        \right.
        &
        \qquad
        \varphi_{10}
        &\colon
        \left\{
          \begin{array}{rcr}
            \sqrt{2}  &\mapsto& -\sqrt{2} \,,
            \\
            i         &\mapsto& i \,,
          \end{array}
        \right.
        \\[1em]
        \varphi_{01}
        &\colon
        \left\{
          \begin{array}{rcr}
            \sqrt{2}  &\mapsto& \sqrt{2} \,,
            \\
            i         &\mapsto& -i \,,
          \end{array}
        \right.
        &
        \qquad
        \varphi_{11}
        &\colon
        \left\{
          \begin{array}{rcr}
            \sqrt{2}  &\mapsto& -\sqrt{2} \,,
            \\
            i         &\mapsto& -i \,.
          \end{array}
        \right.
      \end{alignat*}
      Jeder dieser potenziellen~\Homomorphismen{$K$} lässt sich durch wiederholtes Anwenden von~\cref{galois existence} auch tatsächlich konstruieren.
      Es ergibt sich daraus, dass
      \[
        \Gal\bigl( \Rational(\sqrt{2}, i) / \Rational \bigr)
        \cong
        \Integer_2 \times \Integer_2 \,.
      \]
  \end{enumerate}
\end{example}

\begin{theorem}
  \label{galois inequality}
  Ist die Körperereiterung~$L/K$ endlich, so gilt~$\card{ \Gal(L/K) } \leq [L : K]$.
\end{theorem}

Um \cref{galois inequality} besser zu verstehen sei~$L = K(\alpha)$ eine einfache Körpererweiterung.
Es sei~$\mu_\alpha$ das Minimalpolynom von~$\alpha$ über~$K$.
Es sei~$n \defined \deg(\mu_\alpha)$ es seien~$\beta_1, \dotsc, \beta_m$ die paarweise verschiedenen Nullstellen von~$\mu_\alpha$ in~$L$.
Aus~\cref{galois existence} erhalten wir, dass
\[
  \card{ \Gal(L/K) }
  =
  m \,.
\]
Andererseits ist
\[
  [L : K] = [K(\alpha) : K] = \deg(\mu_\alpha) = n \,.
\]
Die Gleichheit~$\card{\Gal(L/K)} = [L : K]$ gilt also genau dann, wenn~$\mu_\alpha$ über~$L$ in Linearfaktoren zerfällt und alle Nullstellen paarweise verschieden sind, wenn also~$\mu_\alpha$ maximal viele Nullstellen in~$L$ hat.

Diese Bedingung führt zu den Konzepten von normalen und seperablen Körperweiterungen.




\section{Normale Körpererweiterungen}



\subsection{Zerfällungskörper}

\begin{definition}
  Es sei~$(f_i)_{i \in I}$ eine Familie von Polynomen~$f_i \in K[X]$.
  Der Erweiterungskörper~$L$ von~$K$ ist ein \defemph{Zerfällungskörper} der~$f_i$, falls jedes~$f_i$ über~$L$ in Linearfaktoren zerfällt, und die Erweiterung~$L/K$ von den Nullstellen der~$f_i$ erzeugt wird.
\end{definition}

\begin{proposition}
  Es sei~$(f_i)_{i \in I}$ eine Familie von Polynomen~$f_i \in K[X]$.
  \begin{enumerate}
    \item
      Die Familie~$(f_i)_{i \in I}$ besitzt einen Zerfällungskörper.
    \item
      Je zwei Zerfällungskörper der Familie~$(f_i)_{i \in I}$ sind~\isomorph{$K$}.
  \end{enumerate}
\end{proposition}



\subsection{Normale Erweiterungen}

\begin{proposition}
  \label{characterizations of normal field extensions}
  Ist die Körpererweiterung~$L/K$ algebraisch, so sind die folgenden Bedingungen äquivalent:
  \begin{equivlist}
    \item
      Es ist~$L$ der Zerfällungskörper ein Familie~$(f_i)_{i \in I}$ von Polynomen~$f_i \in K[X]$.
    \item
      Jedes irreduzible Polynom~$f \in K[X]$, das in~$L$ ein Nullstelle hat, zerfällt über~$L$ bereits in Linearfaktoren.
    \item
      Jeder~\Homomorphismus{$K$}~$L \to \closure{L}$ hat das gleiche Bild, nämlich~$L$.
  \end{equivlist}
\end{proposition}

\begin{definition}
  Eine algebraische Körpererweiterung~$L/K$, welche die äquivalenten Bedingungen aus \cref{characterizations of normal field extensions} erfüllt, ist \defemph{normal}.
\end{definition}

\begin{proposition}
  \label{restriction of normal}
  Ist die Körpererweiterung~$L/K$ normal und ist~$Z$ ein Zwischenkörper dieser Erweiterung, so ist auch die Erweiterung~$L/Z$ normal.
\end{proposition}

\begin{warning}
  \leavevmode
  \begin{enumerate}
    \item
      In der Situation von \cref{restriction of normal} ist die Erweiterung~$Z/K$ im Allgemeinen nicht normal.
    \item
      Sind~$M/L$ und~$L/K$ zwei normal Körpererweiterungen, so ist die Erweiterung~$M/K$ im Allgemeinen nicht normal.
  \end{enumerate}
\end{warning}



\subsection{Normale Hüllen}

Ist die Körpererweiterung~$L/K$ nicht normal, so liegt dies zwangsweise daran, dass gewisse Polynome aus~$K[X]$ nicht genügend Nullstellen in~$L$ haben.
Dies lässt sich durch Hinzuadjungieren dieser Nullstelen lösen.
Hierdurch lässt sich die Erweiterung~$L/K$ auf möglichst kleine Weise durch einer normalen Erweiterung ergänzen, der sogennanten normalen Hülle von~$L/K$.

\begin{definition}
  Die Körpererweiterung~$L/K$ sei algebraisch.
  Eine \defemph{normale Hülle} der Erweiterung~$L/K$ ist eine Körpererweiterung~$N/L$, so dass die folgenden Bedingungen gelten.
  \begin{itemize}
    \item
      Die Erweiterung~$N/K$ ist normal.
    \item
      Ist~$Z$ ein Zwischenkörper der Erweiterung~$N/L$, so dass~$Z/K$ normal ist, so gilt bereits~$Z = N$.
  \end{itemize}
\end{definition}

\begin{proposition}
  Die Erweiterung~$L/K$ sei algebraisch.
  \begin{enumerate}
    \item
      Es gibt eine normal Hülle für~$L/K$.
    \item
      Je zwei normale Hüllen der Erweiterung~$L/K$ sind~\isomorph{$L$}.
  \end{enumerate}
\end{proposition}





\section{Seperable Körpererweiterungen}



\subsection{Mehrfache Nullstellen}

\begin{definition}
  Die \defemph{(formale) Ableitung} eines Polynoms~$f \in K[X]$ mit~$f = \sum_{i=0}^n a_i X^i$ ist das Polynom~$f' \defined \sum_{i=1}^n i a_i X^{i-1}$.
\end{definition}

\begin{lemma}
  Für alle Polynome~$f, g \in K[X]$ und Skalare~$\lambda \in K$ gelten die Identitäten
  \[
    (f + g)' = f' + g' \,,
    \quad
    (\lambda f)' = \lambda f' \,,
    \quad
    (fg)' = f' g + f g' \,.
  \]
\end{lemma}

\begin{proposition}
  \label{characterizations of multiple roots}
  Für ein Element~$\alpha$ des Erweiterungskörpers~$L$ sind die folgenden Bedingungen äquivalent.
  \begin{equivlist}
    \item
      Das Element~$\alpha$ ist eine mehrfache Nullstelle von~$f$.
    \item
      Das Element~$\alpha$ ist eine gemeinsame Nullstelle von~$f$ und~$f'$.
    \item
      Das Element~$\alpha$ ist eine Nullstelle von~$\ggT(f,f')$.
  \end{equivlist}
\end{proposition}



\subsection{Seperable Polynome}

\begin{definition}
  \leavevmode
  \begin{enumerate}
    \item
      Ein irreduzibles Polynom~$f \in K[X]$ ist \defemph{seperabel} falls es keine mehrfache Nullstelle in~$\closure{K}$ hat.
    \item
      Ein Polynom~$f \in K[X]$ ist \defemph{seperabel} falls jeder irreduzible Faktor von~$f$ seperabel ist.
  \end{enumerate}
  Ein nicht-seperables Polynom ist \defemph{inseperabel}.
\end{definition}

\begin{lemma}
  Ein irreduzibles Polynom~$f \in K[X]$ ist genau dann inseperabel wenn~$f' = 0$ gilt.
\end{lemma}

\begin{proposition}
  \leavevmode
  \begin{enumerate}
    \item
      Gilt~$\ringchar(K) = 0$ so ist jedes Polynom~$f \in K[X]$ seperabel.
    \item
      Gilt~$\ringchar(K) = p > 0$ so ist ein irreduzibles Polynom~$f \in K[X]$ genau dann inseperabel, wenn es ein Polynom~$g \in K[X]$ mit~$f(X) = g(X^p)$ gibt, wenn also~$f$ nur Monome der Form~$X^{np}$ mit~$n \in \Natural$ enthält.
  \end{enumerate}
\end{proposition}

\begin{example}
  \leavevmode
  \begin{enumerate}
    \item
      Über~$\Rational$ ist jedes Polynom seperabel.
      In \enquote{klassischer Galois-Theorie} (also Galois-Theorie über~$\Rational$) ist Seperablität deshalb kein problematisches Konzept.
    \item
      Für den Körper~$K \defined \Finite_p(t)$ ist das Polynom~$X^p - t \in K[X]$ irreduzibel aber inseperabel.
  \end{enumerate}
\end{example}



\subsection{Seperable Elemente und Körpererweiterungen}

\begin{definition}
  Es sei~$\alpha \in L$ algebraisch über~$K$.
  \begin{enumerate}
    \item
      Das Element~$\alpha$ ist \defemph{seperabel} über~$K$, falls sein Minimalpolynom über~$K$ seperabel ist.
    \item
      Ein nicht-seperables Element ist \defemph{inseperabel}.
  \end{enumerate}
\end{definition}

\begin{definition}
  \leavevmode
  \begin{enumerate}
    \item
      Die Körpererweiterung~$L/K$ ist \defemph{seperabel} falls sie algebraisch ist und jedes Element von~$L$ seperabel über~$K$ ist.
    \item
      Eine nicht-seperable Körpererweiterung ist \defemph{inseperabel}.
  \end{enumerate}
\end{definition}

\begin{warning}
  Eine inseperable Körpererweiterung kann auch seperable Elemente enthalten.
\end{warning}

\begin{definition}
  Es ist~$L_{\sep} \defined \{ \alpha \in L \suchthat \text{$\alpha$ ist seperabel über~$K$} \}$ der \defemph{seperable Abschluss} von~$K$ in~$L$.
\end{definition}

\begin{proposition}
  Der seperable Abschluss~$L_{\sep}$ ist ein Zwischenkörper der Erweiterung~$L/K$.
\end{proposition}

\begin{proposition}
  Sind~$M/L/K$ Körpererweiterungen, so ist die Erweiterung~$M/K$ genau dann separabel, wenn die Erweiterungen~$M/L$ und~$L/K$ beide separabel sind.
\end{proposition}



\subsection{Perfekte Körper}

\begin{proposition}
  \label{characterizations of seperable field extensions}
  Für jeden Körper~$K$ sind die folgenden beiden Bedingungen äquivalent.
  \begin{equivlist}
    \item
      Jedes Polynom~$f \in K[X]$ ist seperabel.
    \item
      Jede algebraische Körpererweiterung~$L/K$ ist seperabel.
  \end{equivlist}
\end{proposition}

\begin{definition}
  Ein Körper~$K$ ist \defemph{perfekt} falls er die äquivalenten Bedingungen von \cref{characterizations of seperable field extensions} erfüllt. 
\end{definition}

\begin{example}
  \leavevmode
  \begin{enumerate}
    \item
      Jeder Körper von Charakteristik~$0$ ist perfekt.
    \item
      Jeder algebraisch abgeschlossene Körper ist perfekt.
    \item
      Der Körper~$\Finite_q(t)$ ist nicht perfekt.
  \end{enumerate}
\end{example}



\subsection{Der Frobenius-Homomorphismus}

\begin{definition}
  Es sei~$R$ ein Ring mit~$\ringchar(R) = p$ prim.
  Die Abbildung
  \[
    \Fr
    \colon
    R \to R
    \quad
    r \mapsto r^p
  \]
  ist der \defemph{Frobenius-Homomorphismus} von~$R$.
\end{definition}

\begin{lemma}
  Es sei~$R$ mit~$\ringchar(R) = p$ prim.
  Der Frobenius-\hspace{0pt}Homomorphismus~$\Fr \colon R \to R$ ist ein Ringhomomorphismus.
\end{lemma}

\begin{proposition}
  Es sei~$K$ ein Körper mit~$\ringchar(R) = p$ prim.
  Der Körper~$K$ ist genau dann perfekt, wenn der Frobenius-Homomorphismus~$\Fr \colon K \to K$ surjektiv (und somit bijektiv) ist. 
\end{proposition}

\begin{corollary}
  Jeder endliche Körper ist perfekt.
\end{corollary}



\subsection{Der Satz vom Primitiven Element}

\begin{definition}
  Ist~$L/K$ eine einfache Körpererweiterung mit~$L = K(\alpha)$ für~$\alpha \in L$, so ist~$\alpha$ ein \defemph{primitives Element} der Erweiterung~$L/K$.
\end{definition}

\begin{theorem}[Satz vom Primitiven Element]
  Ist die Körpererweiterung~$L/K$ endlich und seperabel, so ist die Erweiterung~$L/K$ einfach, d.\,h. die es gibt ein primitives Element für die Erweiterung~$L/K$.
\end{theorem}





\clearpage





\section{Primkörper}
\newday{5}

\begin{definition}
  Der \defemph{Primkörper} von~$K$ ist~$\Prim(K) \defined \bigcap \{ K' \suchthat \text{$K' \subseteq K$ ist ein Unterkörper} \}$.
\end{definition}

\begin{proposition}
  Der Primkörper~$\Prim(K)$ ist ein Unterkörper von~$K$, und zwar der kleinste solche Unterkörper.
\end{proposition}

Der eindeutige Ringhomomorphismus~$\Integer \to K$, gegeben durch
\[
  i
  \colon
  \Integer \to K \,,
  \quad
  k \mapsto k \cdot 1_K \,,
\]
erfüllt~$\ker(i) = \genideal{n}$ für~$n = \ringchar(K)$.
Ist~$\ringchar(K) = p$ prim, so folgt, dass~$i$ einen Körperhomomorphismus
\[
  \Finite_p \to K \,,
  \quad
  \class{k} \mapsto k \cdot 1_K
\]
induziert.
Gilt hingegen~$\ringchar(K) = 0$, so ist~$i$ injektiv, und induziert deshalb nach der universellen Eigenschaft der Lokalisierung einen Körperhomomorphismus
\[
  \Rational
  =
  \Quot( \Integer )
  \to
  K \,,
  \quad
  \frac{k}{l}
  \mapsto
  \frac{k \cdot 1_K}{l \cdot 1_K} \,.
\]
In beiden Fällen ist das Bild des induzierten Homomorphismus genau der Primkörper von~$K$.

Wir identifizieren im Folgenden den Primkörper von~$K$ stets mit~$\Finite_p$ oder~$\Rational$.

\begin{corollary}
  Es gilt
  \[
    \Prim(K)
    =
    \begin{cases*}
      \Rational
      &
      falls~$\ringchar(K) = 0$
      \\
      \Finite_p
      &
      falls~$\ringchar(K) = p$ prim ist.
    \end{cases*}
  \]
\end{corollary}





\section{Klassifikation Endlicher Körper}


Ist~$K$ ein endlicher Körper mit~$\ringchar(K) = p$, so ist~$K / \Finite_p$ eine endliche Körpererweiterung.

\begin{proposition}
  Ist~$K$ ein endlicher Körper mit~$\ringchar(K) = p$, so ist gilt~$\card{K} = p^n$ für einen passenden Exponenten~$n \in \Natural_1$.
\end{proposition}

\begin{lemma}
  \label{trick lemma for finite fields}
  Ist~$K$ ein endlicher Körper mit~$\card{K} = q$, so gilt~$x^q = x$ für alle~$x \in K$.
\end{lemma}

Wir zeigen im Folgenden die Umkehrung von \cref{trick lemma for finite fields}:
Für jede Potenz~$p^n$ gibt es einen Körper mit~$p^n$ Elementen, und dieser ist eindeutig bis auf Isomorphie.

\begin{proposition}
  Ist~$K$ ein endlicher Körper mit~$\ringchar(K) = p$ und~$\card{K} = q$, so ist~$K$ ein Zerfällungskörper des Polynoms~$X^q - X \in \Finite_p[X]$, und jedes Element von~$K$ ist eine Nullstelle dieses Polynoms.
\end{proposition}

Wir können diese Beobachtung auch umkehren.
Mithilfe des Frobenius-Homomorpismus ergibt sich das folgende Resultat.

\begin{lemma}
  Ist~$K$ ein Körper mit~$\ringchar(K) = p$ prim, und ist~$n \in \Natural_1$, so ist für~$q \defined p^n$ die Teilmenge
  \[
    \{
      x \in K
    \suchthat
      x^q = x
    \}
    =
    \{
      x \in K
    \suchthat
      \Fr^n(x) = x
    \}
  \]
  ein Unterkörper von~$K$.
\end{lemma}

Dabei erhalten wir aus \cref{characterizations of multiple roots} die folgende Aussage über die Anzahl der auftretenden Nullstellen.

\begin{lemma}
  Es sei~$K$ ein endlicher Körper mit~$\ringchar(K) = p$ und~$\card{K} = q$.
  Dann hat das Polynom~$X^q - X \in \Finite_p[X]$ keine mehrfache Nullstelle in~$K$.
\end{lemma}

\begin{proposition}
  Es sei~$p$ prim und~$q = p^n$ mit~$n \in \Natural_1$.
  Dann hat der Zerfällungskörper des Polynoms~$X^q - X \in \Finite_p[X]$ genau~$q$ Elemente und besteht genau aus den Nullstellen dieses Polynoms.
\end{proposition}

Wir könen nun beide Resultate kombinieren, und außerdem nutzen, dass Zerfällungskörper eindeutig bis auf Isomorphie sind.

\begin{theorem}[Klassifikation endlicher Körper]
  Es sei~$p$ eine Primzahl und~$n \in \Natural_1$.
  Es sei~$q \defined p^n$.
  \begin{enumerate}
    \item
      Es gibt bis auf Isomorphie genau einen Körper~$\Finite_q$ mit~$q$ Elementen.
    \item
      Es gilt~$\ringchar(\Finite_q) = p$.
    \item
      Der Körper~$\Finite_q$ ist der Zerfällungskörper des Polynoms~$X^q - X \in \Finite_q[X]$.
    \item
      Der Körper~$\Finite_q$ besteht bereits aus den Nullstellen des Polynoms~$X^q - X \in \Finite_q[X]$.
    \item
      Jeder endliche Körper~$K$ ist von dieser Form für~$p = \ringchar(K)$ und passendes~$n$.
  \end{enumerate}
\end{theorem}

\begin{proposition}
  Es ist~$\Finite_{p^n}$ genau dann ein Unterkörper von~$\Finite_{p^m}$, bzw. es gibt genau dann einen Körperhomomorphismus~$\Finite_{p^n} \to \Finite_{p^m}$, wenn~$n \divides m$ gilt.
\end{proposition}

\begin{proposition}
  Die Galois-Gruppe~$\Gal(\Finite_{p^n} / \Finite_p)$ ist zyklisch von Ordnung~$n$, und wird vom Frobenius-Homomorphismus~$\Fr \colon \Finite_{p^n} \to \Finite_{p^n}$ erzeugt.
\end{proposition}





