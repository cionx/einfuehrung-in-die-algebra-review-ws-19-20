\chapter{Ringtheorie}


\begin{convention}
  Im Folgenden ist~$R$, sofern nicht anders angegeben, ein Ring.
\end{convention}


\begin{definition}
  Es sei~$S$ ein weiterer Ring.
  Eine Abbilung~$\varphi \colon R \to S$ ist ein \defemph{Ringhomomorphismus} falls für alle Elemente~$r_1, r_2 \in R$ die folgenden Bedingungen gelten:
  \[
    \varphi(1) = 1 \,,
    \quad
    \varphi(r_1 + r_2) = \varphi(r_1) + \varphi(r_2) \,,
    \quad
    \varphi(r_1 r_2) = \varphi(r_1) \spacing \varphi(r_2) \,.
  \]
\end{definition}

\begin{definition}
  Eine Teilmenge~$S$ von~$R$ ist ein Unterring, falls sie die folgenden Bedingungen erfüllt.
  \begin{itemize}
    \item
      Es gilt~$0 \in S$.
    \item
      Für alle Elemente~$s_1, s_2 \in R$ gilt auch~$s_1 + s_2 \in S$.
    \item
      Für jedes Element~$s \in S$ gilt auch~$-s \in S$.
    \item
      Für alle Element~$s_1, s_2 \in S$ gilt auch~$s_1 s_2 \in S$.
  \end{itemize}
\end{definition}

\begin{example}
  Ist~$\varphi \colon R \to S$ ein Ringhomomorphismus, so ist~$\im(\varphi)$ ein Unterring von~$S$.
\end{example}





\section{Einheiten und Nullteiler}



\subsection{Einheiten}

\begin{definition}
  Ein Element~$r \in R$ ist eine \defemph{Einheit}, bzw. ist \defemph{invertierbar}, falls es ein Element~$r' \in R$ gibt, so dass~$r r' = 1$ und~$r' r = 1$ gilt.
  Die Menge der Einheiten in~$R$ wird mit~$R^\times$ notiert.
\end{definition}

\begin{proposition}
  Die Menge~$R^\times$ zusammen mit der Multiplikation von~$R$ ist eine Gruppe.
\end{proposition}

\begin{definition}
  Die Gruppe~$R^\times$ ist die \defemph{Einheitengruppe} von~$R$.
\end{definition}

\begin{example}
  \leavevmode
  \begin{enumerate}
    \item
      Ein kommutativer Ring~$R$ ist ein Körper, genau dann wenn~$R^\times = R \setminus \{ 0 \}$.
    \item
      Ist~$K$ ein Körper, so gilt~$\Mat(n, K)^\times = \GL(n,K)$.
    \item
      Es gilt~$\Integer^\times = \{1, -1\}$.
    \item
      Es gilt~$\Integer_n^\times = \{ \class{k} \in \Integer_n \suchthat \text{$k$ und~$n$ sind teierfremd} \}$.
  \end{enumerate}
\end{example}



\subsection{Nullteiler}

\begin{definition}
  Es sei~$r$ ein Element von~$R$.
  \begin{enumerate}
    \item
      Es ist~$r$ ist ein \defemph{Linksnullteiler} wenn es ein Element~$x \in R$ mit~$x \neq 0$ gibt, so dass~$rx = 0$.
    \item
      Es ist~$r$ ist ein \defemph{Rechtsnullteiler} wenn es ein Element~$x \in R$ mit~$x \neq 0$ gibt, so dass~$xr = 0$.
    \item
      Es ist~$r$ ist ein \defemph{(beidseitiger) Nullteiler} wenn~$r$ ein Links- und auch ein Rechtsnullteiler ist.
  \end{enumerate}
\end{definition}

\begin{example}
  Es sei~$n \in \Natural_1$.
  Ein Element~$\class{k} \in \Integer_n$ ist ein genau dann ein Nullteiler wenn~$k$ und~$n$ nicht teilerfremd sind.
\end{example}

\begin{proposition}
  Ist~$r$ eine Einheit in~$R$, so ist~$R$ weder ein Rechts- noch ein Linksnullteiler.
\end{proposition}

\begin{definition}
  Ein kommutativer Ring~$R$ ist ein \defemph{Integritätsbereich} falls~$R \neq 0$ gilt und~$0$ der einzige Nullteiler in~$R$ ist.
\end{definition}

\begin{example}
  \leavevmode
  \begin{enumerate}
    \item
      Jeder Körper ist ein Integritätsbereich.
    \item
      $\Integer$ ist ein Integritätsbereich.
    \item
      Ist~$R$ ein Integritätsbereich, so ist jeder Unterring von~$R$ ebenfalls ein Integritätsbereich.
  \end{enumerate}
\end{example}





\section{Polynomringe}

\begin{proposition}
  Ist~$R$ ein Integritätsberich, so für jede Indexmenge~$A$ auch der Polynomring~$R[X_\alpha \suchthat \alpha \in A]$ ein Integritätsbereich.
\end{proposition}

\begin{proposition}
  Ist~$R$ ein Integritätsbereich, so gilt~$R[X]^\times = R^\times$.%
  \footnote{
    Ist allgemeiner~$R$ ein beliebiger kommutativer Ring, so ist ein Polynom~$f \in R[X]$ mit~$f = a_n X^n + \dotsb + a_1 X + a_0$ genau dann eine Einheit in~$R[X]$, wenn~$a_0$ eine Einheit in~$R$ ist und die restlichen Koeffizienten~$a_1, \dotsc, a_n$ nilpotent in~$R$ sind.
    Für einen kommutativen Ring~$R$ gilt somit genau dann~$R[X]^\times = R^\times$ wenn~$R$ außer~$0$ keine nilpotenen Elemente besitzt.
  }
  
\end{proposition}





\section{Ideale und Faktorringe}



\subsection{Ideale}

\begin{definition}
  \leavevmode
  \begin{enumerate}
    \item
      Eine Teilmenge~$I$ von~$R$ ist ein \defemph{Ideal} falls~$I$ eine additive Untergruppe von~$R$ ist, so dass~$RI \subseteq I$ und~$IR \subseteq I$ gelten.
    \item
      Gilt zusätzlich~$I \neq R$, so ist~$I$ ein \defemph{echtes Ideal}.
  \end{enumerate}
\end{definition}

\begin{example}
  \leavevmode
  \begin{enumerate}
    \item
      Die Ideale in~$\Integer$ sind genau~$n \Integer \defined \{ na \suchthat a \in \Integer \}$ für~$n \in \Integer$.
    \item
      Ist~$\varphi \colon R \to R'$ ein Ringhomomorphismus, so ist~$\ker(\varphi) = \{ r \in R \suchthat \varphi(r) = 0 \}$ ein Ideal in~$R$.
    \item
      Ein kommutative Ring~$R$ ist genau dann ein Körper, wenn~$R \neq 0$ gilt und~$R$,~$0$ die einzigen beiden Ideale in~$R$ sind.
  \end{enumerate}
\end{example}



\subsection{Erzeugte Ideale}

\begin{definition}
  Für eine Teilmenge~$X$ von~$R$ ist das von~$X$ in~$R$ \defemph{erzeugte Ideal} die Menge
  \[
    \genideal{X}
    \defined
    \{
      r_1 x_1 r'_1 + \dotsb + r_n x_n r'_n
    \suchthat
      n \in \Natural,
      r_i, r'_i \in R,
      x_i \in X
    \} \,.
  \]
\end{definition}

\begin{proposition}
  Es sei~$X$ eine Teilmenge von~$R$.
  \begin{enumerate}
    \item
      Es ist~$\genideal{X}$ ein Ideal in~$R$ mit~$X \subseteq \genideal{X}$.
    \item
      Das Ideal~$\genideal{X}$ ist eindeutig minimal unter allen Idealen von~$R$, welche die Menge~$X$ enthalten:
      Ist~$I$ ein Ideal in~$R$ dass~$\genideal{X}$ enthält, so gilt~$\genideal{X} \subseteq I$.
  \end{enumerate}
\end{proposition}

\begin{proposition}
  Ein Element~$x$ von~$R$ ist genau dann eine Einheit, wenn~$\genideal{x} = R$ gilt.
\end{proposition}



\subsection{Faktorringe}

\begin{lemma}
  \label{construction of quotient rings}
  Es sei~$I$ ein Ideal in~$R$.
  \begin{enumerate}
    \item
      Dann wird auf der Faktorgruppe~$R/I$ durch
      \[
        \class{r_1} \cdot \class{r_2}
        \defined
        \class{r_1 r_2}
      \]
      eine Ringstruktur definiert.
    \item
      Dies ist die eindeutige Ringstruktur auf~$R/I$, so dass die Abbildung~$\pi \colon R \to R/I$ gegeben durch~$\pi(r) = \class{r}$ ein Ringhomomorphismus ist.
  \end{enumerate}
\end{lemma}

\begin{definition}
  Es sei~$I$ ein Ideal in~$R$.
  Der Ring~$R/I$ aus \cref{construction of quotient rings} ist der \defemph{Faktorring} von~$R$ nach~$I$.
  Die Abbildung~$\pi \colon R \to R/I$ ist die \defemph{kanonische Projektion}.
\end{definition}



\subsection{Homomorphiesatz und Isomorphiesatz}

\begin{convention}
  Im Folgenden sei~$I$, sofern nicht anders angegeben, ein Ideal in~$R$.
\end{convention}

\begin{theorem}[Homomorphiesatz]
  Es sei~$\pi \colon R \to R/I$ die kanonische Projektion, und es sei~$S$ ein weiterer Ring.
  \begin{enumerate}
    \item
      Es gilt~$\ker(\pi) = I$.
    \item
      Ist~$\psi \colon R/I \to S$ ein Ringhomomorphismus, so gilt für den entstehenden Ringhomomorphismus~$\psi \circ \pi \colon R \to S$, dass~$I \subseteq \ker(\psi \circ \pi)$.
    \item
      Ist andererseits~$\varphi \colon R \to S$ ein Gruppenhomomorphismus mit~$I \subseteq \ker(\varphi)$, so ist
      \[
        \induced{\varphi}
        \colon
        R/I \to S \,,
        \quad
        \class{r} \mapsto \varphi(r)
      \]
      ein wohldefinierter Gruppenhomorphisms.
      Dies ist der eindeutige Gruppenhomomorphismus, der das folgende Diagramm zum kommutieren bringt:
      \[
        \begin{tikzcd}
          R
          \arrow{r}[above]{\varphi}
          \arrow{d}[left]{\pi}
          &
          S
          \\
          R/I
          \arrow[dashed]{ur}[below right]{\induced{\varphi}}
          &
          {}
        \end{tikzcd}
      \]
      Dabei gilt~$\im(\induced{\varphi}) = \im(\varphi)$ sowie~$\ker(\induced{\varphi}) = \ker(\varphi)/N$.
    \item
      Die beiden obigen Konstruktionen sind invers zueinander, und liefern somit eine Eins-zu-eins-Korrespondenz gegeben durch
      \begin{align*}
        \left\{
          \begin{tabular}{@{}c@{}}
            Ringhomomorphismen \\
            $\varphi \colon R \to S$ mit~$I \subseteq \ker(\varphi)$
          \end{tabular}
        \right\}
        &\onetoone
        \left\{
          \begin{tabular}{@{}c@{}}
            Ringhomomorphismen \\
            $\psi \colon R/I \to S$
          \end{tabular}
        \right\}
        \\
        \varphi
        &\mapsto
        \induced{\varphi} \,,
        \\
        \psi \circ \pi
        &\mapsfrom
        \psi \,.
      \end{align*}
  \end{enumerate}
\end{theorem}

\begin{corollary}
  Es sei~$S$ eine weiterer Ring und~$\varphi \colon R \to S$ ein Ringhomomorphismus.
  Dann induziert~$\varphi$ einen Ringisomorphismus
  \[
    \induced{\varphi}
    \colon
    R/{\ker(\varphi)} \to \im(\varphi) \,,
    \quad
    \class{r}
    \mapsto
    \varphi(r) \,.
  \]
\end{corollary}


\begin{theorem}[Noetherscher Isomorphiesatz]
  Es sei~$J$ ein Ideal in~$R$, das~$I$ enthält.
  Dann ist~$J/I$ ein Ideal in~$R$, und die kanonische Projektion~$\pi \colon R \to R/I$ induziert einen Ringisomorphismus
  \[
    (R/I) / (J/I)
    \to
    R/J \,,
    \quad
    \class{ \class{r} }
    \mapsto
    \class{r} \,.
  \]
\end{theorem}



\subsection{Primideale}

\begin{convention}
  Im Folgenden sei~$R$, sofern nicht anders angegeben, ein kommutativer Ring.
\end{convention}

\begin{definition}
  Ein Ideal~$\pideal$ in~$R$ ist ein \defemph{Primideal} falls~$\pideal$ ein echtes Ideal ist, und für je zwei Element~$r_1, r_2 \in R$ mit~$r_1 r_2 \in \pideal$ bereits~$r_1 \in \pideal$ oder~$r_2 \in \pideal$ gilt.
\end{definition}

\begin{proposition}
  Ein Ideal~$\pideal$ in~$R$ ist genau dann ein Primideal wenn der Faktorring~$R/\pideal$ ein Integritätsbereich ist.
\end{proposition}



\subsection{Maximale Ideale}

\begin{definition}
  Ein Ideal~$\mideal$ in~$R$ ist ein \defemph{maximales Ideal} fall~$\mideal$ ein echtes Ideal ist, und für jedes Ideal~$I$ in~$R$ mit~$\mideal \subseteq I \subseteq R$ bereits~$I = \mideal$ oder~$I = R$ gilt.
\end{definition}

\begin{proposition}
  Ein Ideal~$\mideal$ in~$R$ ist genau dann ein maximales Ideal wenn der Faktorring~$R/\mideal$ ein Körper ist.
\end{proposition}

\begin{corollary}
  Jedes maximale Ideal ist ein Primideal.
\end{corollary}

\begin{proposition}
  Jedes echte Ideal von~$R$ ist in einem maximalen Ideal enthalten.
\end{proposition}



\subsection{Korrespondez von Idealen}

\begin{theorem}
  Es sei~$\pi \colon R \to R/I$ die kanonische Projektion.
  \begin{enumerate}
    \item
      Es gibt eine Eins-zu-eins-\hspace{0pt}Korrespondenz von Idealen, gegeben durch
      \begin{align*}
        \{
          \text{Ideale in~$R$, die~$I$ enthalten}
        \}
        &\onetoone
        \{ \text{Ideale in~$R/I$} \}
        \\
        J
        &\mapsto
        J/I \,,
        \\
        \pi^{-1}(K)
        &\mapsfrom
        K \,.
      \end{align*}
    \item
      In der obigen Korrespondenz ist das Ideal~$J$ genau dann prim in~$R$, wenn das Ideal~$J/I$ prim in~$R/I$ ist.
      Durch Einschränkung der obigen Korrespondenz ergibt sich somit eine Eins-zu-eins-Korrespondenz, gegeben durch
      \begin{align*}
        \{
          \text{Primideale in~$R$, die~$I$ enthalten}
        \}
        &\onetoone
        \{ \text{Primideale in~$R/I$} \}
        \\
        \pideal
        &\mapsto
        \pideal/I \,,
        \\
        \pi^{-1}(\qideal)
        &\mapsfrom
        \qideal \,.
      \end{align*}
    \item
      In der obigen Korrespondenz ist das Ideal~$J$ genau dann maximal in~$R$, wenn das Ideal~$J/I$ maximal in~$R/I$ ist.
      Durch Einschränkung der obigen Korrespondenz ergibt sich somit eine Eins-zu-eins-Korrespondenz, gegeben durch
      \begin{align*}
        \{
          \text{maximale Ideale in~$R$, die~$I$ enthalten}
        \}
        &\onetoone
        \{ \text{maximale Ideale in~$R/I$} \}
        \\
        \mideal
        &\mapsto
        \mideal/I \,,
        \\
        \pi^{-1}(\mideal')
        &\mapsfrom
        \mideal' \,.
      \end{align*}
  \end{enumerate}
\end{theorem}





\section{Hauptidealringe und Euklidische Ringe}

\begin{definition}
  Ein Ideal~$I$ in~$R$ ist ein \defemph{Hauptideal} falls~$I = \genideal{a}$ für ein Element~$a \in R$ gilt.
\end{definition}

\begin{definition}
  Ein kommutativer Ring~$R$ ist ein \defemph{Hauptidealring} falls~$R$ ein Integritätsbereich ist und jedes Ideal in~$R$ ein Hauptideal ist.
\end{definition}

\begin{definition}
  Ein Integritätsbereich~$R$ ist \defemph{eukldisch} falls es im folgenden Sinne ein \enquote{Teilen mit Rest} in~$R$ gibt:
  Es gibt eine Gradabbildung
  \[
    \deg
    \colon
    R \setminus \{ 0 \}
    \to
    \Natural \,,
  \]
  so dass es für alle element~$a, b \in R$ mit~$b \neq 0$ Elemente~$q, r \in R$ gibt, so dass~$a = qb + r$ mit~$r = 0$ oder~$\deg(r) < \deg(b)$.
\end{definition}

\begin{proposition}
  Jeder euklidische Ring ist ein Hauptidealring.
\end{proposition}

\begin{example}
  Der Ring~$\Integer$ ist euklidisch mit~$\deg(n) \defined \abs{n}$.
  Inbesondere ist~$\Integer$ ein Hauptidealring.
\end{example}

\begin{proposition}
  Für einen kommutativen Ring~$K$ sind die folgenden Bedingungen äquivalent.
  \begin{equivlist}
    \item
      Der Ring~$K$ ist ein Körper.
    \item
      Der Polynomring~$K[X]$ ist euklidisch.
    \item
      Der Polynomring~$K[X]$ ist ein Hauptidealring.
  \end{equivlist}
\end{proposition}

\begin{example}
  Ist~$K$ ein Körper, so ist der Polynomring~$K[X]$ ein Hauptidealring, der Polynomring~$K[X,Y]$ allerdings nicht.
\end{example}

\begin{remark}
  Ist~$K$ ein Körper, so lässt sich das \enquote{Teilen mit Rest} in~$K[X]$ durch Polynomdivision berechnen.
\end{remark}

\begin{example}[Die Gausschen Zahlen]
  Der Unterring~$\Integer[i] \defined \{a + bi \suchthat a, b \in \Integer\}$ des Körpers~$\Complex$ ist der \defemph{Ring der Gausschen Zahlen}.
  Die \defemph{Norm} eines Elements~$z \in \Integer$ gegeben durch~$z = a + bi$ mit~$a, b \in \Integer$ ist
  \[
    N(z)
    \defined
    \abs{z}^2
    =
    a^2 + b^2 \,.
  \]

  Es seien~$a, b \in \Integer[i]$ mit~$b \neq 0$.
  In~$\Complex$ können wir das Element~$a/b$ betrachten, und es gibt ein Element~$q \in \Integer[i]$ mit
  \[
    \abs*{ \frac{a}{b} - q }
    \leq
    \frac{\sqrt{2}}{2}
    =
    \frac{1}{\sqrt{2}} \,.
  \]
  Es sei~$r \defined a - bq$.
  Es gilt nun~$a = bq + r$ mit
  \[
    N(r)
    =
    N(a - bq)
    =
    \abs{a-bq}^2
    =
    \abs*{ \frac{a}{b} - q }^2 \abs{b}^2
    \leq
    \frac{1}{2} N(b) \,.
  \]
  Das zeigt, dass~$\Integer[i]$ zusammen mit~$N$ als Gradabbildung ein euklidischer Ring ist.
\end{example}





\section{Lokalisierung und Quotientenkörper}



\subsection{Lokalisierungen}

\begin{definition}
  Eine Teilmenge~$S$ von~$R$ ist \defemph{multiplikativ abgeschlossen} falls~$1 \in S$ gilt, aund für alle Element~$s_1, s_2 \in S$ auch~$s_1 s_2 \in S$ gilt.
\end{definition}

\begin{convention}
  Im Folgenden sei~$S$, sofern nicht anders angegeben, eine multiplikativ abgeschlossene Teilmenge von~$R$.
\end{convention}

\begin{proposition}
  \label{construction of localization}
  \leavevmode
  \begin{enumerate}
    \item
      Auf der Menge~$R \times S$ wird durch
      \[
        (r_1, s_1) \sim (r_2, s_2)
        \iff
        \text{es gibt~$t \in S$ mit~$t(r_1 s_2 - r_2 s_1) = 0$}
      \]
      eine Äquivalenzrelation definiert.
  \end{enumerate}
  Es sei im Folgenden~$S^{-1} R \defined (R \times S) / {\sim}$, und für jedes Paar~$(r,s) \in R \times S$ sei~$r/s \defined \class{(r,s)}$ die Äquivalenzklasse von~$(r,s)$ in~$S^{-1} R$.
  \begin{enumerate}[resume*]
    \item
      Auf~$R^{-1} S$ wird durch
      \[
        \frac{r_1}{s_1} + \frac{r_2}{s_2}
        \defined
        \frac{r_1 s_2 + r_2 s_1}{s_1 s_2} \,,
        \quad
        \frac{r_1}{s_1} \cdot \frac{r_1}{s_1}
        \defined
        \frac{r_1 r_2}{s_1 s_2}
      \]
      eine Ringstruktur definiert.
      Dabei gelten~$0_{S^{-1} R} = 0/1$ und~$1_{S^{-1} R} = 1/1$.
    \item
      Für jedes Element~$s \in S$ ist~$s/1$ eine Einheit in~$S^{-1} R$ mit~$(s/1)^{-1} = 1/s$.
    \item
      Die Abbildung~$i \colon R \to S^{-1} R$ gegeben durch~$i(r) = r/1$ ist ein Ringhomomorphismus.
  \end{enumerate}
\end{proposition}

\begin{definition}
  In der Situation von \cref{construction of localization} ist~$S^{-1} R$ die \defemph{Lokalisierung} von~$R$ an~$S$.
  Die Abbildung~$i$ ist der \defemph{kanonische Homomorphismus}.
\end{definition}

\begin{proposition}
  Es sei~$i \colon R \to S^{-1} R$ der kanonische Homomorphismus.
  \begin{enumerate}
    \item
      Es gilt~$\ker(i) = \{ r \in R \suchthat \text{es gibt~$s \in S$ mit~$sr = 0$} \}$.
    \item
      Es gilt genau dann~$S^{-1} R = 0$ wenn~$0 \in S$.
  \end{enumerate}
\end{proposition}



\subsection{Quotientenkörper}

\begin{example}
  \label{construction of field of fractions}
  Ist~$R$ ein Integritätsbereich, so ist~$S \defined R \setminus \{ 0 \}$ eine multiplikativ abgeschlossene Teilmenge von~$R$ und die Lokalisierung~$S^{-1} R$ ist ein Körper.
  Der kanonische Homomorphismus~$i \colon R \to \Quot(R)$ ist injektiv, weshalb sich~$R$ als ein Unterring von~$\Quot(R)$ auffassen lässt.
\end{example}

\begin{definition}
  In der Situation von \cref{construction of field of fractions} ist~$S^{-1} R$ der \defemph{Quotientenkörper} von~$R$.
  Er wird mit~$\Quot(R)$ notiert.
\end{definition}

\begin{corollary}
  Ein Ring~$R$ ist genau dann ein Integritätsbereich, wenn er Unterring eines Körpers ist.
\end{corollary}

\begin{example}
  \leavevmode
  \begin{enumerate}
    \item
      Es gilt~$\Quot(\Integer) = \Rational$.
    \item
      Ist~$K$ ein Körper, so ist~$K(X) \defined \Quot(K[X])$ der \defemph{Körper der rationalen Funktionen} über~$K$.
  \end{enumerate}
\end{example}

\begin{example}
  Ist allgemeiner~$\pideal$ ein Primideal in~$R$, so ist das Komplement~$S \defined R \setminus \pideal$ eine multiplikativ abgeschlossene Teilmenge von~$R$.
  Die Lokalisierung~$S^{-1} R$ ist die \defemph{Lokalisierung von~$R$ an~$\pideal$} und wird mit~$R_{\pideal}$ notiert.
\end{example}

%\begin{example}
%  Ist~$I$ ein Ideal in~$R$, so ist~$1 + I$ eine multiplikativ abgeschlossene Teilmenge von~$R$.
%\end{example}



\subsection{Universelle Eigenschaft der Lokalisierung}

\begin{theorem}[Universelle Eigenschaft der Lokalisierung]
  Es sei~$S$ eine multiplikativ abgeschlossene Teilmenge von~$R$ und es sei~$i \colon R \to S^{-1} R$ der kanonische Homomorphismus.
  Es sei~$R'$ ein weiterer Ring.
  \begin{enumerate}
    \item
      Ist~$\psi \colon S^{-1} R \to R'$ ein Ringhomomorphismus, so bildet der resultierende Ringhomomorphismus~$\psi \circ i \colon R \to S^{-1} R$ die Elemente von~$S$ auf Einheiten in~$R'$ ab, d.h. es gilt~$(\psi \circ i)(S) \subseteq (R')^\times$.
    \item
      Ist andererseits~$\varphi \colon R \to R'$ ein Ringhomomorphismus mit~$\varphi(S) \subseteq (R')^\times$, so ist die Abbildung
      \[
        \induced{\varphi}
        \colon
        S^{-1} R
        \to
        R' \,,
        \quad
        \frac{r}{s}
        \mapsto
        \varphi(r) \spacing \varphi(s)^{-1}
      \]
      ein wohldefinierter Ringhomomorphismus.
      Dieser Ringhomomorphismus ist eindeutig dadurch bestimmt, dass er das folgende Diagramm zum kommutieren bringt.
      \[
        \begin{tikzcd}
          R
          \arrow{r}[above]{\varphi}
          \arrow{d}[left]{i}
          &
          R'
          \\
          S^{-1} R
          \arrow[dashed]{ur}[below right]{\induced{\varphi}}
          &
          {}
        \end{tikzcd}
      \]
    \item
      Die beiden obigen Konstruktionen sind invers zueinander, und ergeben die Eins-zu-eins-Korrespondenz
      \begin{align*}
        \left\{
          \begin{tabular}{@{}c@{}}
            Ringhomomorphismen\\
            $\varphi \colon R \to R'$ mit~$\varphi(S) \subseteq (R')^\times$
          \end{tabular}
        \right\}
        &\onetoone
        \left\{
          \begin{tabular}{@{}c@{}}
            Ringhomomorphismen\\
            $\psi \colon S^{-1}R \to R'$
          \end{tabular}
        \right\}
        \\
        \varphi
        &\mapsto
        \induced{\varphi} \,,
        \\
        \psi \circ i
        &\mapsfrom
        \psi \,.
      \end{align*}
  \end{enumerate}
\end{theorem}

% TODO: UP of field of fractions as a corollary

\begin{definition}
  Ein Ringhomomorphismus zwischen zwei Körpern ist ein \defemph{Körperhomomorphismus}.
\end{definition}

\begin{corollary}
  Es sei~$R$ ein Integritätsbereich.
  Es sei~$K$ ein Körper und~$\varphi \colon R \to K$ ein injektiver Ringhomomorphismus.
  Der Homomorphismus~$\varphi$ setzt sich eindeutig zu einem Körperhomomorphismus~$\Quot(R) \to K$ fort, d.h. es gibt einen eindeutigen Körperhomomorphismus~$\induced{\varphi} \colon \Quot(R) \to K$, der das folgende Diagramm zum kommutieren bringt:
  \[
    \begin{tikzcd}
      R
      \arrow{r}[above]{\varphi}
      \arrow{d}[left]{i}
      &
      K
      \\
      \Quot(R)
      \arrow[dashed]{ur}[below right]{\induced{\varphi}}
      &
      {}
    \end{tikzcd}
  \]
\end{corollary}





\section{Faktorielle Ringe}

\begin{convention}
  Im Folgenden sei~$R$, sofern nicht anders angegeben, ein Integritätsbereich.
\end{convention}

\begin{definition}
  Sind~$r_1$,~$r_2$ Elemente von~$R$, so ist~$r_1$ ein \defemph{Teiler} von~$r_2$, bzw.~$r_1$ \defemph{teilt}~$r_2$, wenn es ein Element~$a \in R$ mit~$r_2 = a r_1$ gibt.
  Dies wird dann mit~$r_1 \divides r_2$ notiert.
\end{definition}

\begin{proposition}
  Es seien~$r_1, r_2 \in R$.
  \begin{enumerate}
    \item
      Es gilt genau dann~$r_1 \divides r_2$ wenn~$\genideal{r_1} \subseteq \genideal{r_2}$.
    \item
      Es gilt genau dann~$\genideal{r_1} = \genideal{r_2}$, wenn es eine Einheit~$\varepsilon \in R^\times$ mit~$r_1 = \varepsilon r_2$ gibt.
  \end{enumerate}
\end{proposition}

\begin{example}
  Es sei~$K$ ein Körper.
  Dann ist der Polynomring~$K[X]$ ein Hauptidealring und es gilt~$K[X]^\times = K^\times = K \setminus \{0\}$.
  Jedes Ideal in~$K[X]$ ist deshalb von der Form~$\genideal{f}$ für~$f = 0$ oder ein eindeutiges normiertes Polynom~$f \in K[X]$.
\end{example}



\subsection{Primelemente und Irreduzible Elemente}

\begin{definition}
  Es sei~$p$ ein Element von~$R$.
  Es sei~$p \neq 0$ und~$p$ sei keine Einheit.
  \begin{enumerate}
    \item
      Das Element~$p$ ist \defemph{prim} falls es für alle Element~$r_1, r_1 \in R$ aus~$p \divides r_1 r_2$ folgt, dass bereits~$p \divides r_1$ oder~$p \divides r_2$ gilt.
    \item
      Das Element~$p$ ist \defemph{irreduzibel} falls für jede Zerlegung~$p = r_1 r_2$ mit~$r_1, r_2 \in R$ bereits~$r_1 \in R^\times$ oder~$r_2 \in R^\times$ gilt.
  \end{enumerate}
\end{definition}


\begin{proposition}
  Es sei~$p$ ein Element von~$R$.
  \begin{enumerate}
    \item
      Das Element~$p$ ist genau dann prim, wenn~$\genideal{p} \neq 0$ gilt und~$\genideal{p}$ ein Primideal ist.
    \item
      Ein Element~$p$ ist genau dann irreduzibel, wenn für alle~$\genideal{p} \neq 0, R$ gilt, und für jedes Hauptideal~$I$ in~$R$ mit~$\genideal{p} \subseteq I \subseteq R$ gilt bereits~$I = \genideal{p}$ oder~$I = R$.
  \end{enumerate}
\end{proposition}

\begin{proposition}
  Es sei~$p$ ein Elemetn von~$R$.
  \begin{enumerate}
    \item
      Ist~$p$ prim, so ist~$p$ auch irreduzibel.
    \item
      Ist~$R$ aun Hauptidealring, so sind die folgenden Bedingungen an~$p$ äquivalent.
      \begin{equivlist}
        \item
          Das Element~$p$ ist irreduzibel.
        \item
          Das Hauptideal~$\genideal{p}$ ist maximal.
        \item
          Das Element~$p$ ist prim.
      \end{equivlist}
  \end{enumerate}
\end{proposition}



\subsection{Faktorielle Ringe}

\begin{definition}
  Der Integritätsbereich~$R$ ist \defemph{faktoriell} falls die folgenden beiden Bedingungen gelten.
  \begin{itemize}
    \item
      Jedes Element~$r \in R$ mit~$r \neq 0$ besitzt eine Zerlegung
      \begin{equation}
        \label{prime decomposition}
        r = \varepsilon p_1 \dotsm p_n \,,
      \end{equation}
      wobei~$\varepsilon$ eine Einheit ist und~$p_1, \dotsc, p_n$ irreduzibel sind.
    \item
      Diese Zerlegung ist in dem folgenden Sinne \enquote{eindeutig bis auf Einheiten}:
      Es sei
      \[
        r = \varepsilon' p'_1 \dotsm p'_m
      \]
      eine weiter Zerlegung, wobei~$\varepsilon'$ eine Einheit ist und~$p'_1, \dotsc, p'_m$ irreduzibel sind.
      Dann gilt bereits~$n = m$, es gibt eine Permutation~$\sigma \in \Sym_n$ und Einheiten~$\delta_1, \dotsc, \delta_n$ mit
      \[
        p'_i = \delta_i p_{\sigma(i)} \,.
      \]
      (Es gilt dann notwendigerweise~$\epsilon = \delta_1 \dotsm \delta_n \varepsilon'$.)
  \end{itemize}
  Ist~$R$ faktoriell, so ist die Zerlegung~\eqref{prime decomposition} die \defemph{Primfaktorzerlegung} von~$r$.
\end{definition}

\begin{proposition}
  Es sei~$R$ ein faktorieller Ring.
  Ein Element von~$R$ ist genau dann prim, wenn es irreduzibel ist.
\end{proposition}

\begin{proposition}
  Ist~$R$ ein Hauptidealring, so ist~$R$ faktoriell.
\end{proposition}

\begin{corollary}
  Ist~$R$ ein euklidischer Ring, so ist~$R$ faktoriell.
\end{corollary}

\begin{example}
  \leavevmode
  \begin{enumerate}
    \item
      Der Ring~$\Integer$ ist faktoriell.
      Ein Element~$p \in \Integer$ ist genau dann prim wenn~$\abs{p}$ eine Primzahl ist.
    \item
      Der Ring der Gaußschen Zahlen,~$\Integer[i]$, ist faktoriell.
    \item
      Ist~$K$ ein Körper, so ist der Polynomring~$K[X]$ faktoriell.
  \end{enumerate}
\end{example}



\subsection{Der Satz von Gauß}

\begin{convention}
  Es sei im Folgenden~$R$ ein faktorieller Ring.
\end{convention}

\begin{definition}
  Es sei~$f \in R[X]$ ein Polynom gegeben durch~$f = a_n X^n + \dotsb + a_1 X + a_0$.
  Das Polynom~$f$ ist \defemph{primitiv} falls die Koeffizienten~$a_0, \dotsc, a_n$ teilerfremd sind, d.h.\ wenn es keine Nicht-Einheit~$r \in R$ gibt, so dass~$r \divides a_0, \dotsc, a_n$.
\end{definition}

\begin{proposition}[Lemma von Gauß]
  Sind~$f, g \in R[X]$ zwei primitive Polynome, so ist auch das Produkt~$fg$ primitiv.
\end{proposition}

\begin{theorem}[Satz von Gauß]
  Der Polynomring~$R[X]$ ist wieder faktoriell.
  Die irreduziblen Elemente von~$R[X]$ sind genau die folgenden Polynome.
  \begin{enumerate}
    \item
      Die konstanten Polynome, also Elemente von~$R$, welche irreduzibel in~$R$ sind.
    \item
      Die primitiven Polynome, welche irreduzibel in~$\Quot(R)[X]$ ist.
  \end{enumerate}
\end{theorem}

\begin{example}
  Ist~$K$ ein Körper, so ist für jede endliche Anzahl von Variablen~$n \in \Natural$ ist der Polynomring~$K[X_1, \dotsc, X_n]$ faktoriell.
\end{example}



\subsection{Irreduziblitätskriterien}

\begin{proposition}[Eisenstein-Kriterium]
  Es sei~$f \in R[X]$ ein Polynom vom Grad~$n \geq 1$ gegeben durch~$f = a_n X^n + \dotsb + a_1 X + a_0$.
  Es gebe ein Primelement~$p \in R$ mit
  \[
    p \ndivides a_n \,,
    \quad
    p \divides a_{n-1}, \dotsc, a_1, a_0 \,,
    \quad
    p^2 \ndivides a_0 \,.
  \]
  Dann ist das Polynom~$f$ irreduzibel in~$R[X]$ und in~$\Quot(R)[X]$.
\end{proposition}

\begin{proposition}
  Es sei~$f \in R[X]$ ein Polynom und es sei~$a \in R$.
  Das Polynom~$f$ ist genau dann irreduzibel wenn das verschobene Polynom~$f(X + a)$ irreduzibel ist.
\end{proposition}

\begin{proposition}[Reduktionskriterium]
  Es sei~$f \in R[X]$ ein Polynom vom Grad~$n \geq 1$ gegeben durch~$f = a_n X^n + \dotsb + a_1 X + a_0$.
  Es sei~$p \in R$ ein Primelement mit~$p \ndivides a_n$ und es sei~$\eta \colon R[X] \to (R/\genideal{p})[X]$ der durch die kanonische Projektion~$\pi \colon R \to R / \genideal{p}$ induzierte Ringhomomorphismus.
  \begin{enumerate}
    \item
      Ist das Polynom~$\eta(f)$ irreduzibel in~$(R/\genideal{p})[X]$, so ist~$f$ irreduzibel in~$\Quot(R)[X]$.
    \item
      Ist~$f$ zusätzlicherweise primitiv, so ist~$f$ bereits irreduzibel in~$R[X]$.
  \end{enumerate}
\end{proposition}
