\chapter{Ringtheorie}


\begin{convention}
  Im Folgenden ist~$R$ ein Ring, sofern nicht anders angegeben.
\end{convention}


\begin{definition}
  Es sei~$S$ ein weiterer Ring.
  Eine Abbildung~$\varphi \colon R \to S$ ist ein \defemph{Ringhomomorphismus} falls für alle Elemente~$r_1, r_2 \in R$ die folgenden Bedingungen gelten:
  \[
    \varphi(1) = 1 \,,
    \quad
    \varphi(r_1 + r_2) = \varphi(r_1) + \varphi(r_2) \,,
    \quad
    \varphi(r_1 r_2) = \varphi(r_1) \spacing \varphi(r_2) \,.
  \]
\end{definition}

\begin{definition}
  Eine Teilmenge~$S$ von~$R$ ist ein \defemph{Unterring}, falls sie die folgenden Bedingungen erfüllt:
  \begin{itemize}
    \item
      Die Menge~$S$ eine additive Untergruppe von~$R$.
    \item
      Es gilt~$1 \in S$.
    \item
      Für alle Element~$s_1, s_2 \in S$ gilt auch~$s_1 s_2 \in S$.
  \end{itemize}
\end{definition}

\begin{example}
  \leavevmode
  \begin{enumerate}
    \item
      Es sind~$\Integer \subseteq \Rational \subseteq \Real \subseteq \Complex$ jeweils Unterringe.
    \item
      Es ist~$\Integer[i] \defined \{ a + ib \suchthat a, b \in \Integer \}$ ein Unterring von~$\Complex$, der Ring der \defemph{Gaußschen Zahlen}.
    \item
      Ist~$\varphi \colon R \to S$ ein Ringhomomorphismus, so ist~$\im(\varphi)$ ein Unterring von~$S$.
  \end{enumerate}
\end{example}





\section{Ideale und Faktorringe}



\subsection{Ideale}

\begin{definition}
  Eine Teilmenge~$I$ von~$R$ ist ein \defemph{Ideal}, falls die folgenden Bedingungen gelten:
  \begin{itemize}
    \item
      Die Menge $I$ ist eine eine additive Untergruppe von~$R$.
    \item
      Es gilt~$RI \subseteq I$ und~$IR \subseteq I$.
  \end{itemize}
\end{definition}

\begin{example}
  \leavevmode
  \begin{enumerate}
    \item
      Die Ideale in~$\Integer$ sind genau~$n \Integer \defined \{ na \suchthat a \in \Integer \}$ für~$n \in \Integer$.
    \item
      Ist~$\varphi \colon R \to R'$ ein Ringhomomorphismus, so ist~$\ker(\varphi) = \{ r \in R \suchthat \varphi(r) = 0 \}$ ein Ideal in~$R$.
    \item
      Ein kommutativer Ring~$R$ ist genau dann ein Körper, wenn~$R \neq 0$ gilt, und~$R$,~$0$ die einzigen beiden Ideale in~$R$ sind.
  \end{enumerate}
\end{example}



\subsection{Erzeugte Ideale}

\begin{definition}
  Für eine Teilmenge~$X$ von~$R$ ist das von~$X$ in~$R$ \defemph{erzeugte Ideal} die Menge
  \[
    \genideal{X}
    \defined
    \{
      r_1 x_1 r'_1 + \dotsb + r_n x_n r'_n
    \suchthat
      n \in \Natural,
      x_i \in X
      r_i, r'_i \in R,
    \} \,.
  \]
\end{definition}

\begin{proposition}
  Es sei~$X$ eine Teilmenge von~$R$.
  \begin{enumerate}
    \item
      Es ist~$\genideal{X}$ ein Ideal in~$R$ mit~$X \subseteq \genideal{X}$.
    \item
      Das Ideal~$\genideal{X}$ ist eindeutig minimal unter allen Idealen von~$R$, welche die Menge~$X$ enthalten:
      Ist~$I$ ein Ideal in~$R$, das~$\genideal{X}$ enthält, so gilt~$\genideal{X} \subseteq I$.
    \item
      Es gilt~$\genideal{X} = \bigcap \{ I \subseteq R \suchthat \text{$I$ ist ein Ideal, das~$X$ enthält} \}$.
  \end{enumerate}
\end{proposition}



\subsection{Faktorringe}

\begin{convention}
  Im Folgenden ist~$I$ ein Ideal in~$R$, sofern nicht anders angegeben.
\end{convention}

Auf der Faktorgruppe~$R/I$ wird durch
\[
  \class{r_1} \cdot \class{r_2}
  \defined
  \class{r_1 r_2}
\]
eine Ringstruktur definiert, und die kanonische Projektion~$R \to R/I$ wird hierdurch zu einem Ringhomomorphismus.

\begin{definition}
  Der obige Ring~$R/I$ ist der \defemph{Faktorring} von~$R$ nach~$I$.
\end{definition}

\begin{example}
  Für alle~$n \in \Natural_1$ gilt~$\Integer / n \Integer \cong \Integer_n$.
\end{example}



\subsection{Homomorphiesatz und Isomorphiesatz}


\begin{theorem}[Homomorphiesatz]
  Es sei~$\pi \colon R \to R/I$ die kanonische Projektion, und es sei~$S$ ein weiterer Ring.
  Ein Ringhomomorphismus~$\varphi \colon R \to R/I$ induziert genau dann einen Ringhomomorphismus~$\induced{\varphi} \colon R/I \to R$, der das Diagram
  \[
    \begin{tikzcd}
      R
      \arrow{r}[above]{\varphi}
      \arrow{d}[left]{\pi}
      &
      S
      \\
      R/I
      \arrow[dashed]{ur}[below right]{\induced{\varphi}}
      &
      {}
    \end{tikzcd}
  \]
  zum kommutieren bringt, wenn~$I \subseteq \ker(\varphi)$ gilt.
  Es gilt dann
  \[
    \induced{\varphi}( \class{r} )
    \defined
    \varphi(r)
    \qquad
    \text{für alle~$r \in R$,}
  \]
  sowie~$\im(\induced{\varphi}) = \im(\varphi)$ und~$\ker(\induced{\varphi}) = \ker(\varphi)/I$.
  Inbesondere ist der Ringhomomorphismus~$\induced{\varphi}$ eindeutig.
\end{theorem}

\begin{corollary}
  Es sei~$S$ eine weiterer Ring und~$\varphi \colon R \to S$ ein Ringhomomorphismus.
  Dann induziert~$\varphi$ einen Ringisomorphismus
  \[
    \induced{\varphi}
    \colon
    R/{\ker(\varphi)} \to \im(\varphi) \,,
    \quad
    \class{r}
    \mapsto
    \varphi(r) \,.
  \]
\end{corollary}


\begin{theorem}[Noetherscher Isomorphiesatz]
  Es sei~$J$ ein Ideal in~$R$, das~$I$ enthält.
  Dann ist~$J/I$ ein Ideal in~$R/I$, und die kanonische Projektion~$\pi \colon R \to R/I$ induziert einen Ringisomorphismus
  \[
    (R/I) / (J/I)
    \to
    R/J \,,
    \quad
    \class{ \class{r} }
    \mapsto
    \class{r} \,.
  \]
\end{theorem}



\subsection{Korrespondenz von Idealen}

\begin{theorem}
  \label{ideal correspondence}
  Es sei~$\pi \colon R \to R/I$ die kanonische Projektion.
  Es gibt eine Eins-zu-eins-\hspace{0pt}Korrespondenz von Idealen, gegeben durch
  \begin{align*}
    \{
      \text{Ideale in~$R$, die~$I$ enthalten}
    \}
    &\onetoone
    \{ \text{Ideale in~$R/I$} \}
    \\
    J
    &\mapsto
    J/I \,,
    \\
    \pi^{-1}(J')
    &\mapsfrom
    J' \,.
  \end{align*}
\end{theorem}





\section{Einheiten und Nullteiler}



\subsection{Einheiten}

\begin{definition}
  \leavevmode
  \begin{enumerate}
    \item
      Ein Element~$r \in R$ ist eine \defemph{Einheit}, bzw. ist \defemph{invertierbar}, falls es ein Element~$r' \in R$ gibt, so dass~$r r' = 1$ und~$r' r = 1$ gilt.
    \item
      Die Menge der Einheiten in~$R$ ist die \defemph{Einheitengruppe} von~$R$ und wird mit~$R^\times$ notiert.
  \end{enumerate}
\end{definition}

\begin{proposition}
  Zusammen mit der Multiplikation von~$R$ ist~$R^\times$ eine Gruppe.
\end{proposition}

\begin{example}
  \leavevmode
  \begin{enumerate}
    \item
      Ein kommutativer Ring~$R$ ist genau dann ein Körper, wenn~$R^\times = R \setminus \{ 0 \}$ gilt.
    \item
      Ist~$K$ ein Körper, so gilt~$\Mat(n, K)^\times = \GL(n,K)$.
    \item
      Es gilt~$\Integer^\times = \{1, -1\}$.
    \item
      Es gilt~$\Integer_n^\times = \{ \class{k} \in \Integer_n \suchthat \text{$k$ und~$n$ sind teilerfremd} \}$.
  \end{enumerate}
\end{example}

\begin{example}
  \label{units in gaussian integers}
  Für jedes Element~$z \in \Integer[i]$ mit~$z = a + ib$ für~$a, b \in \Integer$ sei
  \[
    N(z)
    \defined
    \abs{z}^2
    =
    a^2 + b^2
    \in \Natural
  \]
  die \defemph{Norm} von~$z$.
  Für alle~$z_1, z_2 \in \Integer[i]$ gilt
  \[
    N(z_1 z_2)
    =
    N(z_1) N(z_2) \,,
  \]
  und es gilt~$N(1) = 1$.
  Für~$z \in \Integer[i]^\times$ gilt somit
  \[
    N(z) N(z^{-1})
    =
    N(z z^{-1})
    =
    N(1)
    =
    1 \,.
  \]
  Dabei gilt~$N(z), N(z^{-1}) \in \Natural$, weshalb bereits~$N(z) = 1$ gelten muss.
  Wir erhalten somit, dass~$z \in \{1, -1, i, -i\}$.
  Andererseits sind~$1$,~$-1$,~$i$,~$-i$ Einheiten in~$\Integer[i]^\times$ (denn~$1/i = -i$).
  Ingesamt gilt somit
  \[
    \Integer[i]^\times
    =
    \{ 1, -1, i, -i \} \,.
  \]
\end{example}

\begin{convention}
  Im Folgenden ist~$R$ ein kommutativer Ring, und~$K$ ein Körper, sofern nicht anders angegeben. 
\end{convention}

\begin{proposition}
  \label{unit via generated ideal}
  Ein Element~$x$ von~$R$ ist genau dann eine Einheit, wenn~$\genideal{x} = R$ gilt.
\end{proposition}

\begin{warning}
  Für nicht-kommutative Ringe gilt \cref{unit via generated ideal} im Allgemeinen nicht.
  So gilt etwa für jede Matrix~$A \in \Mat(n,K)$ mit~$A \neq 0$, dass~$\genideal{A} = \Mat(n,K)$.%
  \footnote{
    Ist allgemeiner~$R$ ein beliebiger Grundring, so ist die Abbildung
    \[
      \{ \text{Ideale in~$R$} \}
      \to
      \{ \text{Ideale in~$\Mat(n,R)$} \} \,,
      \quad
      I
      \mapsto
      \Mat(n,I)
    \]
    eine wohldefinierte Bijektion.
    Ein Körper~$K$ besitzt nur die beiden Ideal~$0$ und~$K$, also besitzt der Matrizenring~$\Mat(n,K)$ nur die beiden Ideale~$\Mat(n,0) = 0$ und~$\Mat(n,K)$.
  }
\end{warning}


\subsection{Nullteiler}

\begin{definition}
  Ein Element~$r$ von~$R$ ist ein \defemph{Nullteiler}, falls es ein Element~$x \in R$ mit~$x \neq 0$ aber~$rx = 0$ gibt.
\end{definition}

\begin{example}
  Es sei~$n \in \Natural_1$.
  Ein Element~$\class{k} \in \Integer_n$ ist ein genau dann ein Nullteiler wenn~$k$ und~$n$ nicht teilerfremd sind.
\end{example}

\begin{proposition}
  Ist~$r$ eine Einheit in~$R$, so ist~$R$ kein Nullteiler.
\end{proposition}

\begin{definition}
  Der kommutative Ring~$R$ ist ein \defemph{Integritätsbereich} falls~$R \neq 0$ gilt und~$0$ der einzige Nullteiler in~$R$ ist.
\end{definition}

\begin{example}
  \leavevmode
  \begin{enumerate}
    \item
      Jeder Körper ist ein Integritätsbereich.
    \item
      $\Integer$ ist ein Integritätsbereich.
    \item
      Für~$n \in \Natural_1$ ist~$\Integer_n$ genau dann ein Integritätsbereich, wenn~$n$ prim ist.
      Dann ist~$\Integer_n$ bereits ein Körper.
    \item
      Allgemeiner ist jeder endliche Integritätsbereich bereits ein Körper.
    \item
      Ist~$R$ ein Integritätsbereich, so ist jeder Unterring von~$R$ ebenfalls ein Integritätsbereich.
  \end{enumerate}
\end{example}





\section{Polynomringe}

Es sei~$I$ eine Indexmenge.
Dann ist
\[
  \Natural^{(I)}
  \defined
  \{
    ( n_i )_{i \in I}
  \suchthat
    n_i \in \Natural \,,
    \text{$n_i = 0$ für alle bis auf endlich viele~$i \in I$}
  \}
\]
ein abelsches Monoid bezüglich der Addition
\[
  ( n_i )_{i \in I}
  +
  ( m_i )_{i \in I}
  \defined
  (n_i + m_i)_{i \in I} \,.
\]
Der Polynomring~$R[X_i \suchthat i \in I]$ ist die Menge der formalen Linearkombinationen
\[
  \sum_{\alpha \in \Natural^{(I)}} r_\alpha X^\alpha
\]
mit~$r_\alpha \in R$ für alle~$\alpha \in \Natural^{(I)}$ und~$r_\alpha = 0$ für alle bis auf endlich viele~$\alpha \in \Natural^{(I)}$.
Die Ringoperationen von~$R[X_i \suchthat i \in I]$ sind gegeben durch
\begin{align*}
  \sum_{\alpha \in \Natural^{(I)}} r_\alpha X^\alpha
  +
  \sum_{\alpha \in \Natural^{(I)}} s_\alpha X^\alpha
  &\defined
  \sum_{\alpha \in \Natural^{(I)}} (r_\alpha + s_\alpha) X^\alpha \,,
  \\
  \sum_{\alpha \in \Natural^{(I)}} r_\alpha X^\alpha
  \cdot
  \sum_{\beta \in \Natural^{(I)}} s_\beta X^\beta
  &\defined
  \sum_{\alpha, \beta \in \Natural^{(I)}} r_\alpha s_\beta X^{\alpha + \beta} \,.
\end{align*}
Wir können~$R$ als einen Unterring von~$R[X_i \suchthat i \in I]$ auffassen, indem wir jedes Ringelement~$r \in R$ mit dem \defemph{konstanten} Polynom~$r X^0$ identifizieren.
Für jeden Index~$i \in I$ setzen wir außerdem
\[
  X_i \defined X^{e_i}
\]
wobei~$e_i \in \Natural^{(I)}$ das Tupel mit Einträgen~$(e_i)_j = \delta_{ij}$ für~$j \in I$ ist.

Ist die Indexmenge~$I$ endlich mit~$I = \{ 1, \dotsc, r \}$, so schreiben wir
\[
  K[X_i \suchthat i \in \{1, \dotsc, n\}]
  \defines
  K[X_1, \dotsc, X_n] \,,
  \quad
  X^{(n_1, \dotsc, n_r)}
  \defines
  X_1^{n_1} \dotsm X_r^{n_r} \,.
\]

\begin{proposition}
  Ist~$R$ ein Integritätsberich, so ist auch~$R[X_i \suchthat i \in I]$ ein Integritätsbereich.
\end{proposition}

\begin{proposition}
  Ist~$R$ ein Integritätsbereich, so gilt~$R[X_i \suchthat i \in I]^\times = R^\times$.%
  \footnote{
    Ist allgemeiner~$R$ ein beliebiger kommutativer Ring, so ist ein Polynom~$f \in R[X_i \suchthat i \in I]$ genau dann eine Einheit in~$R[X_i \suchthat \in I]$, wenn der konstante Term von~$f$ ein Einheit in~$R$ ist, und alle anderen Koeffiezienten von~$f$ nilpotent in~$R$ sind.
  }
\end{proposition}

\begin{theorem}[Universelle Eigenschaft des Polynomrings]
  Es sei~$S$ ein weiterer kommutativer Ring.
  Dann gibt es für jede Familie~$(s_i)_{i \in I}$ von Elementen~$s_i \in S$ und jeden Ringhomomorphismus~$\varphi \colon R \to S$ einen eindeutigen Ringhomomorphismus~$\Phi \colon R[X_i \suchthat i \in I] \to S$ mit~$\Phi(X_i) = s_i$ für jeden  Index~$i \in I$, und~$\restrict{\Phi}{R} = \varphi$.
\end{theorem}

\begin{example}
  Es seien~$R$,~$S$ kommutative Ringe, und es sei~$R$ ein Unterring von~$S$.
  Dann ist die Inklusion~$R \to S$ ein Ringhomomorphismus, und für jede Familie~$(s_i)_{i \in I}$ von Elementen~$s_i \in S$ erhalten wir einen \defemph{Einsetzhomomorphismus}
  \[
    \ev_{(s_i)_{i \in I}}
    \colon
    R[X_i \suchthat i \in I]
    \to
    S
  \]
  der durch~$r \mapsto r$ und~$X_i \mapsto s_i$ gegeben ist.
  Wir setzen in den Polynomen aus~$R[X_i \suchthat i \in I]$ also die Elemente~$s_i$ für die Variablen~$X_i$ ein.
\end{example}

\begin{proposition}
  \leavevmode
  \begin{enumerate}
    \item
      Es gilt~$R[X_1, \dotsc, X_n][X_{n+1}] \cong R[X_1, \dotsc, X_n, X_{n+1}]$
    \item
      Es gilt~$R[X_1, \dotsc, X_n] / \genideal{X_n} \cong R[X_1, \dotsc, X_{n-1}]$.
  \end{enumerate}
\end{proposition}





\section{Primideale und Maximale Ideale}



\subsection{Primideale}

\begin{definition}
  Ein Ideal~$\pideal$ in~$R$ ist \defemph{prim}, bzw. ein \defemph{Primideal}, falls die folgenden Bedingungen gelten:
  \begin{itemize}
    \item
      Für alle Elemente~$r_1, r_2 \in R$ mit~$r_1 r_2 \in \pideal$ gilt bereits~$r_1 \in \pideal$ oder~$r_2 \in \pideal$.
    \item
      Das Ideal~$\pideal$ ist echt.
  \end{itemize}
\end{definition}

\begin{proposition}
  Ein Ideal~$\pideal$ in~$R$ ist genau dann ein Primideal, wenn der Faktorring~$R/\pideal$ ein Integritätsbereich ist.
\end{proposition}



\subsection{Maximale Ideale}

\begin{definition}
  Ein Ideal~$\mideal$ in~$R$ ist ein \defemph{maximal}, falls die folgenden Bedingungen gelten:
  \begin{itemize}
    \item
     Für jedes Ideal~$I$ in~$R$ mit~$\mideal \subseteq I \subseteq R$ gilt bereits~$I = \mideal$ oder~$I = R$.
    \item
      Das Ideal~$\mideal$ ist echt.
  \end{itemize}
\end{definition}

\begin{proposition}
  Ein Ideal~$\mideal$ in~$R$ ist genau dann maximal, wenn~$R/\mideal$ ein Körper ist.
\end{proposition}

\begin{corollary}
  Jedes maximale Ideal ist ein Primideal.
\end{corollary}

\begin{proposition}
  Jedes echte Ideal von~$R$ ist in einem maximalen Ideal enthalten.
\end{proposition}

\begin{example}
  \leavevmode
  \begin{enumerate}
    \item
      Für jede Primzahl~$p$ ist~$\genideal{p}$ ein maximales Ideal in~$\Integer$, da~$\Integer/\genideal{p} \cong \Integer_p$ ein Körper ist.
    \item
      Das Ideal~$\genideal{Y}$ in~$K[X,Y]$ ist prim aber nicht maximal, denn~$K[X,Y] / \genideal{Y} \cong K[X]$ ist ein zwar Integritätsbereich, aber kein Körper.
  \end{enumerate}
\end{example}


\subsection{Korrespondez von Idealen}

\begin{theorem}
  Es sei~$\pi \colon R \to R/I$ die kanonische Projektion.
  \begin{enumerate}
    \item
      In der Ideal-Korrespondenz aus \cref{ideal correspondence} ist das Ideal~$J$ genau dann prim in~$R$, wenn das Ideal~$J/I$ prim in~$R/I$ ist.
      Durch Einschränkung der obigen Korrespondenz ergibt sich somit eine Eins-zu-eins-Korrespondenz, gegeben durch
      \[
        \{
          \text{Primideale in~$R$, die~$I$ enthalten}
        \}
        \onetoone
        \{ \text{Primideale in~$R/I$} \}
      \]
    \item
      In der Ideal-Korrespondenz aus \cref{ideal correspondence} ist das Ideal~$J$ genau dann maximal in~$R$, wenn das Ideal~$J/I$ maximal in~$R/I$ ist.
      Durch Einschränkung der obigen Korrespondenz ergibt sich somit eine Eins-zu-eins-Korrespondenz, gegeben durch
      \[
        \{
          \text{maximale Ideale in~$R$, die~$I$ enthalten}
        \}
        \onetoone
        \{ \text{maximale Ideale in~$R/I$} \}
      \]
  \end{enumerate}
\end{theorem}





\section{Hauptidealringe und Euklidische Ringe}



\subsection{Hauptidealringe}

\begin{definition}
  Ein Ideal~$I$ in~$R$ ist ein \defemph{Hauptideal} falls~$I = \genideal{a}$ für ein Element~$a \in R$ gilt.
\end{definition}

\begin{definition}
  Ein kommutativer Ring~$R$ ist ein \defemph{Hauptidealring}, falls die folgenden Bedingungen gelten:
  \begin{itemize}
    \item
      Jedes Ideal in~$R$ ist ein Hauptideal.
    \item
      Der Ring~$R$ ist ein Integritätsbereich.
  \end{itemize}
\end{definition}

\begin{example}
  \leavevmode
  \begin{enumerate}
    \item
      Der Ring~$\Integer$ ist ein Hauptidealring.
    \item
      Jeder Körper ist ein Hauptidealring.
  \end{enumerate}
\end{example}



\subsection{Euklidische Ringe}

\begin{definition}
  Ein Integritätsbereich~$R$ ist \defemph{euklidisch} falls es im folgenden Sinne ein \enquote{Teilen mit Rest} in~$R$ gibt:
  Es gibt eine Gradabbildung
  \[
    \deg
    \colon
    R \setminus \{ 0 \}
    \to
    \Natural \,,
  \]
  so dass es für alle Element~$a, b \in R$ mit~$b \neq 0$ Elemente~$q, r \in R$ gibt, so dass~
  \[
    a = qb + r
    \quad\text{mit}\quad
    \text{$r = 0$ oder~$\deg(r) < \deg(b)$.}
  \]
\end{definition}

\begin{proposition}
  Jeder euklidische Ring ist ein Hauptidealring.
\end{proposition}

\begin{example}
  \leavevmode
  \begin{enumerate}
    \item
      Der Ring~$\Integer$ ist euklidisch mit Gradabbildung~$\deg(n) \defined \abs{n}$.
    \item
      Der Polynomring~$K[X]$ euklidisch, durch die übliche Gradabbildung.
      Das \enquote{Teilen mit Rest} in~$K[X]$ lässt sich durch Polynomdivision berechnen.
    \item
      Der Körper~$K$ ist euklidisch mit~$\deg(x) = 0$ für alle~$x \in K$ (denn in~$K$ kann man auch schon \enquote{ohne Rest} teilen).
  \end{enumerate}
  Diese Ringe sind also inbesondere Hauptidealringe.
\end{example}

\begin{example}
  Der Ring der Gaußschen Zahlen~$\Integer[i]$ ist euklidisch:

  Wie schon in \cref{units in gaussian integers} ist die Norm eines Elements~$z \in \Integer$, das durch~$z = a + bi$ mit~$a, b \in \Integer$ gegeben ist, definiert als
  \[
    N(z)
    \defined
    \abs{z}^2
    =
    a^2 + b^2 \,.
  \]
  Es seien~$a, b \in \Integer[i]$ mit~$b \neq 0$.
  In~$\Complex$ können wir das Element~$a/b$ betrachten, und es gibt ein Element~$q \in \Integer[i]$ mit
  \[
    \abs*{ \frac{a}{b} - q }
    \leq
    \frac{\sqrt{2}}{2}
    =
    \frac{1}{\sqrt{2}} \,.
  \]
  Es sei~$r \defined a - bq$.
  Es gilt nun~$a = bq + r$ mit
  \[
    N(r)
    =
    N(a - bq)
    =
    \abs{a-bq}^2
    =
    \abs*{ \frac{a}{b} - q }^2 \abs{b}^2
    \leq
    \frac{1}{2} N(b) \,.
  \]
  Das zeigt, dass~$\Integer[i]$ zusammen mit~$N$ als Gradabbildung ein euklidischer Ring ist.
\end{example}

\begin{proposition}
  Für einen kommutativen Ring~$K$ sind die folgenden Bedingungen äquivalent.
  \begin{equivlist}
    \item
      Der Ring~$K$ ist ein Körper.
    \item
      Der Polynomring~$K[X]$ ist euklidisch.
    \item
      Der Polynomring~$K[X]$ ist ein Hauptidealring.
  \end{equivlist}
\end{proposition}

\begin{example}
  Ist~$K$ ein Körper, so ist der Polynomring~$K[X]$ ein Hauptidealring, der Polynomring~$K[X,Y]$ allerdings nicht.
\end{example}

\begin{remark}
  Es sei~$R \neq 0$.
  Ein Element~$a \in R$ ist genau dann keine Einheit, wenn das Ideal~$\genideal{X, a}$ in~$R[X]$ kein Hauptideal ist.
\end{remark}

\begin{example}
  Das Ideal~$\genideal{X, 2}$ in~$\Integer[X]$ ist kein Hauptideal, da~$2$ keine Einheit in~$\Integer$ ist.
\end{example}

\begin{remark}
  Man unterscheidet häufig zwischen den folgenden drei, zunehmend größeren, Klassen an Polynomringen:
  \[
    K[X] \,,
    \quad
    \text{$K[X_1, \dotsc, X_n]$ mit~$n \geq 1$} \,,
    \quad
    \text{$K[X_i \suchthat i \in I]$ mit~$I$ beliebig} \,.
  \]
  Diese drei Klassen an Ringen verhalten sich untereinander recht unterschiedlich, aber die Ringe in jeder Klasse verhalten sich relativ gleich.
\end{remark}





\clearpage





\section{Lokalisierung und Quotientenkörper}
\newday{3}



\subsection{Lokalisierungen}

\begin{definition}
  Eine Teilmenge~$S$ von~$R$ ist \defemph{multiplikativ abgeschlossen} falls die folgenden beiden Bedingungen gelten:
  \begin{itemize}
    \item
      Für alle Element~$s_1, s_2 \in S$ gilt auch~$s_1 s_2 \in S$.
    \item
      Es gilt~$1 \in S$.
  \end{itemize}
\end{definition}

\begin{convention}
  Im Folgenden sei~$S$ eine multiplikativ abgeschlossene Teilmenge von~$R$, sofern nicht anders angegeben.
\end{convention}

Auf der Menge~$R \times S$ wird durch
\[
  (r_1, s_1) \sim (r_2, s_2)
  \iff
  \text{es gibt~$t \in S$ mit~$t(r_1 s_2 - r_2 s_1) = 0$}
\]
eine Äquivalenzrelation definiert.
Es sei
\[
  S^{-1} R \defined (R \times S) / {\sim} \,,
\]
und für jedes Paar~$(r,s) \in R \times S$ sei~$r/s \defined \class{(r,s)}$ die Äquivalenzklasse von~$(r,s)$ in~$S^{-1} R$.
Durch die Vorschriften
\[
  \frac{r_1}{s_1} + \frac{r_2}{s_2}
  \defined
  \frac{r_1 s_2 + r_2 s_1}{s_1 s_2} \,,
  \quad
  \frac{r_1}{s_1} \cdot \frac{r_1}{s_1}
  \defined
  \frac{r_1 r_2}{s_1 s_2}
\]
wird~$S^{-1} R$ zu einem Ring, wobei~$0_{S^{-1} R} = 0/1$ und~$1_{S^{-1} R} = 1/1$.
Die Abbildung
\[
  \iota
  \colon
  R \to S^{-1} R \,,
  \quad
  r \mapsto \frac{r}{1}
\]
ist ein Ringhomomorphismus.

\begin{definition}
  In der obigen Situation ist~$S^{-1} R$ die \defemph{Lokalisierung} von~$R$ an~$S$.
  Die Abbildung~$\iota$ ist der \defemph{kanonische Homomorphismus}.
\end{definition}

\begin{proposition}
  \leavevmode
  \begin{enumerate}
    \item
      Für jedes Element~$s \in S$ ist der Bruch~$s/1$ eine Einheit in~$S^{-1} R$ mit~$(s/1)^{-1} = 1/s$.
    \item
      Für jeden Bruch~$r/s \in S^{-1} R$ und jedes Element~$t \in S$ gilt~$r/s = (tr)/(ts)$, d.\,h. Brüche in~$S^{-1} R$ lassen sich durch Element aus~$S$ erweitern und kürzen.
  \end{enumerate}
\end{proposition}

\begin{proposition}
  Es sei~$\iota \colon R \to S^{-1} R$ der kanonische Homomorphismus.
  \begin{enumerate}
    \item
      Es gilt~$\ker(\iota) = \{ r \in R \suchthat \text{es gibt~$s \in S$ mit~$sr = 0$} \}$.
    \item
      Es gilt genau dann~$S^{-1} R = 0$, wenn~$0 \in S$.
  \end{enumerate}
\end{proposition}



\subsection{Quotientenkörper}

\begin{example}
  \label{construction of field of fractions}
  Ist~$R$ ein Integritätsbereich, so ist~$S \defined R \setminus \{ 0 \}$ eine multiplikativ abgeschlossene Teilmenge von~$R$ und die Lokalisierung~$S^{-1} R$ ist ein Körper.
  Der kanonische Homomorphismus~$\iota \colon R \to \Quot(R)$ ist injektiv, weshalb sich~$R$ als ein Unterring von~$\Quot(R)$ auffassen lässt.
\end{example}

\begin{definition}
  In der Situation von \cref{construction of field of fractions} ist der Körper~$S^{-1} R$ der \defemph{Quotientenkörper} von~$R$.
  Er wird mit~$\Quot(R)$ notiert.
\end{definition}
 
%\begin{corollary}
%  Integritätsbereiche sind genau die Unterringe von Körpern.
%\end{corollary}

\begin{example}
  \leavevmode
  \begin{enumerate}
    \item
      Es gilt~$\Quot(\Integer) = \Rational$.
    \item
      Für jede Indexmenge~$I$ ist~$K(X_i \suchthat i \in I) \defined \Quot( K[X_i \suchthat i \in I] )$ der \defemph{Körper der rationalen Funktionen} in den Variablen~$X_i$ mit Koeffizienten in~$K$.
  \end{enumerate}
\end{example}

\begin{example}
  Ist allgemeiner~$\pideal$ ein Primideal in~$R$, so ist das Komplement~$S \defined R \setminus \pideal$ eine multiplikativ abgeschlossene Teilmenge von~$R$.
  Die Lokalisierung~$S^{-1} R$ wird mit~$R_{\pideal}$ notiert.
\end{example}

%\begin{example}
%  Ist~$I$ ein Ideal in~$R$, so ist~$1 + I$ eine multiplikativ abgeschlossene Teilmenge von~$R$.
%\end{example}



\subsection{Universelle Eigenschaft der Lokalisierung}

\begin{theorem}[Universelle Eigenschaft der Lokalisierung]
  Es sei~$\iota \colon R \to S^{-1} R$ der kanonische Homomorphismus.
  Es sei~$R'$ ein weiterer Ring und es sei~$\varphi \colon R \to R'$ ein Ringhomomorphismus.
  Der Ringhomomorphismus~$\varphi \colon R \to R'$ induziert genau dann einen Ringhomomorphismus~$\induced{\varphi} \colon S^{-1} R \to R'$, der das Diagram
  \[
    \begin{tikzcd}
      R
      \arrow{r}[above]{\varphi}
      \arrow{d}[left]{\iota}
      &
      R'
      \\
      S^{-1} R
      \arrow[dashed]{ur}[below right]{\induced{\varphi}}
      &
      {}
    \end{tikzcd}
  \]
  zum kommutieren bringt, wenn für jedes Element~$s \in S$ das Bild~$\varphi(s)$ eine Einheit in~$R'$ ist, d.\,h. falls~$\varphi(S) \subseteq (R')^\times$ gilt.
  Der Ringhomomorphismus~$\induced{\varphi}$ ist eindeutig, und gegeben durch
  \[
    \induced{\varphi}\biggl( \frac{r}{s} \biggr)
    =
    \varphi(r) \spacing \varphi(s)^{-1}
    \qquad
    \text{für alle~$r/s \in S^{-1} R$.}
  \]
\end{theorem}

%TODO: Make the following an exercise.

%\begin{definition}
%  Ein Ringhomomorphismus zwischen zwei Körpern ist ein \defemph{Körperhomomorphismus}.
%\end{definition}
%
%
%\begin{corollary}
%  Es sei~$R$ ein Integritätsbereich.
%  Es sei~$K$ ein Körper und~$\varphi \colon R \to K$ ein injektiver Ringhomomorphismus.
%  Der Homomorphismus~$\varphi$ setzt sich eindeutig zu einem Körperhomomorphismus~$\Quot(R) \to K$ fort, d.h. es gibt einen eindeutigen Körperhomomorphismus~$\induced{\varphi} \colon \Quot(R) \to K$, der das folgende Diagramm zum kommutieren bringt:
%  \[
%    \begin{tikzcd}
%      R
%      \arrow{r}[above]{\varphi}
%      \arrow{d}[left]{i}
%      &
%      K
%      \\
%      \Quot(R)
%      \arrow[dashed]{ur}[below right]{\induced{\varphi}}
%      &
%      {}
%    \end{tikzcd}
%  \]
%\end{corollary}





\section{Faktorielle Ringe}

\begin{convention}
  Im Folgenden sei~$R$ ein Integritätsbereich, sofern nicht anders angegeben.
\end{convention}

\begin{definition}
  Sind~$r_1$,~$r_2$ Elemente von~$R$, so ist~$r_1$ ein \defemph{Teiler} von~$r_2$, bzw.~$r_1$ \defemph{teilt}~$r_2$, wenn es ein Element~$a \in R$ mit~$r_2 = a r_1$ gibt.
  Dies wird dann mit~$r_1 \divides r_2$ notiert.
\end{definition}

\begin{proposition}
  Es seien~$r_1, r_2 \in R$.
  \begin{enumerate}
    \item
      Es gilt genau dann~$r_1 \divides r_2$ wenn~$\genideal{r_1} \supseteq \genideal{r_2}$.
    \item
      Es gilt genau dann~$\genideal{r_1} = \genideal{r_2}$ wenn es eine Einheit~$\varepsilon \in R^\times$ mit~$r_1 = \varepsilon r_2$ gibt.
  \end{enumerate}
\end{proposition}

\begin{example}
  Der Polynomring~$K[X]$ ist ein Hauptidealring und es gilt~$K[X]^\times = K^\times = K \setminus \{0\}$.
  Jedes Ideal in~$K[X]$ ist deshalb von der Form~$\genideal{f}$ für~$f = 0$ oder ein eindeutiges normiertes Polynom~$f \in K[X]$.
\end{example}



\subsection{Primelemente und Irreduzible Elemente}

\begin{definition}
  Es sei~$p$ ein Element von~$R$.
  Es gelte~$p \neq 0$ und~$p$ sei keine Einheit.
  \begin{enumerate}
    \item
      Das Element~$p$ ist \defemph{prim} falls für alle Elemente~$r_1, r_1 \in R$ mit~$p \divides (r_1 r_2)$ bereits~$p \divides r_1$ oder~$p \divides r_2$.
    \item
      Das Element~$p$ ist \defemph{irreduzibel} falls für jede Zerlegung~$p = r_1 r_2$ mit~$r_1, r_2 \in R$ bereits~$r_1 \in R^\times$ oder~$r_2 \in R^\times$ gilt.
  \end{enumerate}
\end{definition}


\begin{lemma}
  Es sei~$p$ ein Element von~$R$.
  \begin{enumerate}
    \item
      Das Element~$p$ ist genau dann prim, wenn~$\genideal{p} \neq 0$ gilt und~$\genideal{p}$ ein Primideal ist.
    \item
      Ein Element~$p$ ist genau dann irreduzibel, wenn~$\genideal{p} \neq 0, R$ gilt, und für jedes Hauptideal~$I$ in~$R$ mit~$\genideal{p} \subseteq I \subseteq R$ bereits~$I = \genideal{p}$ oder~$I = R$ gilt.
  \end{enumerate}
\end{lemma}

\begin{proposition}
  \label{connection between prime and irreducible}
  Es sei~$p$ ein Element von~$R$.
  \begin{enumerate}
    \item
      Ist~$p$ prim, so ist~$p$ auch irreduzibel.
      Allgemeiner impliziert für~$p \neq 0$ jede der folgenden Aussagen die jeweils nächste:
      \begin{equivlist}
        \item
          \label{principal ideal is maximal}
          Das Hauptideal~$\genideal{p}$ ist maximal.
        \item
          Das Hauptideal~$\genideal{p}$ ist prim.
        \item
          Das Element~$p$ ist prim.
        \item
          \label{element is irreducible}
          Das Element~$p$ ist irreduzibel.
      \end{equivlist}
    \item
      Ist~$R$ ein Hauptidealring, so gilt auch die Implikation~\ref{element is irreducible}~$\implies$~\ref{principal ideal is maximal}, d.\,h. für~$R$ sind die obigen Aussagen äquivalent.
  \end{enumerate}
\end{proposition}

\begin{corollary}
  Es sei~$R$ ein Hauptidealring.
  \begin{enumerate}
    \item
      Ein Ideal~$I$ in~$R$ mit~$I \neq 0$ ist genau dann prim, wenn es bereits maximal ist.
    \item
      Ist~$p$ ein irreduzibles Element von~$R$, so ist~$R / \genideal{p}$ ein Körper.
  \end{enumerate}
\end{corollary}

\begin{example}
  \leavevmode
  \begin{enumerate}
%    \item
%      Da~$\Integer$ ein Integritätsbereich ist, ist ein Ideal~$I \neq 0$ in~$\Integer$ genau dann prim, wenn es maximal ist.
%      Wir haben dies bereits zuvor gesehen:
%      Das Ideal~$I$ ist von der Form~$I = \genideal{p}$ für ein Element~$p \in \Integer$ mit~$p \neq 0$.
%      Es gilt~$\Integer / I \cong \Integer_p$ und somit
%      \begin{align*}
%        {}&
%        \text{$I$ ist prim}
%        \\
%        \iff&{}
%        \text{$\Integer_p$ ist ein Integritätsbereich}
%        \\
%        \iff&{}
%        \text{$\Integer_p$ ist ein Körper}
%        \\
%        \iff&{}
%        \text{$I$ is maximal} \,.
%       \end{align*}
     \item
       Ist~$f$ ein irreduzibles Polynom in~$K[X]$, so ist~$K[X] / \genideal{f}$ ein Körper.
     \item
       Das Ideal~$I = \genideal{Y}$ in~$K[X,Y]$ ist prim aber nicht maximal.
       Dies ist kein Widerspruch zu \cref{connection between prime and irreducible}, da~$K[X,Y]$ kein Hauptidealring ist.
  \end{enumerate}
\end{example}



\subsection{Faktorielle Ringe}

\begin{definition}
  Der Integritätsbereich~$R$ ist \defemph{faktoriell} falls die folgenden Bedingungen gelten.
  \begin{itemize}
    \item
      Jedes Element~$r \in R$ mit~$r \neq 0$ besitzt eine Zerlegung
      \begin{equation}
        \label{prime decomposition}
        r = \varepsilon p_1 \dotsm p_n \,,
      \end{equation}
      wobei~$\varepsilon$ eine Einheit ist und~$p_1, \dotsc, p_n$ irreduzibel sind.
    \item
      Diese Zerlegung ist in dem folgenden Sinne \enquote{eindeutig bis auf Einheiten}:
      Es sei
      \[
        r = \varepsilon' p'_1 \dotsm p'_m
      \]
      eine weitere Zerlegung, wobei~$\varepsilon'$ eine Einheit ist und~$p'_1, \dotsc, p'_m$ irreduzibel sind.
      Dann gilt bereits~$n = m$, und es gibt eine Permutation~$\sigma \in \Sym_n$ und Einheiten~$\delta_1, \dotsc, \delta_n$ mit
      \[
        p'_i = \delta_i p_{\sigma(i)} \,.
      \]
      (Es gilt dann notwendigerweise~$\varepsilon = \delta_1 \dotsm \delta_n \varepsilon'$.)
  \end{itemize}
  Ist~$R$ faktoriell, so ist die Zerlegung~\eqref{prime decomposition} die \defemph{Primfaktorzerlegung} von~$r$.
\end{definition}

\begin{proposition}
  Ist der Ring~$R$ faktoriell, so ist ein Element von~$R$ genau dann prim, wenn es irreduzibel ist.
\end{proposition}

\begin{proposition}
  Jeder Hauptidealring ist faktoriell.
\end{proposition}

\begin{example}
  \leavevmode
  \begin{enumerate}
    \item
      Der Ring~$\Integer$ ist faktoriell.
      Ein Element~$p \in \Integer$ ist genau dann prim wenn~$\abs{p}$ eine Primzahl ist.
    \item
      Der Ring der Gaußschen Zahlen,~$\Integer[i]$, ist faktoriell.
    \item
      Der Polynomring~$K[X]$ faktoriell.
  \end{enumerate}
\end{example}

\begin{example}
  Der Ring~$\Integer[\sqrt{-5}] \defined \Integer[i \sqrt{5}] = \{ a + i b \sqrt{5} \suchthat a, b \in \Integer \}$ ist nicht faktoriell.

  Wie bereits in \cref{units in gaussian integers} sei für~$z \in \Integer[\sqrt{-5}]$ mit~$z = a + i b \sqrt{5}$ für~$a, b \in \Integer$ die Norm von~$z$ definiert als
  \[
    N(z)
    \defined
    \abs{z}^2
    =
    a^2 + 5 b^2
    \in
    \Natural \,.
  \]
  Wie in \cref{units in gaussian integers} erhalten wir mithilfe der Multiplikativität der Norm, dass
  \[
    \Integer[\sqrt{-5}]^\times
    =
    \{ 1, -1 \} \,.
  \]
  Inbesondere gilt für~$z \in \Integer[\sqrt{-5}]^\times$ genau dann~$N(z) = 1$, wenn~$z$ eine Einheit ist.

  Das Element~$z \defined 1 + i \sqrt{5}$ ist irreduzibel in~$\Integer[\sqrt{-5}]^\times$:
  Für jede Zerlegung~$z = z_1 z_2$ in zwei Faktoren~$z_1, z_2 \in \Integer[\sqrt{-5}]$ gilt
  \[
    6
    =
    N(z)
    =
    N(z_1 z_2)
    =
    N(z_1) N(z_2) \,.
  \]
  Es gibt kein Element~$z' \in \Integer[\sqrt{-5}]$ mit~$N(z') = 2$ oder~$N(z') = 3$.
  Somit gilt entweder~$N(z_1) = 6$ und~$N(z_2) = 1$, oder~$N(z_1) = 1$ und~$N(z_2) = 6$.
  In jedem Fall gilt für eines der~$z_i$, dass~$N(z_i) = 1$, so dass dieses~$z_i$ eine Einheit in~$\Integer[\sqrt{-5}]$ ist.
  Das zeigt, dass~$z$ tatsächlich irreduzibel ist.

  Wir erhalten auf ähnliche Weise, dass auch die Elemente
  \[
    1 - i \sqrt{5} \,,
    \quad
    2 \,,
    \quad
    3
  \]
  irreduzibel in~$\Integer[\sqrt{-5}]$ sind.

  Wir betrachten nun für~$6 \in \Integer[\sqrt{-5}]$ die beiden Zerlegungen
  \[
    6 = 2 \cdot 3 \,,
    \quad
    6 = ( 1 + i \sqrt{5} ) ( 1 - i \sqrt{5} ) \,.
  \]
  Dies sind zwei Zerlegungen von~$6$ in irreduzibel Element von~$\Integer[\sqrt{-5}]$.
  Da~$\Integer[\sqrt{-5}]^\times = \{1, -1\}$ gilt, sehen wir aber, dass sich diese beiden Zerlegungen nicht \enquote{gleich bis auf Einheiten} sind.
  Also ist~$\Integer[\sqrt{-5}]$ nicht faktoriell.
\end{example}



\subsection{Der Satz von Gauß}

\begin{convention}
  Es sei im Folgenden~$R$ ein faktorieller Ring, sofern nicht anders angegeben.
\end{convention}

\begin{definition}
  Es sei~$f \in R[X]$ ein Polynom gegeben durch~$f = a_n X^n + \dotsb + a_1 X + a_0$.
  Das Polynom~$f$ ist \defemph{primitiv} falls die Koeffizienten~$a_0, \dotsc, a_n$ insgesamt teilerfremd sind, d.\,h. wenn es keine Nicht-Einheit~$r \in R$ gibt, so dass~$r \divides a_0, \dotsc, a_n$ gilt.
\end{definition}

\begin{example}
  \leavevmode
  \begin{enumerate}
    \item
      Jedes normierte Polynom ist primitiv.
    \item
      Ein Polynom~$f \in R[X]$ ist genau dann primitiv, wenn für jede Einheit~$\varepsilon \in R^\times$ das Polynom~$\varepsilon f$ primitiv ist.
    \item
      Jedes Polynom~$f \in K[X]$ mit~$f \neq 0$ ist primitiv.
    \item
      Das Polynom~$4X + 2$ ist nicht primitiv in~$\Integer[X]$, aber primitiv in~$\Rational[X]$.
  \end{enumerate}
\end{example}

\begin{proposition}[Lemma von Gauß]
  Sind~$f, g \in R[X]$ zwei primitive Polynome, so ist auch ihr Produkt~$fg$ wieder primitiv.
\end{proposition}

\begin{theorem}[Satz von Gauß]
  Der Polynomring~$R[X]$ ist wieder faktoriell.
  Die irreduziblen Elemente von~$R[X]$ sind genau die folgenden Polynome:
  \begin{itemize}
    \item
      Die konstanten Polynome, also Elemente von~$R$, welche irreduzibel in~$R$ sind.
    \item
      Die primitiven Polynome, welche irreduzibel in~$\Quot(R)[X]$ ist.
  \end{itemize}
\end{theorem}

\begin{corollary}
  Es sei~$f \in R[X]$ ein Polynom vom Grad~$\deg(f) \geq 1$.
  \begin{enumerate}
    \item
      Ist~$f$ irreduzibel in~$R[X]$, so ist es auch irreduzibel in~$\Quot(R)[X]$.
    \item
      Die Umkehrung gilt genau dann, wenn~$f$ primitiv ist.
  \end{enumerate}
\end{corollary}

\begin{corollary}
  \label{polynomial rings in multiple variables are ufd}
  Für jede endliche Anzahl von Variablen~$n \in \Natural$ ist~$R[X_1, \dotsc, X_n]$ wieder faktoriell.
\end{corollary}

\begin{example}
  Es sei~$n \in \Natural$ eine endliche Anzahl an Variablen.
  \leavevmode
  \begin{enumerate}
    \item
      Ist~$K$ ein Körper, so ist der Polynomring~$K[X_1, \dotsc, X_n]$ faktoriell.
    \item
      Der Ring~$\Integer[X_1, \dotsc, X_n]$ ist faktoriell.
  \end{enumerate}
\end{example}

%\begin{remark}
%  \Cref{polynomial rings in multiple variables are ufd} gilt auch allgemeiner für Polynomringe~$R[X_i \suchthat i \in I]$ in beliebig vielen Variablen.
%  Dies kann man mithilfe der obigen Formulierung von \cref{polynomial rings in multiple variables are ufd} zeigen, da jedes Polynom~$f \in R[X_i \suchthat i \in I]$ nur endlich viele Variablen~$X_{i_1}, \dotsc, X_{i_n}$ enthält, und daher bereits in dem faktoriellen Unterring~$R[X_{i_1}, \dotsc, X_{i_n}]$ von~$R[X_i \suchthat i \in I]$ enthalten ist.
%\end{remark}


\subsection{Irreduziblitätskriterien}

\begin{proposition}
  Es sei~$f \in K[X]$.
  \begin{enumerate}
    \item
      Gilt~$\deg(f) = 1$, so ist~$f$ irreduzibel.
    \item
      Gilt~$\deg(f) = 2$ oder~$\deg(f) = 3$, so ist~$f$ genau dann irreduzibel in~$K[X]$, wenn~$f$ keine Nullstelle in~$K$ hat.
  \end{enumerate}
\end{proposition}

\begin{example}
  \leavevmode
  \begin{enumerate}
    \item
      Das Polynom~$X^2 - 2 \in \Rational[X]$ hat keine Nullstelle in~$\Rational$ und ist somit irreduzibel.
    \item
      Das Polynom~$X^3 - X + 1 \in \Finite_3[X]$ hat keine Nullstelle in~$\Finite_3$, ist also irreduzibel.
    \item
      Das Polynom~$X^4 + 4 \in \Rational[X]$ besitzt zwar keine Nullstelle in~$\Rational$, ist aber nicht irreduzibel, da
      \[
        X^4 + 4
        =
        (X^2 + 2X + 2) (X^2 - 2X + 2) \,.
      \]
    \item
      Das Polynom~$4X + 2 \in \Integer[X]$ ist vom Grad~$1$ und hat auch keine Nullstelle in~$\Integer$, ist aber dennoch nicht irreduzibel.
  \end{enumerate}
\end{example}

\begin{proposition}[Eisenstein-Kriterium]
  Es sei~$f \in R[X]$ ein Polynom vom Grad~$n \geq 1$, gegeben durch~$f = a_n X^n + \dotsb + a_1 X + a_0$.
  Es gebe ein Primelement~$p \in R$ mit
  \[
    p \ndivides a_n \,,
    \quad
    p \divides a_{n-1}, \dotsc, a_1, a_0 \,,
    \quad
    p^2 \ndivides a_0 \,.
  \]
  \begin{enumerate}
    \item
      Das Polynom~$f$ ist irreduzibel in~$\Quot(R)[X]$.
    \item
      Ist das Polynom~$f$ zusätzlicherweise primitiv, so ist es außerdem irreduzibel in~$R[X]$.
  \end{enumerate}
\end{proposition}

\begin{example}
  \leavevmode
  \begin{enumerate}
    \item
      Das Polynom~$5X^6 + 4 X^4 - 6 X^2 - 18 \in \Integer[X]$ ist primitiv, und nach dem Eisenstein-Kriterium mit~$p = 2$ ist es irreduzibel.
    \item
      Das Polynom~$X^5 - t \in \Rational(t)[X]$ ist nach dem Eisenstein-Kriterium mit~$p = t$ irreduzibel, denn es gilt~$\Rational(t) = \Quot(\Rational[t])$ und~$t$ ist prim in~$\Rational[t]$.
  \end{enumerate}
\end{example}

\begin{proposition}
  Es sei~$f \in R[X]$ ein Polynom und es sei~$a \in R$.
  Das Polynom~$f$ ist genau dann irreduzibel, wenn das verschobene Polynom~$f(X + a)$ irreduzibel ist.
\end{proposition}

\begin{example}
  Das Polynom~$f \defined X^4 + 1 \in \Rational[X]$ ist irreduzibel, denn das verschobene Polynom
  \[
    f(X+1)
    =
    (X + 1)^4 + 1
    =
    X^4 + 4 X^3 + 6 X^2 + 4 X + 2
  \]
  ist nach dem Eisenstein-Kriterium (mit~$p = 2$) irreduzibel.
\end{example}

\begin{proposition}[Reduktionskriterium]
  Es sei~$f \in R[X]$ ein Polynom vom Grad~$n \geq 1$ gegeben durch~$f = a_n X^n + \dotsb + a_1 X + a_0$.
  Es sei~$p \in R$ ein Primelement mit~$p \ndivides a_n$ und es sei
  \[
    R[X]
    \to
    (R/\genideal{p})[X] \,,
    \quad
    g
    \mapsto
    \induced{g}
  \]
  der durch die kanonische Projektion~$\pi \colon R \to R / \genideal{p}$ induzierte Ringhomomorphismus.
  \begin{enumerate}
    \item
      Ist das Polynom~$\induced{f}$ irreduzibel in~$(R/\genideal{p})[X]$, so ist~$f$ irreduzibel in~$\Quot(R)[X]$.
    \item
      Ist~$f$ zusätzlicherweise primitiv, so ist~$f$ bereits irreduzibel in~$R[X]$.
  \end{enumerate}
\end{proposition}

\begin{example}
  Das Polynom~$f = X^3 + 9 X^2 - 4 X + 8 \in \Integer[X]$ wird in~$\Finite_3[X]$ zu dem Polynom~$X^3 - X + 2$.
  Dieses Polynom ist vom Grad~$3$ und hat keine Nullstelle in~$\Finite_3$, ist also irreduzibel in~$\Finite_3[X]$.
  Nach dem Reduktionskriterium ist~$f$ irreduzibel in~$\Rational[X]$.
  Das Polynom~$f$ ist außerdem primitiv, da es normiert ist, und somit bereits irreduzibel in~$\Integer[X]$.
\end{example}





