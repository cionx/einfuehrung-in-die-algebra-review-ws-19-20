\documentclass{scrartcl}

\usepackage{style_general}
\usepackage{style_sheets}



\titlehead{
  \textbf{Repetitorium: Einführung in die Algebra}%
%  \footnote{
%    Online verfügbar unter \url{https://gitlab.com/cionx/einfuehrung-in-die-algebra-review-ws-19-20}.
%  }
  \hfill
  4. März 2020
}
\title{\vspace{-1em}Tag 3}
\author{}
\date{}



\begin{document}

\maketitle
\vspace{-7em}

\begin{exercise}
  Zeigen Sie, dass die folgenden Polynome irreduzibel sind.
  \begin{enumerate}
    \item
      $5 X^3 + 35 X^2 - 210 X + 63 \in \Integer[X]$.
    \item
      $2 X^7 + 8 X^5 + 4 X^3 - 8 X + 12 \in \Rational[X]$.
    \item
      $X^3 + 2 X^2 + X + 1 \in \Integer[X]$.
    \item
      $X^3 + 3 X^2 - 3X + 2 \in \Integer[X]$.
    \item
      $X^2 Y + X Y^2 - X - Y + 1 \in \Rational[X, Y]$.
  \end{enumerate}
\end{exercise}

\begin{solution}
  \begin{enumerate}
    \item
      Nach Eisenstein (mit~$p = 7$) ist das gegebene Polynom irreduzibel in~$\Rational[X]$.
      Das Polynom ist primitv, und somit bereits irreduzibel in~$\Integer[X]$.
    \item
      Da~$2$ in~$\Rational$ eine Einheit ist, dürfen wir das gegebene Polynom~$f$ durch das Polynom~$f/2$ ersetzen.
      Es gilt
      \[
        \frac{f}{2}
        =
        X^7 + 4 X^5 + 2 X^3 - 4 X + 6 \,.
      \]
      Dieses Polynom ist nun nach dem Eisenstein-Kriterium (mit~$p = 2$) irreduzibel in~$\Rational[X]$.
    \item
     Reduktion bezüglich~$p = 2$ ergibt das Polynom
     \[
       X^3 + X + 1 \in \Integer_2[X] \,.
     \]
     Dieses Polynom ist kubisch und hat keine Nullstelle in~$\Integer_2$, ist also irreduzibel in~$\Integer_2[X]$ (denn~$\Integer_2$ ist ein Körper).
     Nach dem Reduktionskriterium ist somit auch das ursprüngliche Polynom irreduzibel in~$\Rational[X]$.
     Da es normiert, und somit primitiv, ist, ist es bereits irreduzibel in~$\Integer[X]$.
   \item
     Wir bezeichnen das gegebene Polynom mit~$f$.
     Es gilt
     \begin{align*}
       f(X + 1)
       &=
       (X+1)^3 + 3 (X+1)^2 - 3 (X+1) + 2
       \\
       &=
       (X^3 + 3 X^2 + 3 X + 1) + (3 X^2 + 6 X + 3) - (3X + 3) + 2
       \\
       &=
       X^3 + 6 X^2 + 6 X + 3 \,.
     \end{align*}
     Dieses Polynom ist normiert und somit primitiv, und wir erhalten aus dem Eisenstein-\hspace{0pt}Kriterium (mit~$p = 3$), dass es irreduzibel in~$\Integer[X]$ ist.
     Somit ist auch~$f$ irreduzibel.
   \item
     Wir nutzen den Isomorphismus~$\Rational[X,Y] \cong \Rational[X][Y]$, und schreiben das gegebene Polynom als
     \[
       X Y^2 + (X^2 - 1) Y + (-X + 1) \,.
     \]
     Als Element von~$\Rational[X][Y]$ ist dieses Polynom primitiv, denn die Koeffiezenten~$X$ und~$-X+1$ sind teilerfremd in~$\Rational[X]$.
     Es gilt~$(-X+1) \divides (X^2 - 1)$ da~$1$ eine Nullstelle von~$X^2 - 1$ ist, aber es gilt auch~$(-X+1) \ndivides X$.
     Dabei ist das Polynom~$-X+1$ prim in~$\Rational[X]$ es Grad~$1$ hat.
     Wir können daher das Eisensteinkriterium (mit~$p = -X+1$) anwenden, um die gewünschte Irreduziblität zu beweisen.
  \end{enumerate}
\end{solution}

\begin{exercise}[subtitle = {Erstklausur~19/20}]
  \begin{enumerate}
    \item
      Entscheiden Sie, ob der Faktorring~$\Rational[X] / \genideal{X^3 - 4X + 2}$ ein Körper ist.
    \item
      Geben Sie eine~\Basis{$\Rational$} von~$\Rational(\sqrt[4]{2})$ an.
  \end{enumerate}
\end{exercise}

\begin{exercise}[subtitle = {Erstklausur~18/19, Zweitklausur~18/19}]
  Es sei~$n \in \Natural_1$ und es sei~$\zeta_n \defined e^{2 \pi i / n}$.
  Beweisen oder widerlegen Sie, dass~$[\Rational(\zeta_n) : \Rational] = n-1$.
\end{exercise}

\begin{exercise}
  \begin{enumerate}
    \item
      Zeigen Sie, dass für alle~$n, m \geq 1$ der Körper~$\Rational(\sqrt[n]{2})$ genau dann ein Unterkörper von~$\Rational(\sqrt[m]{2})$ ist, wenn~$n \divides m$ gilt.
    \item
      Es sei~$p$ eine Primzahl und~$L/K$ eine Körpererweiterung vom Grad~$[L : K] = p$.
      Zeigen Sie, dass die Erweiterung~$L/K$ einfach ist, und bestimmen sie alle Elemente~$\alpha \in L$ mit~$L = K(\alpha)$.
    \item
      Es sei~$K(\alpha)/K$ eine einfache Körpererweiterung deren Grad~$[K(\alpha) : K]$ ungerade ist.
      Zeigen Sie, dass~$K(\alpha) = K(\alpha^2)$ gilt.
    \item
      Es sei~$L/K$ eine Körpererweiterung vom Grad~$[L : K] = 2^n$ für ein~$n \in \Natural$.
      Es sei~$f \in K[X]$ ein kubisches Polynom, das eine Nullstelle in~$L$ hat.
      Zeigen Sie, dass~$f$ bereits eine Nullstelle in~$K$ hat.
  \end{enumerate}
\end{exercise}

\begin{exercise}
  \begin{enumerate}
    \item
      Bestimmen Sie das Minimalpolynom von~$\sqrt[3]{2}$ über~$\Rational$.
    \item
      Bestimmen Sie das Minimalpolynom von~$\zeta_3 \defined e^{2 \pi i / 3}$ über~$\Rational(\sqrt[3]{2})$.
    \item
      Bestimmen Sie den Grad der Körpererweiterung~$\Rational(\sqrt[3]{2}, \zeta_3) / \Rational$.
  \end{enumerate}
\end{exercise}

\begin{exercise}[subtitle = {Erstklausur~19/20}]
  Es sei~$f = X^5 - 14 X^3 - 21 X^2 + 49 X + 28 \in \Rational[X]$.
  \begin{enumerate}
    \item
      Zeigen Sie, dass das Polyom~$f$ irreduzibel ist.
    \item
      Bestimmen Sie den Grad der Körpererweiterung~$\Rational(\sqrt[4]{3}, i) / \Rational$.
    \item
      Entscheiden Sie, ob~$f$ eine Nullstelle in~$\Rational(\sqrt[4]{3}, i)$ hat.
  \end{enumerate}
\end{exercise}

\begin{exercise}[subtitle = {Erstklausur~18/19}]
  Es sei~$L \defined \Rational(\sqrt[4]{2}, \omega)$ mit~$\omega \defined e^{2 \pi i / 8}$.
  Bestimmen Sie~$[L : \Rational]$.
  (\emph{Tipp}: Es gilt~$\omega + \omega^{-1} = \sqrt{2}$.)
\end{exercise}

\begin{exercise}
  Zeigen Sie, dass die folgenden Polynome irreduzibel sind.
  \begin{enumerate}
    \item
      $X Y^3 + X^2 Y + 3 X Y^2 + X^2 + 3 X Y + 2 X + Y + 2 \in \Rational[X]$.
    \item
      $X^3 + Y^3 + X^2 Y + X Y^2 + XY + 6X + 6Y \in \Rational[X,Y]$.
  \end{enumerate}
\end{exercise}

\begin{exercise}
  Es sei~$R$ ein kommutativer Ring und~$I$ ein Ideal in~$R$ mit Komplement~$S \defined R \setminus I$.
  Zeigen Sie, dass das Ideal~$I$ genau dann prim ist, wenn die Menge~$S$ multiplikativ abgeschlossen ist.
\end{exercise}

\begin{exercise}
  Es sei~$R$ ein kommutativer Ring und es sei~$\pideal$ ein Ideal in~$R$.
  Zeigen Sie, dass das Ideal~$\pideal$ genau dann prim ist, wenn es einen Körper~$K$ und einen Ringhomomorphismus~$\varphi \colon R \to K$ gibt, so dass~$\pideal = \ker(\varphi)$ gilt.
\end{exercise}





\clearpage




\printsolutions





\end{document}
