\documentclass{scrartcl}

\usepackage{style_general}
\usepackage{style_sheets}



\titlehead{
  \textbf{Repetitorium: Einführung in die Algebra}%
%  \footnote{
%    Online verfügbar unter \url{https://gitlab.com/cionx/einfuehrung-in-die-algebra-review-ws-19-20}.
%  }
  \hfill
  4. März 2020
}
\title{\vspace{-1em}Tag 3}
\author{}
\date{}



\begin{document}

\maketitle
\vspace{-7em}

\begin{exercise}
  Zeigen Sie, dass die folgenden Polynome irreduzibel sind.
  \begin{enumerate}
    \item
      $5 X^3 + 35 X^2 - 210 X + 63 \in \Integer[X]$.
    \item
      $2 X^7 + 8 X^5 + 4 X^3 - 8 X + 12 \in \Rational[X]$.
    \item
      $X^3 + 2 X^2 + X + 1 \in \Integer[X]$.
    \item
      $X^3 + 3 X^2 - 3X + 2 \in \Integer[X]$.
    \item
      $X^2 Y + X Y^2 - X - Y + 1 \in \Rational[X, Y]$.
  \end{enumerate}
\end{exercise}

\begin{solution}
  \begin{enumerate}
    \item
      Nach Eisenstein mit~$p = 7$ ist das gegebene Polynom irreduzibel in~$\Rational[X]$.
      Das Polynom ist primitv, und somit bereits irreduzibel in~$\Integer[X]$.
    \item
      Da~$2$ eine Einheit in~$\Rational$ ist, dürfen wir das gegebene Polynom~$f$ durch das Polynom
      \[
        \frac{f}{2}
        =
        X^7 + 4 X^5 + 2 X^3 - 4 X + 6 \,.
      \]
      ersetzen.
      Dieses Polynom ist nun nach dem Eisenstein-Kriterium mit~$p = 2$ irreduzibel in~$\Rational[X]$.
    \item
     Reduktion bezüglich des Primelements~$p = 2$ von~$\Integer$ ergibt das Polynom
     \[
       X^3 + X + 1 \in \Finite_2[X] \,.
     \]
     Dieses Polynom ist kubisch und hat keine Nullstelle in~$\Finite_2$, ist also irreduzibel in~$\Finite_2[X]$ (denn~$\Finite_2$ ist ein Körper).
     Nach dem Reduktionskriterium ist somit auch das ursprüngliche Polynom irreduzibel in~$\Rational[X]$.
     Es ist bereits irreduzibel in~$\Integer[X]$, denn es primitiv ist, da es normiert ist.
   \item
     Wir bezeichnen das gegebene Polynom mit~$f$.
     Es gilt
     \begin{align*}
       f(X + 1)
       &=
       (X+1)^3 + 3 (X+1)^2 - 3 (X+1) + 2
       \\
       &=
       (X^3 + 3 X^2 + 3 X + 1) + (3 X^2 + 6 X + 3) - (3X + 3) + 2
       \\
       &=
       X^3 + 6 X^2 + 6 X + 3 \,.
     \end{align*}
     Wir erhalten aus dem Eisenstein-Kriterium mit~$p = 3$, dass~$f(X+1)$ irreduzibel in~$\Rational[X]$.
     Das Polynom~$f(X+1)$ ist primitiv, da es normiert ist, und deshalb auch schon irreduzibel in~$\Integer[X]$.
     Es folgt, dass auch~$f$ irreduzibel in~$\Integer[X]$ ist.
   \item
     Wir nutzen den Isomorphismus~$\Rational[X,Y] \cong R[Y]$ für den faktoriellen Ring~$R \defined \Rational[X]$, und schreiben das gegebene Polynom als
     \[
       X Y^2 + (X^2 - 1) Y + (-X + 1) \,.
     \]
     Als Element von~$R[Y]$ ist dieses Polynom primitiv, denn die Koeffizienten~$X$ und~$-X+1$ sind teilerfremd in~$\Rational[X]$.
     Das Element~$p \defined X-1$ von~$R$ ist prim und es gilt~$p \divides (-X+1)$.
     Es gilt~$p \divides (X^2 - 1)$ da~$1$ eine Nullstelle von~$X^2 - 1$ ist, bzw. da~$X^2 - 1 = (X+1)(X-1)$ gilt.
     Aber es gilt auch~$p \ndivides X$.
     Es folgt nun aus dem Eisenstein-Kriterium, dass das Polynom~$f$ irreduzibel in~$R[Y]$ ist, und somit auch irreduzibel in~$K[X,Y]$.
  \end{enumerate}
\end{solution}

\begin{exercise}[subtitle = {Erstklausur~19/20}]
  \begin{enumerate}
    \item
      Entscheiden Sie, ob der Faktorring~$\Rational[X] / \genideal{X^3 - 4X + 2}$ ein Körper ist.
    \item
      Geben Sie eine~\Basis{$\Rational$} von~$\Rational(\sqrt[4]{2})$ an.
  \end{enumerate}
\end{exercise}

\begin{solution}
  \begin{enumerate}
    \item
      Das Polynom~$X^3 - 4X + 2 \in \Rational[X]$ ist irreduzibel nach dem Eisenstein-Kriterium mit~$p = 2$.
      Da~$\Rational[X]$ ein Hauptidealring ist, ist der Faktorring~$\Rational[X] / \genideal{X^3 - 4X + 2}$ deshalb ein Körper.
    \item
      Das gegebene Element~$\sqrt[4]{2}$ ist eine Nullstelle des Polynoms~$X^4 - 2 \in \Rational[X]$.
      Dieses Polynom ist normiert und ist nach dem Eisenstein-Kriterium mit~$p = 2$ auch irreduzibel.
      Es ist also bereits das Minimalpolynom von~$\sqrt[4]{2}$ über~$\Rational$.
      Es folgt, dass
      \[
        [\Rational(\sqrt[4]{2}) : \Rational]
        =
        \deg(X^4 - 2)
        =
        4 \,.
      \]
      Hieraus folgt ferner, dass die Elemente
      \[
        1 \,,
        \quad
        \sqrt[4]{2} \,,
        \quad
        \sqrt[4]{2}^2 = \sqrt{2} \,,
        \quad
        \sqrt[4]{2}^3 \,,
      \]
      eine~\Basis{$\Rational$} von~$\Rational(\sqrt[4]{2})$ sind.
  \end{enumerate}
\end{solution}


\begin{exercise}[subtitle = {Erstklausur~18/19, Zweitklausur~18/19}]
  Es sei~$n \in \Natural_1$ und es sei~$\zeta_n \defined e^{2 \pi i / n}$.
  Beweisen oder widerlegen Sie, dass~$[\Rational(\zeta_n) : \Rational] = n-1$.
\end{exercise}

\begin{solution}
  Für~$n = 4$ gilt~$\zeta_4 = i$.
  Das Minimalpolynom von~$i$ über~$\Rational$ ist~$X^2 + 1 \in \Rational[X]$, denn dieses Polynom ist normiert, hat~$i$ als Nullstelle, und ist irreduzibel in~$\Rational[X]$, da es quadratisch ist und keine Nullstelle in~$\Rational$ hat.
  Deshalb gilt
  \[
    [\Rational(\zeta_4) : \Rational]
    =
    [\Rational(i) : \Rational]
    =
    \deg(X^2 + 1)
    =
    2 \,.
  \]
  Die gegebene Behauptung ist also falsch.
\end{solution}

\begin{exercise}
  \begin{enumerate}
    \item
      Zeigen Sie, dass für alle~$n, m \geq 1$ der Körper~$\Rational(\sqrt[n]{2})$ genau dann ein Unterkörper von~$\Rational(\sqrt[m]{2})$ ist, wenn~$n \divides m$ gilt.
    \item
      Es sei~$p$ eine Primzahl und~$L/K$ eine Körpererweiterung vom Grad~$[L : K] = p$.
      Zeigen Sie, dass die Erweiterung~$L/K$ einfach ist, und bestimmen sie alle Elemente~$\alpha \in L$ mit~$L = K(\alpha)$.
    \item
      Es sei~$K(\alpha)/K$ eine einfache Körpererweiterung deren Grad~$[K(\alpha) : K]$ ungerade ist.
      Zeigen Sie, dass~$K(\alpha) = K(\alpha^2)$ gilt.
    \item
      Es sei~$L/K$ eine Körpererweiterung vom Grad~$[L : K] = 2^n$ für ein~$n \in \Natural$.
      Es sei~$f \in K[X]$ ein kubisches Polynom, das eine Nullstelle in~$L$ hat.
      Zeigen Sie, dass~$f$ bereits eine Nullstelle in~$K$ hat.
  \end{enumerate}
\end{exercise}

\begin{solution}
  \begin{enumerate}
    \item
      Gilt~$n \divides m$, so gilt~$\sqrt[n]{2} = ( \sqrt[m]{2} )^{m/n} \in \Rational( \sqrt[m]{2} )$ und somit auch~$\Rational( \sqrt[n]{2} ) \subseteq \Rational( \sqrt[m]{2} )$.

      Das Element~$\sqrt[n]{2}$ ist eine Nullstelle des Polynoms~$X^n - 2 \in \Rational[X]$.
      Dieses Polynom ist normiert, und es ist irreduzibel nach dem Eisenstein-Kriterium mit~$p = 2$.
      Also ist es das Minimalpolynom von~$\sqrt[n]{2}$ über~$\Rational$.
      Wir erhalten somit, dass
      \[
        [\Rational(\sqrt[n]{2}) : \Rational]
        =
        \deg(X^n - 2)
        =
        n \,.
      \]

      Gilt nun~$\Rational( \sqrt[n]{2} ) \subseteq \Rational( \sqrt[m]{2} )$, so ist der Körper~$\Rational( \sqrt[n]{2} )$ ein Zwischenkörper der Körpererweiterung~$\Rational( \sqrt[m]{2} ) / \Rational$.
      Der Grad~$n = [ \Rational( \sqrt[n]{2} ) : \Rational ]$ ist damit nach der Gradformel ein Teiler des Grads~$m = [ \Rational( \sqrt[m]{2} ) : \Rational ]$.
  \end{enumerate}
\end{solution}

\begin{exercise}
  \begin{enumerate}
    \item
      Bestimmen Sie das Minimalpolynom von~$\sqrt[3]{2}$ über~$\Rational$.
    \item
      Bestimmen Sie das Minimalpolynom von~$\zeta_3 \defined e^{2 \pi i / 3}$ über~$\Rational(\sqrt[3]{2})$.
    \item
      Bestimmen Sie den Grad der Körpererweiterung~$\Rational(\sqrt[3]{2}, \zeta_3) / \Rational$.
  \end{enumerate}
\end{exercise}

\begin{solution}
  \begin{enumerate}
    \item
      Das Element~$\sqrt[3]{2}$ ist eine Nullstelle des Polynoms~$X^3 - 2 \in \Rational[X]$.
      Dieses Polynom ist normiert, und es ist irreduzibel nach dem Eisenstein-Kriterium mit~$p = 2$.
      Also ist es bereits das Minimalpolynom von~$\sqrt[3]{2}$ über~$\Rational$.
    \item
      Es sei~$Z \defined \Rational(\sqrt[3]{2})$.
      Die Einheitswurzel~$\zeta_3$ ist eine Nulllstelle des Polynoms~$X^3 - 1 \in Z[X]$, und es gilt
      \[
        X^3 - 1
        =
        (X - 1)(X^2 + X + 1) \,.
      \]
      Also ist~$\zeta_3$ bereits eine Nullstelle des Polynoms~$f = X^2 + X + 1 \in Z[X]$.
      Dieses Polynom hat keine Nullstellen in~$Z$, denn~$Z$ ist ein Unterkörper von~$\Real$, aber
      \[
        f = \biggl( X + \frac{1}{2} \biggr)^2 + \frac{3}{4}
      \]
      hat keine reellen Nullstellen.
      Es folgt, dass~$f$ irreduzibel ist, da~$f$ quadratisch ist.
      Da~$f$ außerdem normiert ist, folgt nun, dass es bereits das Minimalpolynom von~$\zeta_3$ über~$Z$ ist.
    \item
      Es gelten
      \[
        [\Rational(\sqrt[3]{2}) : \Rational]
        =
        \deg(X^3 - 2)
        =
        3 \,,
        \quad
        [\Rational(\sqrt[3]{2}, \zeta_3) : \Rational(\sqrt[3]{2})]
        =
        \deg(X^2 + X + 1)
        =
        2 \,,
      \]
      und somit nach der Gradformel
      \[
        [\Rational(\sqrt[3]{2}, \zeta_3) : \Rational]
        =
        [\Rational(\sqrt[3]{2}, \zeta_3) : \Rational(\sqrt[3]{2})]
        \cdot
        [\Rational( \sqrt[3]{2} ) : \Rational]
        =
        2 \cdot 3
        =
        6 \,.
      \]
  \end{enumerate}
\end{solution}

\begin{exercise}[subtitle = {Erstklausur~19/20}]
  Es sei~$f = X^5 - 14 X^3 - 21 X^2 + 49 X + 28 \in \Rational[X]$.
  \begin{enumerate}
    \item
      Zeigen Sie, dass das Polyom~$f$ irreduzibel ist.
    \item
      Bestimmen Sie den Grad der Körpererweiterung~$\Rational(\sqrt[4]{3}, i) / \Rational$.
    \item
      Entscheiden Sie, ob~$f$ eine Nullstelle in~$\Rational(\sqrt[4]{3}, i)$ hat.
  \end{enumerate}
\end{exercise}

\begin{solution}
  \begin{enumerate}
    \item
      Das Polynom~$f$ ist irreduzibel nach dem Eisenstein-Kriterium mit~$p = 7$.
    \item
      Das Element~$\sqrt[4]{3}$ ist eine Nullstelle des Polynoms~$X^4 - 3 \in \Rational[X]$.
      Dieses Polynom ist irreduzibel nach dem Eisenstein-Kriterium mit~$p = 3$, und es ist normiert.
      Also ist es bereits das Minimalpolynom von~~$\sqrt[4]{3}$ über~$\Rational$.

      Es sei~$Z \defined \Rational( \sqrt[4]{3} )$.
      Die imaginäre Einheit~$i$ ist eine Nullstelle des Polynoms~$X^2 + 1 \in Z[X]$.
      Dieses Polynom hat keine Nullstelle in~$Z$, denn es hat keine Nullstelle in~$\Real$, und~$Z$ ist ein Unterkörper von~$\Real$.
      Es folgt, dass~$X^2 + 1$ irreduzibel in~$Z[X]$ ist, da es sich um ein quadratisches Polynom handelt. 
      Außerdem ist das Polynom normiert.
      Es folgt, dass~$X^2 + 1$ das Minimalpolynom von~$i$ über~$Z$ ist.

      Es folgt nun, dass
      \[
        [\Rational(\sqrt[4]{3}, i) : Z]
        =
        \deg(X^2 + 1)
        =
        2 \,,
        \quad
        [Z : \Rational]
        =
        \deg(X^4 - 3)
        = 4 \,,
      \]
      und deshalb nach der Gradformel
      \[
        [ \Rational(\sqrt[4]{3}, i) : \Rational ]
        =
        [ \Rational(\sqrt[4]{3}, i) : Z ]
        \cdot
        [ Z : \Rational ]
        =
        2 \cdot 4
        =
        8 \,.
      \]
    \item
      Wir nehmen an, dass~$f$ eine Nullstelle~$\alpha$ in~$\Rational(\sqrt[4]{3})$ hat.
      Dann ist~$f$ bereits das Minimalpolynom von~$\alpha$ über~$\Rational$, denn~$f$ ist irreduzibel und normiert.
      Also gilt
      \[
        [\Rational(\alpha) : \Rational]
        =
        \deg(f)
        =
        5 \,.
      \]
      Andererseits ist~$\Rational(\alpha)$ ein Zwischenkörper der Erweiterung~$\Rational(\sqrt[4]{3}, i) / \Rational$.
      Wir erhalten deshalb aus der Gradformel, dass der Grad~$[\Rational(\alpha) : \Rational] = 5$ ein Teiler des Grades~$[\Rational(\sqrt[4]{3}, i) : \Rational] = 8$ ist.
      Das stimmt aber nicht!
  \end{enumerate}
\end{solution}

\begin{exercise}[subtitle = {Erstklausur~18/19}]
  Es sei~$L \defined \Rational(\sqrt[4]{2}, \omega)$ mit~$\omega \defined e^{2 \pi i / 8}$.
  Bestimmen Sie~$[L : \Rational]$.
  (\emph{Tipp}: Es gilt~$\omega + \omega^{-1} = \sqrt{2}$.)
\end{exercise}

\begin{exercise}
  Zeigen Sie, dass die folgenden Polynome irreduzibel sind.
  \begin{enumerate}
    \item
      $X Y^3 + X^2 Y + 3 X Y^2 + X^2 + 3 X Y + 2 X + Y + 2 \in \Rational[X]$.
    \item
      $X^3 + Y^3 + X^2 Y + X Y^2 + XY + 6X + 6Y \in \Rational[X,Y]$.
  \end{enumerate}
\end{exercise}

\begin{exercise}
  Es sei~$R$ ein kommutativer Ring und~$\pideal$ ein Ideal in~$R$ mit Komplement~$S \defined R \setminus \pideal$.
  Zeigen Sie, dass das Ideal~$\pideal$ genau dann prim ist, wenn die Menge~$S$ multiplikativ abgeschlossen ist.
\end{exercise}

\begin{solution}
  Das Ideal~$\pideal$ ist genau dann prim, wenn~$\pideal \neq R$ gilt, und für je zwei Elemente~$r_1, r_2 \in R$ im~$r_1 r_2 \in \pideal$ auch schon~$r_1 \in \pideal$ oder~$r_2 \in \pideal$ gilt.
  Die Teilmenge~$S$ von~$R$ ist genau dann multiplikativ abgeschlossen, wenn~$1 \in S$ gilt, und für je zwei Elemente~$s_1, s_2 \in S$ auch~$s_1 s_2 \in S$.

  Die Bedingung~$\pideal \neq R$ ist äquivalent dazu, dass~$1 \notin \pideal$, dass also~$1 \in S$.
  Die Bedingung
  \[
    r_1 r_2 \in \pideal
    \implies
    \text{$r_1 \in \pideal$ oder~$r_2 \in \pideal$}
  \]
  lässt sich äquivalenterweise als
  \[
    r_1, r_2 \notin \pideal \implies r_1 r_2 \notin \pideal
  \]
  formulieren, und somit als
  \[
    s_1, s_2 \in S \implies s_1 s_2 \in S \,.
  \]
  Insgesamt haben wir somit gezeigt, dass das Ideal~$\pideal$ genau dann prim ist, wenn das Komplement~$S = R \setminus \pideal$ eine multiplikativ abgeschlossene Teilmenge von~$R$ ist.
\end{solution}

\begin{exercise}
  Es sei~$R$ ein kommutativer Ring und es sei~$\pideal$ ein Ideal in~$R$.
  Zeigen Sie, dass das Ideal~$\pideal$ genau dann prim ist, wenn es einen Körper~$K$ und einen Ringhomomorphismus~$\varphi \colon R \to K$ gibt, so dass~$\pideal = \ker(\varphi)$ gilt.
\end{exercise}

\begin{solution}
  Ist das Ideal~$\pideal$ prim, so ist der Faktorring~$R/\pideal$ eine Integritätsbereich und die kanonische Projektion~$\pi \colon R \to R/\pideal$ ist ein Ringhomomorphismus mit~$\ker(\pi) = \pideal$.
  Da~$R/\pideal$ ein Integritätsbereich ist, können wir seinen Quotientenkörper~$\Quot(R/\pideal)$ bilden, sowie die Inklusion~$\iota \colon R/\pideal \to \Quot(R/\pideal)$ betrachten.
  Die Verknüpfung
  \[
    \iota \circ \pi
    \colon
    R
    \to
    \Quot(R/\pideal)
  \]
  ist ein Ringhomomorphismus, und aufgrund der Injektivität von~$\iota$ gilt, dass
  \[
    \ker(\iota \circ \pi)
    =
    \ker(\pi)
    =
    \pideal
  \]
  Wir können also~$K = \Quot(R/\pideal)$ und~$\varphi = \iota \circ \pi$ wählen.

  Umgekehrterweise gebe es nun einen Körper~$K$ und einen Ringhomomorphismus~$\varphi \colon R \to K$ mit~$\pideal = \ker(\varphi)$.
  Dann ist~$\im(\varphi)$ ist ein Unterring des Körpers~$K$, und somit ein Integritätsbereich.
  Es gilt
  \[
    R / \pideal
    =
    R / {\ker(\varphi)}
    \cong
    \im(\varphi) \,,
  \]
  weshalb auch~$R/\pideal$ ein Integritätsbereich ist.
  Dies ist äquivalent dazu, dass~$\pideal$ ein Primideal ist.
\end{solution}




\clearpage




\printsolutions





\end{document}
