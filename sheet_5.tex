\documentclass{scrartcl}

\usepackage{style_general}
\usepackage{style_sheets}



\titlehead{
  \textbf{Repetitorium: Einführung in die Algebra}%
%  \footnote{
%    Online verfügbar unter \url{https://gitlab.com/cionx/einfuehrung-in-die-algebra-review-ws-19-20}.
%  }
  \hfill
  6. März 2020
}
\title{\vspace{-1em}Tag 5}
\author{}
\date{}



\begin{document}

\maketitle
\vspace{-7em}

\begin{exercise}[subtitle = {Erstklausur~18/19, Zweitklausur~18/19}]
  Entscheiden Sie für die folgenden natürlichen Zahlen jeweils, ob es einen Körper~$K$ mit dieser Anzahl an Elementen gibt:
  \[
    1 \,,
    \quad
    64 \,,
    \quad
    72 \,,
    \quad
    125 \,,
    \quad
    8192 \,
    \quad
    143 \,,
    \quad
    343 \,,
    \quad
    46 \,,
    \quad
    12312312312312312 \,.
  \]
\end{exercise}

\begin{exercise}[subtitle = {Erstklausur~18/19, Zweitklausur~19/20}]
  Entscheiden Sie jeweils, ob die folgenden Aussagen wahr oder falsch sind.
  \begin{enumerate}
    \item
      Es sei~$f \in \Rational[X]$ irreduzibel mit~$n \defined \deg(f)$.
      Dann gilt~$\card{ \Gal(f) } = n$.
    \item
      Ist~$f \in \Rational[X]$ ein Polynom vom Grad~$n$, so ist~$\card{ \Gal(f) }$ ein Teiler von~$n!$.
    \item
      Ist~$K(\alpha)/K$ eine einfache Körpererweiterung, so ist~$\Gal( K(\alpha) / K )$ zyklisch.
    \item
      Ist~$K(\alpha)/K$ eine einfache Körpererweiterung, so ist~$\Gal( K(\alpha) / K )$ abelsch.
    \item
      Jede Sylow-Untergruppe einer auflösbaren~\Gruppe{$p$} ist auflösbar.
    \item
      Ist~$K$ ein endlicher Körper und~$L/K$ eine endliche Körpererweiterung, so ist diese Erweiterung~$L/K$ eine Galois-Erweiterung.
  \end{enumerate}
\end{exercise}

\begin{exercise}
  Entscheiden Sie, für welche natürlichen Zahlen~$n$ mit~$3 \leq n \leq 34$ man das reguläre~\Eck{$n$} mit Zirkel und Lineal (aus den beiden Punkten~$0,1 \in \Complex$) konstruieren kann.
\end{exercise}

\begin{exercise}[subtitle = {Erstklausur~18/19}]
  Es sei~$f \in \Rational[X]$ ein irreduzibles Polynom vom Grad~$4$, so dass~$\Gal(f) \cong \Sym_4$ gilt.
  Ferner sei~$L$ ein Zerfällungskörper von~$f$ über~$\Rational$.
  Zeigen Sie, dass es einen Zwischenkörper~$Z$ der Erweiterung~$L / \Rational$ gibt, so dass~$[Z : \Rational] = 3$ gilt.
\end{exercise}

\begin{exercise}
  Zeigen Sie, dass jede~\Gruppe{$p$}~$G$ auflösbar ist.
  (\emph{Tipp:} Nutzen Sie, dass~$\zenter(G) \neq 1$, falls~$G \neq 1$.)
\end{exercise}

\begin{exercise}[subtitle = {Erstklausur~19/20}]
  Es sei~$f \defined X^5 - 14 X^3 - 21 X^2 + 49 X + 28 \in \Rational[X]$.
  Es sei~$\alpha \in \Complex$ eine Nullstelle von~$f$.
  Zeigen Sie, dass~$\alpha$ nicht durch Zirkel und Lineal konstruierbar ist.
\end{exercise}

\begin{exercise}
  Es sei~$n \in \Natural_1$ und~$K$ ein Körper mit~$\ringchar(K) \ndivides n$.
  Zeigen Sie, dass~$\card{ \Ein_n(K) } = n$ gilt.
\end{exercise}

\end{document}
