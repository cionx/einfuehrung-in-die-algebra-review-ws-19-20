\documentclass{scrartcl}

\usepackage{style_general}
\usepackage{style_sheets}



\titlehead{
  \textbf{Repetitorium: Einführung in die Algebra}%
%  \footnote{
%    Online verfügbar unter \url{https://gitlab.com/cionx/einfuehrung-in-die-algebra-review-ws-19-20}.
%  }
  \hfill
  6. März 2020
}
\title{\vspace{-1em}Tag 5}
\subtitle{Verbesserte Version}
\author{}
\date{}



\begin{document}

\maketitle
\vspace{-7em}

\begin{exercise}[subtitle = {Erstklausur~18/19, Zweitklausur~18/19}]
  Entscheiden Sie für die folgenden natürlichen Zahlen jeweils, ob es einen Körper~$K$ mit dieser Anzahl an Elementen gibt:
  \[
    1 \,,
    \quad
    64 \,,
    \quad
    72 \,,
    \quad
    125 \,,
    \quad
    8192 \,,
    \quad
    143 \,,
    \quad
    343 \,,
    \quad
    46 \,,
    \quad
    12312312312312312 \,.
  \]
\end{exercise}

\begin{solution}
  Für eine beliebige natürliche Zahl~$q \in \Natural$ gibt es genau dann einen Körper mit~$q$ Elementen, falls~$q$ eine nicht-triviale Potenz einer Primzahl ist, d.\,h. falls es eine Primzahl~$p$ und einen Exponenten~$n \in \Natural_1$ gibt, so dass~$q = p^n$ gilt.
  \begin{enumerate}
    \item
      Es gibt keinen Körper mit nur einem Element, da~$1$ keine nicht-triviale Potenz einer Primzahl ist.
      Alternativ gilt für jeden Körper~$K$, dass~$\card{K} \geq 2$, da für die beiden Elemente~$0, 1 \in K$ bereits~$0 \neq 1$ gilt.
    \item
      Es gilt~$64 = 2^6$, also gibt es einen Körper mit~$64$ Elementen.
    \item
      Es gilt~$72 = 8 \cdot 9 = 2^3 \cdot 3^2$, also gibt es keinen Körper mit~$72$ Elementen.
    \item
      Es gilt~$125 = 5^3$, also gibt es einen Körper mit~$125$ Elementen.
    \item
      Es gilt~$8192 = 2^{13}$, also gibt es einen Körper mit~$8192$ Elementen.
    \item
      Es gilt~$143 = 11 \cdot 13$, also gibt es keinen Körper mit~$143$ Elementen.
    \item
      Es gilt~$343 = 7^3$, also gibt es einen Körper mit~$343$ Elementen.
    \item
      Es gilt~$46 = 2 \cdot 23$, also gibt es keinen Körper mit~$46$ Elementen.
    \item
      Die gegebene Zahl ist gerade, also durch~$2$ teilbar.
      Anhand der Quersumm der gegebenen Zahl ergibt sich, dass sie auch durch~$3$ teilbar ist.
      Also handelt es sich um keine Potenz einer Primzahl, und somit auch keinen Körper mit dieser Anzahl an Elementen.
  \end{enumerate}
\end{solution}

\begin{exercise}[subtitle = {Erstklausur~18/19, Zweitklausur~19/20}]
  Entscheiden Sie jeweils, ob die folgenden Aussagen wahr oder falsch sind.
  \begin{enumerate}
    \item
      Es sei~$f \in \Rational[X]$ irreduzibel mit~$n \defined \deg(f)$.
      Dann gilt~$\card{ \Gal(f) } = n$.
    \item
      Ist~$f \in \Rational[X]$ ein Polynom vom Grad~$n$, so ist~$\card{ \Gal(f) }$ ein Teiler von~$n!$.
    \item
      Ist~$K(\alpha)/K$ eine einfache Körpererweiterung, so ist~$\Gal( K(\alpha) / K )$ zyklisch.
    \item
      Ist~$K(\alpha)/K$ eine einfache Körpererweiterung, so ist~$\Gal( K(\alpha) / K )$ abelsch.
    \item
      Jede Sylow-Untergruppe einer endlichen Gruppe ist auflösbar.
    \item
      Ist~$K$ ein endlicher Körper und~$L/K$ eine endliche Körpererweiterung, so ist diese Erweiterung eine Galois-Erweiterung.
  \end{enumerate}
\end{exercise}

\begin{solution}
  \begin{enumerate}
    \item
      Dies Aussage ist falsch:

      Es sei~$L$ ein Zerfällungskörper von~$f$.
      Dann gilt~$\Gal(f) \cong \Gal(L/\Rational)$ per Definition von~$f$, und somit insbesondere~$\card{ \Gal(f) } = \card{ \Gal(L/K) }$.
      Außerdem ist die Erweiterung~$L/\Rational$ dann eine Galois-Erweiterung:
      Sie ist endlich und normal, da~$L$ der Zerfällungskörper des Polynoms~$f \in \Rational[X]$ ist, und separabel, da~$\ringchar(\Rational) = 0$ gilt, und~$\Rational$ somit perfekt ist.
      Wir erhalten somit, dass
      \[
        [L : K]
        =
        \card{ \Gal(L/\Rational) }
        =
        \card{ \Gal(f) }
      \]
      gilt.
      Würde gegebene Aussage stimmen, so würde also~$[L : \Rational] = n$ gelten.

      Wählen wir aber das Polynom~$f = X^3 - 2 \in \Rational[X]$, so ist dies nicht der Fall:
      Das Polynom ist irreduzibel nach Eisenstein mit~$p = 2$.
      (Alternativ ist~$f$ kubisch und hat keine Nullstelle in~$\Rational$.)
      Es sei~$L$ der Zerfällungskörper von~$\Rational$ in~$\Complex$.
      Die Erweiterung~$L/\Rational$ hat den Zwischenkörper~$Z \defined \Rational(\sqrt[3]{2})$.
      Das Polynom~$f$ ist irreduzibel, normiert, und hat~$\sqrt[3]{2}$ als Nullstelle, ist also das Minimalpolynom von~$\sqrt[3]{2}$ über~$\Rational$.
      Also gilt~$[Z : \Rational] = 3$.
      Würde auch~$[L : \Rational] = 3$ gelten, so würde sich aus der Gradformel ergeben, dass~$[L : Z] = 1$, und somit~$L = Z$.
      Es wäre also~$Z = \Rational(\sqrt[3]{2})$ bereits ein Zerfällungskörper von~$f$ über~$\Rational$.
      Dies ist aber nicht der Fall, denn~$f$ zerfällt über~$Z$ nicht in Linearfaktoren, da~$f$ auch nicht-reelle Nullstellen hat, während aber~$Z$ ein Unterkörper von~$\Real$ ist.

    \item
      Die Aussage ist wahr:

      Es sei$~L$ ein Zerfällungskörper von~$f$ über~$K$.
      Dann gilt~$\Gal(f) \cong \Gal(L/K)$ per Definition von~$\Gal(f)$.
      Insbesondere gilt~$\card{ \Gal(f) } = \card{ \Gal(L/K) }$.

      Per Definitionen des Zerfällungskörpers~$L$ zerfällt das Polynom~$f$ über~$L$ in Linearfaktoren.
      Es seien~$\alpha_1, \dotsc, \alpha_n$ die nicht-notwendigerweise paarweise verschiedenen Nullstellen von~$f$ in~$L$.
      Jedes Element~$\varphi$ von~$\Gal(L/K)$ ist ein~\Homomorphismus{$K$}~$\varphi \colon L \to L$ und bildet daher die Nullstellen von~$f$ wieder auf Nullstellen von~$f$ ab.
      Zusammen mit der Injektivität von~$\varphi$ erhalten wir, dass~$\varphi$ die Nullstellen~$\alpha_1, \dotsc, \alpha_n$ permutiert.
      Wir erhalten somit einen Gruppenhomomorphismus
      \[
        \rho
        \colon
        \Gal(L/K)
        \to
        \Sym_n
      \]
      der einen~\Homomorphismus{$K$}~$\varphi \in \Gal(L/K)$ auf die eindeutige Permutation~$\sigma \in \Sym_n$ schickt, so dass
      \[
        \varphi( \alpha_i ) = \alpha_{\sigma(i)}
        \qquad
        \text{für alle~$i = 1, \dotsc, n$.}
      \]

      Per Definition von~$L$ gilt~$L = K(\alpha_1, \dotsc, \alpha_n)$.
      Deshalb ist jeder~\Homomorphismus{$K$}~$\varphi$ aus~$\Gal(L/K)$ bereits eindeutig durch seine Wirkung auf den~$\alpha_i$ bestimmt.
      Dies bedeutet für den obigen Gruppenhomomorphismus~$\rho$, dass er injektiv ist.
      Zusammen mit dem Satz von Lagrange erhalten wir somit, dass
      \[
        \card{ \Gal(L/K) }
        =
        \card{ \im(\rho) }
        \divides
        \card{ \Sym_n }
        =
        n! \,.
      \]
    \item
      Die Aussage ist falsch:

      Wir wissen aus der Vorlesung, dass es für jede endliche Gruppe~$G$ (und jede Charakteristik) eine Galois-Erweiterung~$L/K$ mit~$\Gal(L/K) \cong G$ gibt, und dass jede Galois-Erweiterung (nach dem Satz vom primitiven Element) zyklisch ist.
      Wir erhalten somit, dass es für jede endliche Gruppe~$G$ eine einfache Körpererweiterung~$K(\alpha)/K$ mit~$\Gal(K(\alpha)/K) \cong G$ gibt.
      Insbesondere können nun~$G$ als nicht-zyklisch wählen.
    \item
      Die Aussage ist falsch:

      Wir können wie im vorherigen Aufgabenteil vorgehen, indem wir~$G$ außerdem als nicht-abelsch wählen.
    \item
      Die Aussage ist wahr:

      Jede~Sylow-Untergruppe einer endlichen Gruppe ist insbesondere eine~\Gruppe{$p$}, und somit auflösbar.
    \item
      Die Aussage ist wahr:

      Die Erweiterung~$L/K$ ist per Annahme endlich, und sie ist seperabel, da der Körper~$K$ endlich und somit perfekt ist.
      Es bleibt zu zeigen, dass die Erweiterung~$L/K$ normal ist.

      Es sei~$p$ die Charakteristik von~$K$, und somit auch die Charakteristik von~$L$.
      Da~$K$ endlich ist, ist~$p$ eine Primzahl.
      Es gilt~$\card{L} = \card{K} \cdot [L : K]$, und da sowohl~$K$ als auch die Erweiterung~$L/K$ beide endlich sind, ist somit auch~$L$ endlich.
      Aus der Klassifikation endlicher Körper folgt mit~$q \defined \card{L}$ für die Erweiterung~$L/\Finite_p$, dass~$L$ der Zerfällungskörper des Polynom~$X^q - X \in \Finite_p[X]$ ist.
      Also ist die Erweiterung~$L/\Finite_p$ normal.
      Hieraus ergibt sich, dass auch die Erweiterung~$L/K$ normal ist.
  \end{enumerate}
\end{solution}

\begin{exercise}[subtitle = {Erstklausur~18/19}]
  Entscheiden Sie, für welche natürlichen Zahlen~$n$ mit~$3 \leq n \leq 34$ man das reguläre~\Eck{$n$} mit Zirkel und Lineal (aus den beiden Punkten~$0,1 \in \Complex$) konstruieren kann.
\end{exercise}

\begin{solution}
  Des regelmäßige~\Eck{$n$} ist für~$n \geq 3$ genau dann mit Zirkel und Lineal konstruierbar, falls die Zahl~$n$ von der Form~$n = 2^m p_1 \dotsm p_r$ ist, wobei~$n \in \Natural$ und~$p_1, \dotsc, p_r$ paarweise verschiedene Fermat-Primzahlen sind.
  Die einzigen Fermat-Primzahlen kleiner-gleich als~$34$ sind~$3$,~$5$,~$17$.
  Wir erhalten somit für~$n$ die möglichen Werte
  \begin{align*}
    3 &= 3 \,,
    \\
    2 \cdot 3 &= 6 \,,
    \\
    4 \cdot 3 &= 12 \,,
    \\
    8 \cdot 3 &= 24 \,,
    \\
    5 &= 5 \,,
    \\
    2 \cdot 5 &= 10 \,,
    \\
    4 \cdot 5 &= 20 \,,
    \\
    17 &= 17 \,,
    \\
    2 \cdot 17 &= 34 \,,
    \\
    3 \cdot 5 &= 15 \,,
    \\
    2 \cdot 3 \cdot 5 &= 30 \,,
    \\
    3 \cdot 7 &= 21 \,.
  \end{align*}
\end{solution}

\begin{exercise}[subtitle = {Erstklausur~18/19}]
  Es sei~$f \in \Rational[X]$ ein irreduzibles Polynom vom Grad~$4$, so dass~$\Gal(f) \cong \Sym_4$ gilt.
  Ferner sei~$L$ ein Zerfällungskörper von~$f$ über~$\Rational$.
  Zeigen Sie, dass es einen Zwischenkörper~$Z$ der Erweiterung~$L / \Rational$ gibt, so dass~$[Z : \Rational] = 3$ gilt.
\end{exercise}

\begin{solution}
  Die Körpererweiterung~$L/\Rational$ ist eine Galois-Erweiterung:
  Diese Erweiterung ist separabel, denn es gilt~$\ringchar(\Rational) = 0$, weshalb~$\Rational$ separabel ist.
  Sie ist endlich, da sie der Zerfällungskörper des Polynoms~$f$ ist.
  Sie ist normal, da~$L$ per Definition der Zerfällungskörper eines Polynoms aus~$\Rational[X]$ ist.

  Es sei~$G \defined \Gal(L / \Rational)$.
  Da~$L$ ein Zerfällungskörper von~$f$ ist, gilt~$G \cong \Gal(f) \cong \Sym_4$.
  Da die Erweiterung~$L/\Rational$ eine Galois-Erweiterung ist, gilt somit
  \[
    [L : \Rational]
    =
    \card{ G }
    =
    \card{ \Sym_4 }
    =
    24 \,.
  \]
  Wir nutzen nun die Galois-Korrespondenz:
  Die Existenz eines Zwischenkörpers~$Z$ der Erweiterung~$L / \Rational$ vom Grade~$[Z : \Rational] = 3$ ist äquivalent zur Existenz einer Untergruppe~$H$ von~$G$ mit Index~$[G : H] = 3$.
  Dass~$[G : H] = 3$ gilt ist äquivalent zu~$\card{H} = 8$, da~$\card{G} = 24$ gilt.
  Es wird also die Existenz einer~\Sylow{$2$}-Untergruppe von~$G$ benötigt.
  Diese Existenz folgt aus dem ersten Sylowsatz.
\end{solution}

\begin{exercise}
  Zeigen Sie, dass jede~\Gruppe{$p$}~$G$ auflösbar ist.
  (\emph{Tipp:} Nutzen Sie, dass~$\zenter(G) \neq 1$, falls~$G \neq 1$.)
\end{exercise}

\begin{solution}
  Wir zeigen die Aussage per Induktion über die Ordnung von~$G$.
  Gilt~$\card{G} = 1$, so ist die Gruppe~$G$ trivial, und somit auflösbar.
  Dies liefert den Induktionsanfang.

  Für den Induktionsschritt sei nun~$\card{G} > 1$.
  Wir betrachten das Zentrum~$\zenter(G)$.
  Aus der Nicht-Trivialität von~$G$ folgt, dass auch~$\zenter(G)$ nicht-trivial ist, da~$G$ eine~\Gruppe{$p$} ist.
  Da~$\zenter(G)$ ein Normalteiler in~$G$ ist, ist die Gruppe~$G$ genau dann auflösbar, wenn~$G/{\zenter(G)}$ und~$\zenter(G)$ beide auflösbar sind.

  Das Zentrum~$\zenter(G)$ ist auflösbar, da es abelsch ist.
  Die Faktorgruppe~$G/{\zenter(G)}$ ist eine~\Gruppe{$p$} von echt kleinerer Kardinalität als~$G$, und somit nach Induktion auflösbar.
  Wir erhalten somit die Auflösbarkeit von~$G$.
\end{solution}

\begin{exercise}[subtitle = {Erstklausur~19/20}]
  Es sei~$f \defined X^5 - 14 X^3 - 21 X^2 + 49 X + 28 \in \Rational[X]$.
  Es sei~$\alpha \in \Complex$ eine Nullstelle von~$f$.
  Zeigen Sie, dass~$\alpha$ nicht durch Zirkel und Lineal konstruierbar ist.
\end{exercise}

\begin{solution}
  Wäre~$\alpha$ durch Zirkel und Lineal konstruierbar, so wäre der Grad~$[\Rational(\alpha) : \Rational]$ eine Zweierpotenz.
  Diesen Grad können wir mithilfe des Minimalpolynoms von~$\alpha$ bestimmen:
  Das Polynom ist irreduzibel nach des Eisenstein-Kriterium mit~$p = 7$.
  Es ist außerdem normiert und hat~$\alpha$ als Nullstelle.
  Also ist~$f$ das Minimalpolynom von~$\alpha$.
  Es gilt deshalb
  \[
    [\Rational(\alpha) : \Rational]
    =
    \deg(f)
    =
    5 \,.
  \]
  Dies ist keine Zweierpotenz.
\end{solution}

\begin{exercise}
  Es sei~$n \in \Natural_1$ und~$K$ ein Körper mit~$\ringchar(K) \ndivides n$.
  Zeigen Sie, dass~$\card{ \Ein_n } = n$ gilt.
\end{exercise}

\begin{solution}
  Die Menge~$\Ein_n$ ist genau die Nullstellenmenge des Polynom~$f \defined X^n - 1 \in \Rational[X]$.
  Für die Ableitung~$f' \defined n X^{n-1}$ gilt~$f' \neq 0$, da nach Annahme~$\ringchar(K) \ndivides n$ gilt.
  Es ist~$0$ die einzige Nullstelle von~$f'$, und~$0$ ist keine~\te{$n$} Einheitswurzel.
  Wir sehen somit, dass die Polynome~$f$ und~$f'$ keine gemeinsamen Nullstellen haben.
  Dies bedeutet, dass~$f$ keine mehrfache Nullstelle hat.
  Also gilt
  \[
    \card{ \Ein_n }
    =
    \deg(f)
    =
    n \,.
  \]
\end{solution}

\begin{exercise}
  Für jedes natürliche Zahl~$n \in \Natural_1$ sei~$\zeta_n \defined e^{2 \pi i / n} \in \Complex$.
  Bestimmen Sie den Grad der Körpererweiterung~$\Rational(\zeta_{84}) / \Rational(\zeta_{21})$.
\end{exercise}

\begin{solution}
  Für alle~$n \in \Natural_1$ gilt
  \[
    [\Rational(\zeta_n) : \Rational]
    =
    \varphi(n) \,,
  \]
  wobei~$\varphi$ die Eulersche~\Funktion{$\varphi$} bezeichnet.
  (Das Minimalpolynom von~$\zeta_n$ über~$\Rational$ ist das~\te{$n$} Kreisteilungspolynom~$\Phi_n$, denn~$\Phi_n$ ist normiert und irreduzibel, und hat~$\zeta_n$ als Nullstelle.)
  Wir erhalten für~$n = 21$, dass
  \[
    \varphi(21)
    =
    \varphi(3 \cdot 7)
    =
    \varphi(3) \cdot \varphi(7)
    =
    2 \cdot 6 
    =
    12 \,,
  \]
  und für~$n = 84$, dass
  \[
    \varphi(84)
    =
    \varphi(2^2 \cdot 3 \cdot 7)
    =
    \varphi(2^2) \cdot \varphi(3) \cdot \varphi(7)
    =
    1 \cdot 2 \cdot 2 \cdot 6
    =
    24 \,.
  \]
  Wir erhalten nun mithilfe der Gradformel, dass
  \[
    [ \Rational(\zeta_{84}) : \Rational(\zeta_{21}) ]
    =
    \frac{ [\Rational(\zeta_{84}) : \Rational] }{ [\Rational(\zeta_{21}) : \Rational] }
    =
    \frac{24}{12}
    =
    2 \,.
  \]
\end{solution}




\clearpage





\printsolutions





\end{document}
