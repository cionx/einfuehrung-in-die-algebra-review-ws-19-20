\chapter{Gruppentheorie}

\begin{convention}
  Im Folgenden ist~$G$, sofern nicht anders angegeben, eine Gruppe.
\end{convention}

\begin{definition}
  Eine Teilmenge~$H$ von~$G$ ist eine \defemph{Untergruppe} falls die folgenden drei Bedingungen erfüllt sind.
  \begin{enumerate}
    \item
      Es gilt~$1 \in H$.
    \item
      Für je zwei Elemente~$h_1, h_2 \in H$ gilt auch~$h_1 h_2 \in H$.
    \item
      Für jedes Element~$h \in H$ gilt auch~$h^{-1} \in H$.
  \end{enumerate}
  Dass~$H$ eine Untergruppe von~$G$ ist, wird mit~$H \subgroupeq G$ notiert.
\end{definition}

\begin{convention}
  Im Folgenden ist~$H$, sofern nicht anders angegeben, eine Untergruppe von~$G$.
\end{convention}





\section{Ordnungen und Indizes}

\begin{definition}
  Die \defemph{Ordnung} von~$G$ ist~$\card{G} \in \Natural_{1} \cup \{ \infty \}$.
\end{definition}



\subsection{Nebenklassen}

\begin{definition}
  Es sei~$g$ ein Element von~$G$.
  \begin{enumerate}
    \item
      Die \defemph{Linksnebenklasse} von~$g$ bzgl.~$H$ ist die Teilmenge~$gH$ von~$G$.
    \item
      Die \defemph{Linksnebenklasse} von~$g$ bzgl.~$H$ ist die Teilmenge~$gH$ von~$G$.
    \item
      Die Menge der Links- bzw. Rechtsnebenklassen in~$G$ bzgl.~$H$ sind
      \[
        G/H \defined \{ gH \suchthat g \in G \}
        \quad\text{und}\quad
        H \backslash G \defined \{ Hg \suchthat g \in G \} \,.
      \]
  \end{enumerate}
\end{definition}

\begin{proposition}
  Die Abbildung
  \[
    G/H
    \to
    H \backslash G \,,
    \quad
    gH
    \mapsto
    (gH)^{-1}
    =
    Hg^{1}
  \]
  ist eine wohldefinierte Bijektion.
\end{proposition}

\begin{corollary}
  Die beiden Mengen~$G/H$ und~$H \backslash G$ sind gleichmächtig.
\end{corollary}

\begin{definition}
  Die Kardinalität der Menge~$G/H$, bzw.~$H \backslash G$ ist der \defemph{Index} von~$H$ in~$G$.
  Er wird mit~$[G : H]$ bezeichnet, d.h.
  \[
    [G : H]
    \defined
    \card{ G/H }
    =
    \card{ H \backslash G } \,.
  \]
\end{definition}

\begin{lemma}
  \leavevmode
  \begin{enumerate}
    \item
      Jedes Element~$g \in G$ ist in seiner jeweiligen Nebenklasse~$gH$ enthalten.
    \item
      Für je zwei Element~$g_1, g_2 \in G$ sind die Nebenklassen~$g_1 H$ und~$g_2 H$ entweder gleich oder disjunkt.
      Dabei gilt genau dann~$g_1 H = g_2 H$ wenn~$g_2^{-1} g_1 \in H$, genau dann wenn~$g_1^{-1} g_2 \in H$.
    \item
      Für jedes ELement~$g \in G$ die Abbildung
      \[
        H \to gH \,,
        \quad
        h \mapsto gh
      \]
      bijektiv.
      Insbesondere gilt~$\card{gH} = \card{H}$.
  \end{enumerate}
\end{lemma}

\begin{corollary}
  Die Gruppe~$G$ ist die disjunkte Vereinigung der Linksnebenklassen bzgl.~$H$, und es gilt
  \[
    \card{G} = \card{H} \cdot [G : H] \,.
  \]
\end{corollary}

\begin{theorem}[Lagrange]
  Ist~$G$ endlich, so sind~$\card{H}$ und~$[G : H]$ jeweils Teiler von~$\card{G}$.
\end{theorem}



\subsection{Erzeugte Untergruppen}

\begin{convention}
  Im Folgenden ist~$S$, sofern nicht anders angegeben, eine Teilmenge von~$G$.
\end{convention}

\begin{definition}
  \leavevmode
  \begin{enumerate}
    \item
      Die von~$S$ \defemph{erzeugte Untergruppe} von~$G$ ist die Menge
      \[
        \gen{S}
        \defined
        \bigl\{
          s_1^{\varepsilon_1} \dotsm s_n^{\varepsilon_n}
        \suchthat[\big]
          n \in \Natural,
          s_1, \dotsc, s_n \in S,
          \varepsilon_1, \dotsc, \varepsilon_n \in \{1, -1\}
        \bigr\}
      \]
    \item
      Die Menge~$S$ ist ein \defemph{Erzeugendensystem} von~$G$ falls~$G = \gen{S}$.
  \end{enumerate}
\end{definition}

\begin{proposition}
  \leavevmode
  \begin{enumerate}
    \item
      Die von~$S$ erzeugte Untegruppe~$\gen{S}$ ist eine Untergruppe von~$G$.
    \item
      Es ist~$\gen{S}$ die kleinste Untergrupe von~$G$, die~$S$ enthält:
      Es gilt~$S \subseteq \gen{S}$, und für jede Untergruppe~$H$ von~$G$ mit~$S \subseteq H$ gilt auch~$\gen{S} \subseteq H$.
    \item
      Es gilt~$\gen{S} = \bigcap \{ H \suchthat H \subgroupeq G, S \subseteq H \}$.
  \end{enumerate}
\end{proposition}



\subsection{Zyklische Gruppen}

\begin{definition}
  Die Gruppe~$G$ ist \defemph{zyklisch} wenn sie von einem einzelnen Element erzeugt wird, d.h. es gibt ein Element~$g \in G$ mit~$G = \gen{g}$.
\end{definition}

\begin{theorem}[Klassifikation zyklischer Gruppen]
  Die Gruppe~$G$ sei zyklisch.
  Es gilt
  \[
    G
    \cong
    \begin{cases*}
        \Integer
        &
        falls~$\card{G} = \infty$,
        \\
        \Integer_n
        &
        falls~$\card{G} = n < \infty$.
    \end{cases*}
  \]
\end{theorem}

\begin{proposition}
  Die Gruppe~$G$ sei endlich und~$p \defined \card{G}$
  Dann ist~$G$ zyklisch, und jedes Element~$g \in G$ mit~$g \neq e$ ist ein Erzeuger für~$G$.
\end{proposition}

\begin{corollary}
  Für jede Primzahl~$p$ ist~$\Integer_p$ die bis auf Isomorphie eindeutige Gruppe von Ordnung~$p$.
\end{corollary}



\subsection{Die Ordnung eines Elements}

\begin{definition}
  Es sei~$g \in G$.
  Die \defemph{Ordnung} von~$g$ ist
  \[
    \ord(g)
    \defined
    \inf
    {}
    \{
      n \in \Natural
    \suchthat
      n \geq 1,
      g^n = 1
    \}
    \in
    \Natural_1 \cup \{ \infty \} \,.
  \]
\end{definition}

\begin{proposition}
  Für jedes Gruppenelement~$g \in G$ gilt~$\ord(g) = \card{\gen{g}}$.
\end{proposition}

\begin{corollary}
  Ist~$G$ endlich, so gilt~$\ord(g) \divides \card{G}$ für jedes~$g \in G$;
  insbesondere ist~$\ord(g)$ endlich.
\end{corollary}

\begin{corollary}
  Ist~$G$ endlich, so gilt für~$n \defined \card{G}$, dass~$g^n = 1$ für jedes~$g \in G$.
\end{corollary}





\section{Gruppenaktionen}



\subsection{Grundlegendes}

\begin{definition}
  Es sei~$G$ eine Gruppe.
  \begin{enumerate}
    \item
      Eine \defemph{Aktion} von~$G$ auf einer Menge~$X$  ist eine Abbildung
      \[
        G \times X \to X \,,
        \quad
        (g,x)
        \mapsto
        g.x
      \]
      welche die folgenden beiden Eigenschaften erfüllt:
      \begin{enumerate}
        \item
          Es gilt~$1.x = x$ für jedes Element~$x \in X$.
        \item
          Es gilt~$g.(h.x) = (gh).x$ für alle Gruppenelemente~$g, h \in G$ und jedes Element~$x \in X$. 
      \end{enumerate}
    \item
      Eine~\defemph{\Menge{$G$}} ist eine Menge~$X$ zusammen mit einer Aktion von~$G$ auf~$X$.
  \end{enumerate}
\end{definition}

\begin{convention}
  Im Folgenden ist~$X$, sofern nicht anders angegeben, eine~\Menge{$G$}.
\end{convention}

\begin{definition}
  Es sei~$Y$ eine weitere~\Menge{$G$}.
  Eine Abbildung~$\varphi \colon X \to Y$ ist ein~\defemph{\Homomorphismus{$G$}} falls
  \[
    \varphi(g.x) = g.\varphi(x)
    \qquad
    \text{für alle~$g \in G$,~$x \in X$.}
  \]
\end{definition}



\subsection{Bahnen und Stabilisatoren}

\begin{definition}
  Es sei~$x \in X$.
  \begin{enumerate}
    \item
      Die \defemph{Bahn} von~$x$ ist die Teilmenge~$G.x$ von~$X$ gegeben durch~$G.x \defined \{ g.x \suchthat g \in G \}$.
    \item
      Der \defemph{Stabilisator} von~$x$ ist die Teilmenge~$G_x$ von~$G$ gegeben durch~$G_x \defined \{ g \in G \suchthat g.x = x \}$.
  \end{enumerate}
\end{definition}

\begin{proposition}
  Für jedes Element~$x \in X$ ist der Stabilisator~$G_x$ eine Untergruppe von~$G$.
\end{proposition}

% TODO: Exercise: Konjugierte Stabilisatoren.

%Ist~$X$ eine~\Menge{$G$}, so ist für jedes Gruppenelement~$g \in G$ die Abbildung
%\[
%  \lambda_g
%  \colon
%  X \to X \,,
%  \quad
%  x \mapsto g.x
%\]
%eine Bijektion (mit~$\lambda_g^{-1} = \lambda_{g^{-1}}$).
%Die entstehende Abbildung
%\[
%  \lambda
%  \colon
%  G \to \Sym(X) \,,
%  \quad
%  g \mapsto \lambda_g
%\]
%ist ein Gruppenhomomorphismus.
%Ist umgekehrt~$\phi \colon G \to \Sym(X)$ ein Gruppenhomomorphismus, so wird durch
%\[
%  g.x \defined \phi(g)(x)
%  \qquad
%  \text{für alle~$g \in G$,~$x \in X$}
%\]
%eine Aktion von~$G$ auf~$X$ definiert.
%Diese beiden Konstruktionen sind inverse zueinander, und wir erhalten das folgende Resultat.
%
%\begin{proposition}
%  Es sei~$G$ eine Gruppe und~$X$ eine Menge.
%  Dann gibt es eine Eins-zu-eins-Korrespondenz gegeben durch
%  \begin{align*}
%    \{
%      \text{Aktionen von~$G$ auf~$X$}
%    \}
%    &\longonetoone
%    \left\{
%      \begin{tabular}{@{}c@{}}
%        Gruppenhomomorphismen\\
%        $G \to \Sym(X)$
%      \end{tabular}
%    \right\} \,,
%    \\
%    *
%    &\longmapsto
%    [g \mapsto [x \mapsto g*x]]
%    \\
%    [(g,x) \mapsto \phi(g)(x)]
%    &\longmapsfrom
%    \phi \,.
%  \end{align*}
%\end{proposition}

Für je zwei Gruppenelement~$g_1, g_2 \in G$ und ein Element~$x \in X$ gilt
\[
  g_1.x = g_2.x
  \iff
  g_2^{-1} g_1.x = x
  \iff
  g_2^{-1} g_1 \in G_x
  \iff
  g_1 G_x = g_2 G_x \,.
\]
Hieraus erhalten wir das folgende Resultat.

\begin{corollary}
  \label{isomorphism theorem for G-sets}
  Für jedes Element~$x \in X$ ist die Abbildung
  \[
    G/G_x \to G.x \,,
    \quad
    g G_x \mapsto g.x
  \]
  ein wohldefinierter, bijektiver~\Homomorphismus{$G$}.
\end{corollary}

\begin{corollary}
  Für jedes Element~$x \in X$ gilt~$\card{G.x} = [G : G_x]$.
\end{corollary}

\begin{corollary}
  Ist~$G$ endlich, so ist für jedes Element~$x \in X$ die Kardinalität~$\card{G.x}$ ein Teiler der Gruppenordnung~$\card{G}$.
\end{corollary}

% TODO: Beispiele.



\subsection{Transitivität}

\begin{proposition}
  \label{characterizations of transitive actions}
  Es sei~$X \neq \emptyset$.
  Die folgenden Bedingungen für~$X$ sind äquivalent.
  \begin{enumerate}
    \item
      Es gibt für alle Elemente~$x_1, x_2 \in X$ ein Gruppenelement~$g \in G$ mit~$g.x_1 = x_2$.
    \item
      Für jedes Element~$x \in X$ gilt~$G.x = X$.
    \item
      Es gibt ein Element~$x \in X$ mit~$G.x = X$.
    \item
      Die Menge~$X$ besteht aus genau einer~\Bahn{$G$}.
  \end{enumerate}
\end{proposition}

\begin{definition}
  Eine Gruppenaktion von~$G$ auf~$X$ ist \defemph{transitiv}, falls sie die äquivalenten Bedingungen aus \cref{characterizations of transitive actions} erfüllt.
\end{definition}

\begin{warning}
  Ob die eindeutige Gruppenaktion von~$G$ auf der leeren Menge transitiv ist, ist Geschmackssache.
  In dieser Vorlesung ist diese Wirkung \emph{nicht} transitiv.
\end{warning}



\subsection{Fixpunkte}

\begin{definition}
  Ein Element~$x \in X$ ist ein \defemph{Fixpunkt}, falls~$g.x = x$ für jedes Gruppenelement~$g \in G$.
  Die Menge der Fixpunkte wird mit~$X^G$ bezeichnet.
\end{definition}

\begin{proposition}
  Für ein Element~$x \in X$ sind die folgenden Bedingungen äquivalent.
  \begin{equivlist}
    \item
      Das Element~$x$ ist ein Fixpunkt.
    \item
      Die Bahn~$G.x$ ist gegeben durch~$G.x = \{ x \}$.
    \item
      Der Stabilisator~$G_x$ ist gegeben durch~$G_x = G$.
  \end{equivlist}
  Insbesondere sind Fixpunkte durch ihre Bahnen, und auch durch ihre Stabilisatoren charakterisiert.
\end{proposition}



\subsection{Die Bahnenformel}

\begin{proposition}
  Die~\Menge{$G$}~$X$ ist die disjunkte Vereinigung ihrer~\Bahnen{$G$}.
\end{proposition}

\begin{corollary}
  Die~\Menge{$G$}~$X$ sei endlich, und es seien~$x_1, \dotsc, x_n$ ein Repräsentatensystem der~\Bahnen{$G$} von~$X$.
  Es gilt
  \[
    \card{X}
    =
    \sum_{i=1}^n \card{G.x_i}
    =
    \card*{X^G}
    +
    \sum_{\substack{i=1, \dotsc, n \\ x_i \notin X^G}}
    \card{G.x_i}
  \]
\end{corollary}

\begin{theorem}[Bahnenformel]
  Die~\Menge{$G$}~$X$ sei endlich, und es seien~$x_1, \dotsc, x_n$ ein Repräsentatensystem der~\Bahnen{$G$} von~$X$.
  Es gilt
  \[
    \card{X}
    =
    \sum_{i=1}^n [G : G_{x_i}]
    =
    \card*{X^G}
    +
    \sum_{\substack{i=1, \dotsc, n \\ x_i \notin X^G}}
    [G : G_{x_i}] \,.
  \]
\end{theorem}



\section{Konjugation}

\begin{definition}
  Es sei~$g \in G$.
  \begin{enumerate}
    \item
      Die Abbildung~$G \to G$,~$x \mapsto g x g^{-1}$ ist die \defemph{Konjugation} mit~$g$.
    \item
      Die \defemph{Konjugationsklasse} eines Gruppenelements~$x \in G$ ist die Menge~$x^G \defined \{ g x g^{-1} \suchthat g \in G \}$.
    \item
      Die \defemph{Konjugationsklasse} einer Untergruppe~$H$ von~$G$ ist die Menge~$\{ g H g^{-1} \suchthat g \in G \}$.
  \end{enumerate}
\end{definition}

\begin{proposition}
  Für jedes Grupenelement~$g \in G$ ist die Konjugationsabbildung
  \[
    c_g
    \colon
    G \to G \,
    \quad
    x \mapsto g x g^{-1}
  \]
  ein Gruppenautomorphismus.
  Es gilt~$c_g^{-1} = c_{g^{-1}}$.
\end{proposition}

\begin{proposition}
  \leavevmode
  \begin{enumerate}
    \item
      Die Gruppe~$G$ ist die disjunkte Vereinigung ihrer Konjugationsklassen.
    \item
      Die Menge~$\{ \text{Untergruppen von~$G$} \}$ ist die disjunkte Vereinigung der Konjugationsklassen.
  \end{enumerate}
\end{proposition}

\begin{definition}
  Zwei Gruppenelemente von~$G$, bzw.\ Untegruppen von~$G$, sind \defemph{konjugiert} zueinander, falls sie in der gleichen Konjugationsklasse liegen.
  (Mit anderen Worten, zwei Gruppenelemente~$x_1, x_2 \in G$, bzw. zwei Untergruppen~$H_1, H_2 \subgroupeq G$ sind konjugiert zueinander, falls es ein Element~$g \in G$ mit~$x_2 = g x_1 g^{-1}$, bzw.\ mit~$H_2 = g H_1 g^{-1}$ gibt.)
\end{definition}



\subsection{Zentrum und Zentralisatioren}

\begin{definition}
  \leavevmode
  \begin{enumerate}
    \item
      Das \defemph{Zentrum} von~$G$ ist die Menge~$\zenter(G) \defined \{ g \in G \suchthat \text{$g h = h g$ für alle~$h \in G$} \}$.
    \item
      Der \defemph{Zentralisator} eines Gruppenenelements~$g \in G$ ist die Menge~$\zentralizer_G(g) \defined \{ h \in G \suchthat gh = hg \}$.
  \end{enumerate}
\end{definition}

\begin{proposition}
  \leavevmode
  \begin{enumerate}
    \item
      Das Zentrum~$\zenter(G)$ ist eine Untergruppe von~$G$.
    \item
      Für jedes Gruppenelement~$g \in G$ ist sein Zentralisator~$\zentralizer_G(g)$ eine Untergruppe von~$G$.
    \item
      Es gilt~$\zenter(G) = \bigcap_{g \in G} \zentralizer_G(g)$.
  \end{enumerate}
\end{proposition}



\subsection{Anwendung der Bahnenformel}

Die Gruppe~$G$ operiert auf sich selbst durch Konjugation, d.h. vermöge
\[
  g.x = g x g^{-1}
  \qquad
  \text{für alle~$g \in G$,~$x \in G$.}
\]
Die Bahnen dieser Gruppenaktion sind genau die Konjugationsklassen von~$G$.
Der Stabilisator eines Elements~$x \in G$ ist genau der Zentralisator~$\zentralizer_G(x)$.
Die Fixpunkte sind genau die Element des Zentrums.
Aus der Bahnenformel ergibt sich somit das folgende Resultat.

\begin{theorem}
  Die Gruppe~$G$ sei endlich, und seien~$x_1, \dotsc, x_n$ ein Repräsentantensystem der Konjugationsklassen von~$G$.
  Es gilt
  \[
    \card{G}
    =
    \sum_{i=1}^n [G : \zentralizer_G(x_i)]
    =
    \card{ \zenter(G) }
    +
    \sum_{\substack{i = 1, \dotsc, n \\ x_i \notin \zenter(G)}}
    [G : \zentralizer_G(x_i)] \,.
  \]
\end{theorem}




\section{Normalteiler}

\begin{proposition}
  \label{characterizations of normal subgroups}
  Für eine Untergruppe~$N$ von~$G$ sind die folgenden Bedingungen äquivalent:
  \begin{equivlist}
    \item
      Es gilt~$g N g^{-1} = N$ für jedes Gruppenelement~$g \in G$.
    \item
      Es gilt~$g N g^{-1} \subseteq N$ für jedes Gruppenelement~$g \in G$.
    \item
      Es gilt~$gN = Ng$ für jedes Gruppenelement~$g \in G$.
    \item
      Für jedes Gruppenelement~$g_1 \in G$ gibt es ein Gruppenelement~$g_2 \in G$ mit~$g_1 N = N g_2$.
    \item
      Für jedes Gruppenelement~$g_1 \in G$ gibt es ein Gruppenelement~$g_2 \in G$ mit~$N g_1 = g_2 N$.
  \end{equivlist}
\end{proposition}

\begin{definition}
  Es sei~$H$ eine Untergruppe von~$G$.
  Die Untergruppe~$N$ ist \defemph{normal}, bzw. ein \defemph{Normalteiler}, falls sie die äquivalenten Bedingungen aus \cref{characterizations of normal subgroups} erfüllt.
  Dass~$N$ ein Normalteiler ist, wird mit~$N \normgroupeq G$ notiert.
\end{definition}

\begin{example}[Kerne]
  Ist~$\varphi \colon G \to G'$ ein Gruppenhomomorphsimus, so ist
  \[
    \ker(\varphi)
    \defined
    \{
      g \in G
    \suchthat
      \varphi(g) = 1
    \}
  \]
  eine normale Untergruppe von~$G$.
  Dabei ist der Homomorphismus~$\varphi$ genau dann injektiv, wenn~$\ker(\varphi) = 1$ gilt.
\end{example}

\begin{convention}
  Im Folgenden ist~$N$, sofern nicht anders angegeben, eine normale Untergruppe von~$G$.
\end{convention}




\subsection{Normalisatoren}

\begin{definition}
  Der \defemph{Normalisator} von~$H$ (in~$G$) ist die Menge~$\normalizer_G(H) \defined \{ g \in G \suchthat g H g^{-1} = H \}$.
\end{definition}

\begin{proposition}
  Es seien~$H$,~$K$ zwei Untergruppen von~$G$.
  Es ist~$H$ genau dann einen normale Untergruppe von~$K$, wenn~$H \subgroupeq K \subgroupeq \normalizer_G(H)$.
\end{proposition}

\begin{corollary}
  
\end{corollary}

Es sei~$X$ die Menge der zu $H$ konjugierten Untergruppen.
Die Gruppe~$G$ operiert auf der Menge~$X$ vermöge
\[
  g.H' \defined g H g^{-1} \,.
\]
Es gilt~$H \in X$, und die Bahn von~$H$ ist ganz~$X$.
Der Stabilisator von~$H$ ist genau der Normalisator~$\normalizer_G(H)$.
Wir erhalten somit aus \cref{isomorphism theorem for G-sets} das folgende Resultat.

\begin{corollary}
  Die Abbildung
  \[
    G / {\normalizer_G(H)}
    \to
    \{
      \text{zu~$H$ konjugierte Untergruppen von~$G$}
    \} \,,
    \quad
    g {\normalizer_G(H)}
    \mapsto
    g H g^{-1}
  \]
  ist eine wohldefinierte Bijektion.
\end{corollary}

\begin{corollary}
  Ist~$G$ endlich, so gilt die Anzahl der zu~$H$ konjugierten Untergruppen von~$G$ durch den Index~$[G : \normalizer_G(H)]$ gegeben.
\end{corollary}





\section{Faktorgruppen}



\subsection{Konstruktion}

\begin{proposition}
  \label{construction of quotient group}
  Für jedes Gruppenelement~$g \in G$ sei~$\class{g} \defined gN$.
  \begin{enumerate}
    \item
      Die Multiplikation~$\class{g_1} \cdot \class{g_2} \defined \class{g_1 g_2}$ ist eine wohldefinierte Gruppenstruktur auf~$G/N$.
    \item
      Dies ist die eindeutige Gruppenstruktur auf~$G/N$, für welche die Abbildung
      \[
        \rho
        \colon
        G \to G/N \,,
        g \mapsto \class{g}
      \] ein Gruppenhomomorphismus ist.
  \end{enumerate}
\end{proposition}

\begin{definition}
  In der Situation von \cref{construction of quotient group} ist~$G/N$ die \defemph{Faktorgruppe} von~$G$ nach~$N$, und die Abbildung~$\rho$ ist die \defemph{kanonische Projektion}.
\end{definition}



\subsection{Korrespondenz von Untergruppen}

\begin{theorem}
  Es sei~$\rho \colon G \to G/N$ die kanonische Projektion.
  \begin{enumerate}
    \item
      Es gibt eine Eins-zu-eins-Korrespondenz von Untergruppen, gegeben durch
      \begin{align*}
        \left\{
          \begin{tabular}{@{}c@{}}
            Untergruppen von~$G$,\\
            die~$N$ enthalten
          \end{tabular}
        \right\}
        &\onetoone
        \{ \text{Untergruppen von~$G/N$} \}
        \\
        H
        &\mapsto
        H/N \,,
        \\
        \rho^{-1}(H')
        &\mapsfrom
        H' \,.
      \end{align*}
    \item
      In der obigen Korrespondenz ist eine Untergruppe~$H$ von~$G$ genau dann ein Normalteiler, wenn~$H/N$ ein Normalteiler von~$G/N$ ist.
      Durch Einschränkung der obigen Korrespondenz ergibt sich daher die Eins-zu-eins-Korrespondenz
      \[
        \left\{
          \begin{tabular}{@{}c@{}}
            normale Untergruppen von~$G$,\\
            die~$N$ enthalten
          \end{tabular}
        \right\}
        \onetoone
        \{ \text{normale Untergruppen von~$G/N$} \} \,.
      \]
  \end{enumerate}
\end{theorem}



\subsection{Homomorphiesatz und Isomorphiesätze}

\begin{theorem}[Homomorphiesatz]
  Es sei~$\rho \colon G \to G/N$ die kanonische Projektion, und es sei~$H$ eine weitere Gruppe.
  \begin{enumerate}
    \item
      Es gilt~$\ker(\rho) = N$.
    \item
      Ist~$\psi \colon G/N \to H$ ein Gruppenhomomorphismus, so gilt für den Gruppenhomomorphismus~$\psi \circ \rho \colon G \to H$, dass~$N \subseteq \ker(\psi \circ \rho)$.
    \item
      Ist andererseits~$\varphi \colon G \to H$ ein Gruppenhomomorphismus mit~$N \subseteq \ker(\varphi)$, so ist
      \[
        \induced{\varphi}
        \colon
        G/N \to H \,,
        \quad
        \class{g} \mapsto \varphi(g)
      \]
      ein wohldefinierter Gruppenhomorphisms.
      Dies ist der eindeutige Gruppenhomomorphismus, der das folgende Diagramm zum kommutieren bringt:
      \[
        \begin{tikzcd}
          G
          \arrow{r}[above]{\varphi}
          \arrow{d}[left]{\rho}
          &
          H
          \\
          G/N
          \arrow[dashed]{ur}[below right]{\induced{\varphi}}
          &
          {}
        \end{tikzcd}
      \]
      Dabei gilt~$\im(\induced{\varphi}) = \im(\varphi)$ sowie~$\ker(\induced{\varphi}) = \ker(\varphi)/N$.
    \item
      Die beiden obigen Konstruktionen sind invers zueinander, und liefern somit eine Eins-zu-eins-Korrespondenz gegeben durch
      \begin{align*}
        \left\{
          \begin{tabular}{@{}c@{}}
            Gruppenhomomorphismen \\
            $\varphi \colon G \to H$ mit~$N \subseteq \ker(\varphi)$
          \end{tabular}
        \right\}
        &\onetoone
        \left\{
          \begin{tabular}{@{}c@{}}
            Gruppenhomomorphismen \\
            $\psi \colon G/N \to H$
          \end{tabular}
        \right\}
        \\
        \varphi
        &\mapsto
        \induced{\varphi} \,,
        \\
        \psi \circ \rho
        &\mapsfrom
        \psi \,.
      \end{align*}
  \end{enumerate}
\end{theorem}

\begin{corollary}
  Es sei~$H$ eine weiter Gruppe und~$\varphi \colon G \to H$ ein Gruppenhomomorphismus.
  Dann induziert~$\varphi$ einen Gruppenisomorphismus
  \[
    \induced{\varphi}
    \colon
    G/\ker(\varphi) \to \im(\varphi) \,,
    \quad
    \class{g}
    \mapsto
    \varphi(g) \,.
  \]
  Der Homomorphismus~$\varphi$ lässt sich also wie folgt in einen Epimorphismus, Isomorphismus und Monomorphismus zerlegen.
  \[
    \begin{tikzcd}
      G
      \arrow{r}[above]{\varphi}
      \arrow[two heads]{d}[left]{\rho}
      &
      H
      \\
      G/\ker(\varphi)
      \arrow{r}[above]{\induced{\rho}}
      &
      \im(\varphi)
      \arrow[hook]{u}
    \end{tikzcd}
  \]
\end{corollary}

\begin{corollary}[Noethersche Isomorphiesätze]
  \leavevmode
  \begin{enumerate}
    \item
      Es seien~$H$,~$N$ Untergruppen von~$G$, wobei~$N$ ein Normalteiler in~$G$ ist.
      \begin{enumerate}
        \item
          Das Produkt~$HN = \{ hn \suchthat h \in H, n \in N \}$ ist eine Untergruppe von~$G$.
        \item
          Es ist~$N$ ein Normalteiler in~$HN$ und~$H \cap N$ ein Normalteiler in~$H$.
        \item
          Die Inklusion~$H \to HN$ induziert einen Gruppenisomorphismus
          \[
            H / (H \cap N)
            \mapsto
            HN / N \,,
            \quad
            \class{h}
            \mapsto
            \class{h} \,.
          \]
          Dies lässt sich grafisch wie folgt veranschaulichen.
          \[
            \begin{tikzcd}
              {}
              &
              HN
              &
              {}
              \\
              N
              \arrow[very thick, dash]{ur}
              &
              {}
              &
              H
              \arrow[dash]{ul}
              \\
              {}
              &
              H \cap N
              \arrow[dash]{ul}
              \arrow[very thick, dash]{ur}
              &
              {}
            \end{tikzcd}
          \]
      \end{enumerate}
    \item
      Es seien~$N$,~$K$ zwei Normalteiler in~$G$ mit~$N \subgroupeq K \subgroupeq G$.
      \begin{enumerate}
        \item
          Es ist~$N$ normal in~$K$, und es ist~$K/N$ normal in~$G/N$.
        \item
          Die Identität~$G \to G$ induziert einen wohldefinierten Gruppenisomorphismus
          \[
            (G/N) / (K/N)
            \to
            G / K \,,
            \quad
            \class{ \class{g} }
            \mapsto
            \class{g} \,.
          \]
          Beide Gruppen beschreiben im folgenden Diagram den obersten Teil.
          \[
            \begin{tikzcd}
              G
              \\
              K
              \arrow[very thick, dash]{u}
              \\
              N
              \arrow[dash]{u}
              \\
              1
              \arrow[dash]{u}
            \end{tikzcd}
          \]
      \end{enumerate}
  \end{enumerate}
\end{corollary}
