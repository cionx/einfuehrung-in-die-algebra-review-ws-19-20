\documentclass{scrartcl}

\usepackage{style_general}
\usepackage{style_sheets}



\titlehead{
  \textbf{Repetitorium: Einführung in die Algebra}%
%  \footnote{
%    Online verfügbar unter \url{https://gitlab.com/cionx/einfuehrung-in-die-algebra-review-ws-19-20}.
%  }
  \hfill
  3. März 2020
}
\title{\vspace{-1em}Tag 2}
\author{}
\date{}



\begin{document}

\maketitle
\vspace{-7em}

\begin{exercise}
  Es sei~$G$ eine~\Gruppe{$p$}.
  \begin{enumerate}
    \item
      Es sei~$X$ eine endliche~\Menge{$G$}.
      Zeigen Sie, dass~$\card{X} \equiv \card{ X^G } \Mod{p}$ gilt.
    \item
      Es gelte~$G \neq 1$.
      Zeigen Sie, dass auch~$\zenter(G) \neq 1$ gilt.
  \end{enumerate}
\end{exercise}

\begin{solution}
  \begin{enumerate}
    \item
      Es sei~$x_1, \dotsc, x_n$ ein Repräsentantensystem der~\Bahnen{$G$} von~$X$.
      Nach der Bahnenformel gilt
      \begin{equation}
        \label{orbit formula}
        \card{X}
        =
        \card{ X^ G }
        +
        \sum_{\substack{ i = 1, \dotsc, n \\ x_i \in X^G }}
        [G : G_{x_i}] \,,
      \end{equation}
      wobei der Index~$[G : G_{x_i}]$ für jeden Repräsentanten~$x_i$ mit~$x_i \notin X^G$ ein Teiler von~$\card{G}$ ist, der verschieden von~$1$ ist.
      Da~$G$ eine~\Gruppe{$p$} ist, gilt dabei~$p \divides [G : G_{x_i}]$ für jeden Repräsentanten~$x_i$ mit~$x_i \notin X^G$.
      Aus~\eqref{orbit formula} folgt somit, dass~$\card{X} \equiv \card{X^G} \Mod{p}$.
    \item
      Die Gruppe~$G$ wird auf~$X \defined G$ durch Konjugation, also vermöge~$g.x \defined g x g^{-1}$ für alle~$g \in G$,~$x \in X$.
      Es gilt~$\zenter(G) = X^G$ und somit nach dem vorherigen Aufgabenteil
      \[
        \card{\zenter(G)}
        =
        \card{X^G}
        \equiv
        \card{X}
        =
        \card{G}
        \equiv
        0
        \pmod{p} \,.
      \]
  \end{enumerate}
\end{solution}

\begin{exercise}
  \begin{enumerate}
    \item
      Bestimmen Sie für jede Primzahl~$p$ jeweils alle~\Sylow{$p$}-Untergruppen von~$\Sym_3$.
    \item
      Es sei~$p$ eine Primzahl es sei~$n \in \Natural_1$.
      Es sei~$\Borel(n, \Finite_p)$ die Untergruppe von~$\GL(n, \Finite_p)$ der invertierbaren oberen Dreiecksmatrizen.
      Bestimmen Sie eine~\Sylow{$p$}-Untergruppe von~$\Borel(n, \Finite_p)$.
  \end{enumerate}
\end{exercise}

\begin{solution}
  \begin{enumerate}
    \item
      Es ist~$\Sym_3 = \{ 1, \tau_{12}, \tau_{13}, \tau_{23}, \rho_1, \rho_2 \}$ wobeit~$\tau_{ij} = (i,j)$ Transpositionen sind und~$\rho_1 = (1,2,3)$ und~$\rho_2 = (3,2,1)$ zwei Dreierzykel sind.
      Die Untergruppen von~$\Sym_3$ sind gegeben durch
      \[
        \Sym_3 \,,
        \quad
        \gen{ \tau_{ij} } = \{ 1, \tau_{ij} \} \,,
        \quad
        \gen{ \rho_1 } = \gen{ \rho_2 } = \{ 1, \rho_1, \rho_2 \} \,,
        \quad
        1 \,.
      \]
      Für~$p \neq 2, 3$ ist~$1$ die eindeutige~\Sylow{$p$}-Untergruppe von~$\Sym_3$, die~\Sylow{$2$}-Untergruppen von~$\Sym_3$ sind~$\gen{ \tau_{ij} }$ und die eindeutige~\Sylow{$3$}-Untergruppe ist~$\gen{\rho_1} = \gen{\rho_2}$.
    \item
      Eine obere Dreicksmatrix in~$\Mat(n, \Finite_p)$ ist genau invertierbar, wenn ihre Diagonaleinträge nicht verschwinden, also Elemente von~$\Finite_p^\times$ sind.
      Die Einträge oberhalb der Diagonalen dürfen beliebig sein, und es gibt~$\binom{n}{2} = n(n-1)/2$ viele solche Einträge.
      Wir erhalten somit, dass
      \[
        \card{ \Borel(n, \Finite_p) }
        =
        (p-1)^n p^{n(n-1)/2} \,.
      \]
      Es sei~$\Uni(n, \Finite_p)$ die Untergruppe von~$\Borel(n, \Finite_p)$ all jener oberen Dreicksmatrizen, deren Diagonaleinträge alle Eins sind.
      Es gilt
      \[
        \card{ \Uni(n, \Finite_p) }
        =
        p^{n(n-1)/2} \,.
      \]
      Da~$p \ndivides (p-1)$ gilt, ist~$\Uni(n, \Finite_p)$ eine~\Sylow{$p$}-Untergruppe von~$\Borel(n, \Finite_p)$.

      Die Gruppe~$\Uni(n, \Finite_p)$ ist der Kern des Gruppenhomomorphismus
      \[
        \Borel(n, \Finite_p)
        \to
        ( \Finite_p^\times )^n \,,
        \quad
        \begin{bsmallmatrix}
          a_1 & *       & \cdots  & *       \\
              & \ddots  & \ddots  & \vdots  \\
              &         & \ddots  & *       \\
              &         &         & a_n
        \end{bsmallmatrix}
        \mapsto
        (a_1, \dotsc, a_n)
      \]
      und somit bereits eine normale Untergruppe von~$\Borel(n, \Finite_p)$.
      Insbesondere ist~$\Uni(n, \Finite_p)$ deshalb bereits die einzige~\Sylow{$p$}-Untergruppe von~$\Borel(n, \Finite_p)$.
  \end{enumerate}
\end{solution}

\begin{exercise}
  Es sei~$G$ eine endliche Gruppe mit~$\card{G} = 231$.
  Zeigen Sie, dass~$G$ mindestens zwei echte, nicht-triviale Normalteiler besitzt.
\end{exercise}


\begin{solution}
  Es gilt~$\card{G} = 231 = 3 \cdot 7 \cdot 11$.
  Für jede Primzahl~$p$ sei~$n_p$ die Anzahl der~\Sylow{$p$}-Untergruppen von~$G$.
  Nach dem dritten Sylowsatz ist~$n_p$ ein Teiler von~$\card{G}$, und somit
  \[
    n_p \in \{ 1, 3, 7, 11, 21, 33, 77, 231 \} \,.
  \]
  Nach dem driten Sylowsatz gilt außerdem~$n_p \equiv 1 \Mod{p}$.
  Wir erhalten damit, dass~$n_7 = 1$ und~$n_{11} = 1$.
  Die~\Sylow{$7$}- und~\Sylow{$11$}-Untergruppen von~$G$ sind also eindeutig, und somit normal.
  Diese Normalteiler sind nicht-trivial, da~$7$ und~$11$ Teiler von~$\card{G}$ sind.
\end{solution}

\begin{exercise}[subtitle = {Erstklausur 18/19}]
  Es sei~$G$ eine Gruppe.
  \begin{enumerate}
    \item
      Zeigen Sie, dass jeder Gruppenhomomorphismus~$G \to \Integer_2$ durch seinen Kern bestimmt ist.
    \item
      Zeigen Sie, dass es für jede Untergruppe~$N$ von~$G$ vom Index~$2$ einen Gruppenhomomorphismus~$\varphi \colon G \to \Integer_2$ gibt, so dass~$N = \ker(\varphi)$ gilt.
  \end{enumerate}
  Die Gruppe~$G$ sei nun endlich mit~$\card{G} = 54$.
  \begin{enumerate}[resume*]
    \item
      Bestimmen Sie, wieviele Gruppenhomomorphismen~$G \to \Integer_2$ es gibt.
  \end{enumerate}
\end{exercise}

\begin{solution}
  \begin{enumerate}
    \item
      Es sei~$\Integer_2 = \{1, s\}$, und es sei~$\varphi \colon G \to \Integer_2$ ein Gruppenhomomorphismus.
      Für~$g \in G$ gilt~$\varphi(g) = 1$ falls~$g \in \ker(\varphi)$, und~$\varphi(g) \neq 1$, und somit~$\varphi(g) = s$, falls~$g \notin \ker(\varphi)$.
      Der Funktionswert~$\varphi(g)$ ist also eindeutig durch~$\ker(\varphi)$ bestimmt.
    \item
      Die Untergruppe~$N$ ist bereits ein Normalteiler in~$G$, da~$[G : N] = 2$ gilt.
      Für die Faktorgruppe~$G/N$ gilt
      \[
        \card{G/N}
        =
        [G : N]
        =
        2
      \]
      und somit~$G/N \cong \Integer_2$.
      Wir fixieren einen Isomorphismus~$\psi \colon G/N \to \Integer_2$ und definieren~$\varphi \defined \psi \circ \pi$, wobei~$\pi \colon G \to G/N$ die kanonische Projektion bezeichnet.%
      \footnote{
        Es gibt tatsächlich nur genau einen Gruppenisomorphismus~$G/N \to \Integer_2$, weshalb~$\psi$ bereits eindeutig ist.
      }
      Es gilt
      \[
        \ker(\varphi) = \ker(\pi) = N \,,
      \]
      da~$\psi$ ein Isomorphismus ist.
    \item
      Ist~$\varphi \colon G \to \Integer_2$ ein nicht-trivialer Gruppenhomomorphismus, so gilt~$\im(\varphi) = \Integer_2$.
      Es gilt
      \[
        G / {\ker(\varphi)}
        \cong
        \im(\varphi)
      \]
      und somit
      \[
        [G : \ker(\varphi)]
        =
        \card{ G / {\ker(\varphi)} }
        =
        { \im(\varphi) }
        =
        2 \,.
      \]
      Zusammen mit den vorherigen beiden Aufgabenteilen erhalten wir somit eine Bijektion
      \begin{align*}
        \left\{
          \begin{tabular}{c}
            nicht-triviale Gruppenhomo-\\
            morphismen~$G \to \Integer_2$
          \end{tabular}
       \right\}
        &\to
        \left\{
          \begin{tabular}{c}
            Untergruppen von~$G$\\
            von Index~$2$
          \end{tabular}
        \right\} \,,
        \\
        \varphi
        &\mapsto
        \ker(\varphi) \,.
      \end{align*}
      Die Untergruppen von~$G$ von Index~$2$ sind genau die Untergruppen von Kardinalität~$27$, also die~\Sylow{$3$}-Untergruppen von~$G$.
      Es sei~$n_3$ die Anzahl dieser Untergruppen.
      Nach dem dritten Sylowsatz gilt~$n_3 \divides \card{G} = \{ 54 \}$ und somit
      \[
        n_3 \in \{ 1, 2, 3, 6, 9, 18, 27, 54 \} \,.
      \]
      Außerdem gilt~$n_3 \equiv 1 \Mod{3}$, und somit bereits~$n_3 = 1$.

      Wir haben nun insgesamt gezeigt, dass es nur einen nicht-trivialen Gruppenhomomorphismus~$G \to \Integer_2$ gibt.
      Zusammen mit dem trivialen Homomorphismus erhalten wir somit genau zwei Gruppenhomomorphismen~$G \to \Integer_2$.
  \end{enumerate}
\end{solution}

\begin{exercise}
  \leavevmode
  \begin{enumerate}
    \item
      Es sei~$d \in \Natural$ mit~$d \geq 2$.
      Zeigen Sie für den Unterring~$\Integer[\sqrt{-d}] \defined \{ a + i b \sqrt{d} \suchthat a, b \in \Integer \}$ von~$\Complex$, dass~$\Integer[\sqrt{-d}]^\times = \{1, -1\}$ gilt.
    \item
      Zeigen Sie für den Unterring~$\Integer[\sqrt{2}] \defined \{ a + b \sqrt{2} \suchthat a, b \in \Integer \}$ von~$\Real$, dass~$1 + \sqrt{2}$ eine Einheit in~$\Integer[\sqrt{2}]$ ist.
      Folgern Sie, dass die Einheitengruppe~$\Integer[\sqrt{2}]^\times$ unendlich ist.
  \end{enumerate}
\end{exercise}

\begin{solution}
  \begin{enumerate}
    \item
      Wie gehen analog zur Bestimmung der Einheitengruppe von~$\Integer[i]$ vor:
      Die Norm eines Elements~$z \in \Integer[\sqrt{-d}]$ mit~$z = a + i b \sqrt{d}$ für~$a, b \in \Integer$ ist
      \[
        N(z)
        \defined
        \abs{z}^2
        =
        a^2 + b d^2
        \in 
        \Natural \,.
      \]
      Die Norm ist multiplikativ und erfült~$N(1) = 1$.
      Für~$z \in \Integer[\sqrt{-d}]^\times$ gilt deshalb
      \[
        N(z) N(z^{-1})
        =
        N(z z^{-1})
        =
        N(1)
        =
        1 \,.
      \]
      Da~$N(z)$,~$N(z^{-1})$ zwei natürliche Zahlen sind, muss bereits~$N(z) = 1$ gelten.
      Da~$d \geq 2$ gilt, erhalten wir, dass bereits~$z = \pm 1$
      Andererseits sind~$1$ und~$-1$ Einheiten in~$\Integer[\sqrt{-d}]$.
    \item
      In~$\Real$ können wir das Inverse von~$1 + \sqrt{2}$ betrachten, und es gilt
      \[
        \frac{1}{1 + \sqrt{2}}
        =
        \frac{ 1 - \sqrt{2} }{ (1 + \sqrt{2}) (1 - \sqrt{2}) }
        =
        \frac{1 - \sqrt{2}}{1 - 2}
        =
        \frac{1 - \sqrt{2}}{-1}
        =
        -1 + \sqrt{2}
        \in
        \Integer[\sqrt{2}] \,,
      \]
      Deshalb ist~$1 + \sqrt{2}$ eine Einheit in~$\Integer[\sqrt{2}]$.

      Es gilt~$1 + \sqrt{2} > 1$ weshalb die Potenzen~$(1 + \sqrt{2})^n$ für~$n \in \Natural_1$ paarweise verschieden sind.
      Da~$1 + \sqrt{2}$ eine Einheit in~$\Integer[\sqrt{2}]$ ist, sind auch diese Potenzen Einheiten in~$\Integer[\sqrt{2}]$.
      Wir haben somit unendlich Elemente von~$\Integer[\sqrt{2}]^\times$ konstruiert.
  \end{enumerate}
\end{solution}

\begin{exercise}
  Es sei~$R$ ein kommutativer Ring.
  \begin{enumerate}
    \item
      Es sei~$I$ ein Ideal in~$R$.
      Zeigen Sie, dass
      \[
        I[X]
        \defined
        \biggl\{
          \sum_i a_i X^i \in R[X]
        \suchthat[\bigg]
          \text{$a_i \in I$ für jeden Index~$i$}
          \spacing
        \biggr\}
      \]
      ein Ideal in~$R[X]$ ist, so dass~$R[X] / I[X] \cong (R/I)[X]$ gilt.
    \item
      Entscheiden Sie, ob~$\pideal[X]$ ein Primideal ist, wenn~$\pideal$ ein Primideal in~$R$ ist.
    \item
      Entscheiden Sie, ob~$\mideal[X]$ ein maximales Ideal ist, wenn~$\mideal$ ein maximales Ideal in~$R$ ist.
  \end{enumerate}
\end{exercise}

\begin{solution}
  \begin{enumerate}
    \item
      Die kanonische Projektion~$\pi \colon R \to R/I$ induziert durch koeffizientenweises Anwenden einen Ringhomomorphismus
      \[
        \pi_*
        \colon
        R[X]
        \to
        (R/I)[X] \,,
        \quad
        \sum_i r_i X^i
        \mapsto
        \sum_i \class{r_i} X^i \,.
      \]
      Aus der Surjektivität von~$\pi$ und~$\pi(0) = 0$ ergibt sich, dass der Homomorphismus~$\pi_*$ ebenfalls surjektiv ist.
      Außerdem gilt~$\ker(\pi_*) = I[X]$.
      Es folgt, dass~$I[X]$ ein Ideal in~$R[X]$ ist, und dass
      \[
        R[X] / I[X]
        =
        R[X] / {\ker(\pi_*)}
        \cong
        \im(\pi_*)
        =
        (R/I)[X] \,.
      \]
    \item
      Der Faktorring~$R / \pideal$ ist ein Integritätsbereich, da~$\pideal$ ein Primideal ist.
      Also ist auch der Faktorring~$R[X] / \pideal[X] \cong (R/\pideal)[X]$ ein Integritätsbereich.
      Deshalb ist~$\pideal[X]$ ein Primideal in~$R[X]$.
    \item
      Die Maximalität von~$R$ bedeutet, dass der Faktorring~$R / \mideal$ ein Körper ist.
      Aber der Faktorring~$R[X] / \mideal[X] \cong (R/\mideal)[X]$ ist kein Körper mehr, weshalb das Ideal~$\mideal[X]$ nicht maximal ist.
  \end{enumerate}
\end{solution}

\begin{exercise}
  Es sei~$R$ ein kommutativer Ring und~$I$ ein Ideal in~$R$.
  \begin{enumerate}
    \item
      Es sei~$\pideal$ ein Ideal in~$R$, das~$I$ enthält.
      Zeigen Sie, dass das Ideal~$\pideal$ genau dann prim in~$R$ ist, wenn das Ideal~$\pideal/I$ prim in~$R/I$ ist.
    \item
      Es sei~$\mideal$ ein Ideal in~$R$, das~$I$ enthält.
      Zeigen Sie, dass das Ideal~$\mideal$ genau dann maximal in~$R$ ist, wenn das Ideal~$\mideal/I$ maximal in~$R/I$ ist.
  \end{enumerate}
\end{exercise}

\begin{solution}
  \begin{enumerate}
    \item
      Nach dem Noetherschen Isomorphiesatz gilt~$(R/I) / (\pideal/I) \cong R/\pideal$.
      Das Ideal~$\pideal$ ist genau dann prim wenn der Faktorring~$R / \pideal$ ein Integritätsbereich ist, und das Ideal~$\pideal/I$ ist genau dann prim wenn der Faktorring~$(R/I) / (\pideal/I)$ ein Integritätsbereich ist.
      Dank des obigen Isomorphismus’ erhalten wir hieraus die gewünschte Äquivalenz.
    \item
      Wir können wie im vorherigen Aufgabenteil vorgehen, indem wir \enquote{Integritätsbreich} durch \enquote{Körper} ersetzen.
  \end{enumerate}
\end{solution}

\begin{exercise}
  Zeigen Sie, dass der Unterring~$\Integer\bigl[ \sqrt{-2} \bigr] \defined \{ a + i b \sqrt{2} \suchthat a, b \in \Integer \}$ von~$\Complex$ euklidisch ist.
\end{exercise}

\begin{solution}
  Wir gehen wie für~$\Integer[i]$ vor:
  Die Norm von~$z \in \Integer[\sqrt{-2}]$ mit~$z = a + i b \sqrt{2}$ für~$a, b \in \Integer$ ist
  \[
    N(z)
    =
    \abs{z}^2
    =
    a^2 + 2 b^2
    \in
    \Natural \,.
  \]
  Es seien~$a, b \in \Integer[\sqrt{-2}]$ mit~$b \neq 0$.
  Für die komplexe Zahl~$a/b$ gibt ein Element~$q \in \Integer[\sqrt{-2}]$ mit
  \[
    \abs*{ \frac{a}{b} - q }
    <
    \frac{ \sqrt{3} }{2} \,.
  \]
  (Denn~$\Integer[\sqrt{-2}$ ist ein Gitter in~$\Complex$, mit Breite~$1$ und Höhe~$\sqrt{2}$.)
  Es sei~$r \defined a - bq$.
  Per Definition von~$r$ gilt~$a = qb + r$, und durch die Wahl von~$q$ erhalten wir
  \[
    N(r)
    =
    N(a - bq)
    =
    \abs{a - bq}^2
    =
    \abs*{ \frac{a}{b} - q }^2 \abs{b}^2
    \leq
    \frac{3}{4} N(b)
  \]
  mit~$3/4 < 1$.
  Das zeigt, dass der Ring~$\Integer[-\sqrt{2}]$ zusammen mit der Norm~$N$ als Gradabbildung euklidisch ist.
\end{solution}

\begin{exercise}
  Zeigen Sie, dass jeder endliche Integritätsbereich bereits ein Körper ist.
\end{exercise}

\begin{solution}
  Es sei~$K$ ein endlicher Integritätsbereich.
  Es sei~$x \in K$ mit~$x \neq 0$.
  Die Multiplikationsabbildung
  \[
    \lambda
    \colon
    K \to K \,,
    \quad
    y \mapsto xy
  \]
  ist ein Homomorphismus von additiven Gruppen, und es gilt~$\ker(\lambda) = 0$ da~$K$ ein Integritätsbereich ist.
  Aus der Endlichkeit von~$K$ folgt, dass~$\lambda$ auch surjektiv ist.
  Also gibt es ein Element~$y \in K$ mit~$\lambda(y) = 1$, also~$x y = 1$.
  Somit ist~$x$ eine Einheit in~$K$.
\end{solution}





\clearpage





\printsolutions





\end{document}
