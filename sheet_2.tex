\documentclass{scrartcl}

\usepackage{style_general}
\usepackage{style_sheets}



\titlehead{
  \textbf{Repetitorium: Einführung in die Algebra}%
%  \footnote{
%    Online verfügbar unter \url{https://gitlab.com/cionx/einfuehrung-in-die-algebra-review-ws-19-20}.
%  }
  \hfill
  3. März 2020
}
\title{\vspace{-1em}Tag 2}
\author{}
\date{}



\begin{document}

\maketitle
\vspace{-7em}

\begin{exercise}
  Es sei~$G$ eine~\Gruppe{$p$}.
  \begin{enumerate}
    \item
      Es sei~$X$ eine endliche~\Menge{$G$}.
      Zeigen Sie, dass~$\card{X} \equiv \card{ X^G } \Mod{p}$ gilt.
    \item
      Es gelte~$G \neq 1$.
      Zeigen Sie, dass auch~$\zenter(G) \neq 1$ gilt.
  \end{enumerate}
\end{exercise}

\begin{exercise}
  \begin{enumerate}
    \item
      Bestimmen Sie für jede Primzahl~$p$ jeweils alle~\Sylow{$p$}-Untergruppen von~$\Sym_3$.
    \item
      Es sei~$p$ eine Primzahl es sei~$n \in \Natural_1$.
      Es sei~$\Borel(n, \Finite_p)$ die Untergruppe von~$\GL(n, \Finite_p)$ der invertierbaren oberen Dreiecksmatrizen.
      Bestimmen Sie eine~\Sylow{$p$}-Untergruppe von~$\Borel(n, \Finite_p)$.
  \end{enumerate}
\end{exercise}

\begin{exercise}
  Es sei~$G$ eine endliche Gruppe mit~$\card{G} = 231$.
  Zeigen Sie, dass~$G$ mindestens zwei echte, nicht-triviale Normalteiler besitzt.
\end{exercise}

\begin{exercise}[subtitle = {Erstklausur 18/19}]
  Es sei~$G$ eine Gruppe.
  \begin{enumerate}
    \item
      Zeigen Sie, dass jeder Gruppenhomomorphismus~$G \to \Integer_2$ durch seinen Kern bestimmt ist.
    \item
      Zeigen Sie, dass es für jede Untergruppe~$N$ von~$G$ vom Index~$2$ einen Gruppenhomomorphismus~$\varphi \colon G \to \Integer_2$ gibt, so dass~$N = \ker(\varphi)$ gilt.
  \end{enumerate}
  Die Gruppe~$G$ sei nun endlich mit~$\card{G} = 54$.
  \begin{enumerate}[resume*]
    \item
      Bestimmen Sie, wieviele Gruppenhomomorphismen~$G \to \Integer_2$ es gibt.
  \end{enumerate}
\end{exercise}

\begin{exercise}
  \leavevmode
  \begin{enumerate}
    \item
      Es sei~$d \in \Natural$ mit~$d \geq 2$.
      Zeigen Sie für den Unterring~$\Integer[\sqrt{-d}] \defined \{ a + i b \sqrt{d} \suchthat a, b \in \Integer \}$ von~$\Complex$, dass~$\Integer[\sqrt{-d}]^\times = \{1, -1\}$ gilt.
    \item
      Zeigen Sie für den Unterring~$\Integer[\sqrt{2}] \defined \{ a + b \sqrt{2} \suchthat a, b \in \Integer \}$ von~$\Real$, dass~$1 + \sqrt{2}$ eine Einheit in~$\Integer[\sqrt{2}]$ ist.
      Folgern Sie, dass die Einheitengruppe~$\Integer[\sqrt{2}]^\times$ unendlich ist.
  \end{enumerate}
\end{exercise}

\begin{exercise}
  Es sei~$R$ ein kommutativer Ring.
  \begin{enumerate}
    \item
      Es sei~$I$ ein Ideal in~$R$.
      Zeigen Sie, dass
      \[
        I[X]
        \defined
        \biggl\{
          \sum_i a_i X^i \in R[X]
        \suchthat[\bigg]
          \text{$a_i \in I$ für jeden Index~$i$}
          \spacing
        \biggr\}
      \]
      ein Ideal in~$R[X]$ ist, so dass~$R[X] / I[X] \cong (R/I)[X]$ gilt.
    \item
      Entscheiden Sie, ob~$\pideal[X]$ ein Primideal ist, wenn~$\pideal$ ein Primideal in~$R$ ist.
    \item
      Entscheiden Sie, ob~$\mideal[X]$ ein maximales Ideal ist, wenn~$\mideal$ ein maximales Ideal in~$R$ ist.
  \end{enumerate}
\end{exercise}

\begin{exercise}
  Es sei~$R$ ein kommutativer Ring und~$I$ ein Ideal in~$R$.
  \begin{enumerate}
    \item
      Es sei~$\pideal$ ein Ideal in~$R$, das~$I$ enthält.
      Zeigen Sie, dass das Ideal~$\pideal$ genau dann prim in~$R$ ist, wenn das Ideal~$\pideal/I$ prim in~$R/I$ ist.
    \item
      Es sei~$\mideal$ ein Ideal in~$R$, das~$I$ enthält.
      Zeigen Sie, dass das Ideal~$\mideal$ genau dann maximal in~$R$ ist, wenn das Ideal~$\mideal/I$ maximal in~$R/I$ ist.
  \end{enumerate}
\end{exercise}

\begin{exercise}
  Zeigen Sie, dass der Unterring~$\Integer\bigl[ \sqrt{-2} \bigr] \defined \{ a + i b \sqrt{2} \suchthat a, b \in \Integer \}$ von~$\Complex$ euklidisch ist.
\end{exercise}

\begin{exercise}
  Zeigen Sie, dass jeder endliche Integritätsbereich bereits ein Körper ist.
\end{exercise}

\begin{exercise}
  Es sei~$R$ ein kommutativer Ring und~$I$ ein Ideal in~$R$ mit Komplement~$S \defined R \setminus I$.
  Zeigen Sie, dass das Ideal~$I$ genau dann prim ist, wenn die Menge~$S$ multiplikativ abgeschlossen ist.
\end{exercise}

%\begin{exercise}
%  Es sei~$R$ ein kommutativer Ring.
%  Es sei~$r$ ein Element von~$R$ und es sei~$\lambda_r \colon R \to R$ die Multiplikation mit~$r$, d.h. es gelte~$\lambda_r(x) = rx$ für jedes Element~$x \in R$.
%  \begin{enumerate}
%    \item
%      Zeigen Sie, dass die folgenden Bedingungen äquivalent sind.
%      \begin{equivlist}
%        \item
%          Das Element~$r$ ist eine Einheit in~$R$.
%        \item
%          Die Abbildung~$\lambda_r$ ist bijektiv.
%        \item
%          Die Abbildung~$\lambda_r$ ist surjektiv.
%      \end{equivlist}
%    \item
%      Zeigen Sie, dass~$r$ genau dann ein Nullteiler ist, wenn die Abbildung~$\lambda_r$ nicht injektiv ist.
%  \end{enumerate}
%  Der kommutative Ring~$R$ sei nun endlich.
%  \begin{enumerate}[resume*]
%    \item
%      Folgern Sie, dass jedes Element von~$R$ entweder eine Einheit oder ein Nullteiler ist.
%    \item
%      Folgern Sie ferner, dass jeder endliche Integritätsbereich bereits ein Körper ist.
%  \end{enumerate}
%\end{exercise}

%\begin{exercise}
%  Es sei~$R$ ein kommutativer Ring.
%  Zeigen Sie, dass~$R[X,Y] / (Y - X^2) \cong R[X]$ gilt.
%\end{exercise}

\begin{exercise}
  Es sei~$R$ ein kommutativer Ring und es sei~$\pideal$ ein Ideal in~$R$.
  Zeigen Sie, dass das Ideal~$\pideal$ genau dann prim ist, wenn es einen Körper~$K$ und einen Ringhomomorphismus~$\varphi \colon R \to K$ gibt, so dass~$\pideal = \ker(\varphi)$ gilt.
\end{exercise}

\end{document}
