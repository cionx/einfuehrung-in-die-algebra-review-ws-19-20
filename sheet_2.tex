\documentclass{scrartcl}

\usepackage{style_general}
\usepackage{style_sheets}



\titlehead{
  \textbf{Repetitorium: Einführung in die Algebra}%
%  \footnote{
%    Online verfügbar unter \url{https://gitlab.com/cionx/einfuehrung-in-die-algebra-review-ws-19-20}.
%  }
  \hfill
  3. März 2020
}
\title{Tag 2}
\author{}
\date{}



\begin{document}

\maketitle
\vspace{-6em}

\begin{exercise}
  Es sei~$p$ eine Primzahl es sei~$n \in \Natural_1$.
  Es sei~$\Borel(n, \Finite_p)$ die Untergruppe von~$\GL(n, \Finite_p)$ der den invertierbaren oberen Dreiecksmatrizen.
  Bestimmen Sie eine~\Sylow{$p$}-Untergruppe von~$\Borel(n, \Finite_p)$.
\end{exercise}

\begin{exercise}
  Es sei~$G$ eine endliche Gruppe mit~$\card{G} = 231$.
  Zeigen Sie, dass~$G$ mindestens zwei echte, nicht-triviale Normalteiler besitzt.
\end{exercise}

\begin{exercise}
  Es sei~$G$ eine endliche Menge mit~$\card{G} = 24$.
\end{exercise}

\begin{exercise}
  Es sei~$R$ ein koummtativer Ring.
  Es sei~$r$ ein Element von~$R$ und es sei~$\lambda_r \colon R \to R$ die Multiplikation mit~$r$, d.h. es gelte~$\lambda_r(x) = rx$ für jedes Element~$x \in R$.
  \begin{enumerate}
    \item
      Zeigen Sie, dass die folgenden Bedingungen äquivalent sind.
      \begin{equivlist}
        \item
          Das Element~$r$ ist eine Einheit in~$R$.
        \item
          Die Abbildung~$\lambda_r$ ist bijektiv.
        \item
          Die Abbildung~$\lambda_r$ ist surjektiv.
      \end{equivlist}
    \item
      Zeigen Sie, dass das Element~$r$ genau dann ein Nullteiler ist, wenn die Abbildung~$\lambda_r$ nicht injektiv ist.
  \end{enumerate}
  Der kommutative Ring~$R$ sei nun endlich.
  \begin{enumerate}[resume*]
    \item
      Folgern Sie, dass jedes Element von~$R$ entweder eine Einheit oder ein Nullteiler ist.
    \item
      Folgern Sie ferner, dass jeder endliche Integritätsbereich ein Körper ist.
  \end{enumerate}
\end{exercise}

\begin{exercise}
  Es sei~$R$ ein kommutativer Ring.
  \begin{enumerate}
    \item
      Es sei~$I$ ein Ideal in~$R$.
      Zeigen Sie, dass
      \[
        I[X]
        \defined
        \biggl\{
          \sum_i a_i X^i \in R[X]
        \suchthat[\bigg]
          \text{$a_i \in I$ für jeden Index~$i$}
          \spacing
        \biggr\}
      \]
      ein Ideal in~$R[X]$ ist, so dass~$R[X] / I[X] \cong (R/I)[X]$ gilt.
    \item
      Entscheiden Sie, ob~$\pideal[X]$ ein Primideal ist, wenn~$\pideal$ ein Primideal in~$R$ ist.
    \item
      Entscheiden Sie, ob~$\mideal[X]$ ein maximales Ideal ist, wenn~$\mideal$ ein maximales Ideal in~$R$ ist.
  \end{enumerate}
\end{exercise}


\begin{exercise}
  Zeigen Sie, dass der Ring~$\Integer\bigl[ \sqrt{-2} \bigr] \defined \{ a + i b \sqrt{2} \suchthat a, b \in \Integer \}$ euklidisch ist.
\end{exercise}

\begin{exercise}
  Es sei~$I$ ein Ideal in einem kommutativen Ring~$R$, und es sei~$S \defined R \setminus I$ das Komplement von~$I$.
  Zeigen Sie, dass das Ideal~$I$ genau dann prim ist, wenn die Menge~$S$ multiplikativ abgeschlossen ist.
\end{exercise}

\begin{exercise}
  Es sei~$R$ ein kommutativer Ring und es sei~$\pideal$ ein Ideal in~$R$.
  Zeigen Sie, dass das Ideal~$\pideal$ genau dann prim ist, wenn es einen Körper~$K$ und einen Ringhomomorphismus~$\varphi \colon R \to K$ gibt, so dass~$\pideal = \ker(\varphi)$ gilt.
\end{exercise}

\end{document}
