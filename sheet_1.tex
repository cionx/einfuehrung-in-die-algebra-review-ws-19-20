\documentclass{scrartcl}

\usepackage{style_general}
\usepackage{style_sheets}



\titlehead{
  \textbf{Repetitorium: Einführung in die Algebra}%
%  \footnote{
%    Online verfügbar unter \url{https://gitlab.com/cionx/einfuehrung-in-die-algebra-review-ws-19-20}.
%  }
  \hfill
  2. März 2020
}
\title{\vspace{-1em}Tag 1}
\author{}
\date{}



\begin{document}

\maketitle
\vspace{-7em}

% TODO: Burnside.

\begin{exercise}
  \begin{enumerate}
    \item
      Zeigen Sie, dass jede echte Untergruppe der symmetrischen Gruppe~$\Sym_3$ zyklisch ist.
    \item
      Bestimmen Sie alle Untergruppen von~$\Sym_3$.
  \end{enumerate}
\end{exercise}

\begin{solution}
  \begin{enumerate}
    \item
      Ist~$H$ eine Untergruppe von~$\Sym_3$, so ist~$\card{H}$ nach dem Satz von Lagrange ein Teiler von~$\card{\Sym_3} = 6$.
      Ist~$H$ zudem eine echte Untergruppe, so ist~$\card{H}$ ein echter Teiler von~$6$.
      Es gilt dann also~$\card{H} \in \{1, 2, 3\}$.

      Im Fall~$\card{H} = 1$ gilt~$H = 1$;
      insbesondere ist~$H$ in diesem Fall zyklisch.
      In den Fällen~$\card{H} = 2$ und~$\card{H} = 3$ ist~$\card{H}$ prim und~$H$ somit zyklisch.
    \item
      Es gilt~$\Sym_3 = \{ 1, \tau_{12}, \tau_{13}, \tau_{23}, \rho_1, \rho_2 \}$, wobei~$\tau_{ij} \defined (i,j)$ die Transpositonen sind und~$\rho_1 \defined (1, 2, 3)$ und~$\rho_2 \defined (3, 2, 1)$ zwei Dreierzykel sind.
      Es gilt~$\tau_{ij}^2 = 1$, sowie~$\rho_i^3 = 1$ und~$\rho_1 = \rho_2^{-1}$.
      Da nach dem vorherigen Aufgabenteil jede echte Untergruppe von~$G$ bereits zyklisch ist, ergeben sich für~$\Sym_3$ die folgenden Untergruppen:
      \begin{align*}
        \Sym_3 &{} {}
        \\
        \gen{1} &= 1
        \\
        \gen{ \tau_{12} } &= \{ 1, \tau_{12} \} \,,
        \\
        \gen{ \tau_{13} } &= \{ 1, \tau_{13} \} \,,
        \\
        \gen{ \tau_{23} } &= \{ 1, \tau_{23} \} \,,
        \\
        \gen{ \rho_1 } &= \gen{ \rho_2 } = \{ 1, \rho_1, \rho_2 \} \,.
      \end{align*}
  \end{enumerate}
\end{solution}

\begin{exercise}
  Es seien~$G$,~$H$ zwei endliche Gruppen deren Ordnungen~$\card{G}$,~$\card{H}$ teilerfremd sind.
  Zeigen Sie, dass jeder Gruppenhomomorphismus~$\varphi \colon G \to H$ bereits trivial ist.
\end{exercise}

\begin{solution}
  Das Bild~$\im(\varphi)$ ist eine Untergruppe von~$H$.
  Nach dem Satz von Lagrange ist deshalb~$\card{\im(\varphi)}$ ein Teiler von~$\card{H}$.
  Andererseits gilt~$\im(\varphi) \cong G / \ker(\varphi)$ und deshalb
  \[
    \card{ \im(\varphi) }
    =
    \card{ G / {\ker(\varphi)} }
    =
    [G : \ker(\varphi)] \,.
  \]
  Der Index~$[G : \ker(\varphi)]$ ist nach dem Satz von Lagrange ein Teiler von~$\card{G}$.
  Also ist~$\card{ \im(\varphi) }$ auch ein Teiler von~$\card{G}$.

  Wir haben nun gezeigt, dass~$\card{ \im(\varphi) }$ ein gemeinsamer Teiler von~$\card{G}$ und~$\card{H}$ ist.
  Es folgt, dass~$\card{ \im(\varphi) } = 1$ gilt, da~$\card{G}$ und~$\card{H}$ teilerfremd sind.
  Also gilt~$\im(\varphi) = 1$, weshalb~$\varphi$ der triviale Gruppenhomomorphismus ist.
\end{solution}

\begin{exercise}
  Es sei~$G$ eine Gruppe mit~$\card{G} = 77$, und es sei~$X$ eine Menge mit~$\card{X} = 17$.
  Zeigen Sie, dass jede Gruppenaktion von~$G$ auf~$X$ mindestens fünf Bahnen und mindestens drei Fixpunkte besitzt.
\end{exercise}

\begin{solution}~
  Für jedes Element~$x \in X$ gilt~$\card{G.x} = [G : G_x]$, wobei der Index~$[G : G_x]$ nach dem Satz von Lagrange ein Teiler von~$\card{G} = 77$ ist.
  Die Kardinalität der Bahn~$G.x$ kann also nur einer der Werte
  \[
    1 \,,
    \quad
    7 \,,
    \quad
    11 \,,
    \quad
    77
  \]
  sein.
  Da~$X$ die disjunkte Vereinigung der~\Bahnen{$G$} ist, erhalten wir die folgenden möglichen Bahnengrößen:
  \begin{align*}
    17
    &=
    11 + \underbrace{1 + \dotsb + 1}_{6}
    \\
    &=
    7 + 7 + 1 + 1 + 1
    \\
    &=
    7 + \underbrace{1 + \dotsb + 1}_{10}
    \\
    &=
    \underbrace{1 + \dotsb + 1}_{17} \,.
  \end{align*}
  Die Anzahl der Bahnen entspricht jeweils der Anzahl der auftretenden Summanden, und die Anzahl der Fixpunkte der Vielfachheit des Summanden~$1$.
  Wir erhalten, dass es mindestens fünf Bahnen gibt, da jede Zerlegung aus mindestens fünf Summanden besteht, und es mindestens drei Fixpunkte gibt, da in jeder Zerlegung der Summand~$1$ mindestens dreimal vorkommt.
\end{solution}

\begin{exercise}
  Es sei~$G$ eine Gruppe und~$X$ eine~\Menge{$G$}.
  \begin{enumerate}
    \item
      Zeigen Sie, dass~$G_{g.x} = g G_x g^{-1}$ für jedes Element~$x \in X$ und jedes Gruppenelement~$g \in G$.
    \item
      Die obige Gleichheit zeigt, dass je zwei Elemente in der gleichen~\Bahn{$G$} konjugierte Stabilisatoren haben.
      Entscheiden Sie ob auch die Umkehrung dieser Aussage gilt, d.h. ob Element von~$X$ mit konjugierten Stabilisatoren bereits in der gleichen~\Bahn{$G$} liegen müssen.
  \end{enumerate}
\end{exercise}

\begin{solution}
  \begin{enumerate}
    \item
      Wir zeigen zunächst die Inklusion~$g G_x g^{-1} \subseteq G_{g.x}$.
      Hierfür sei~$h \in g G_x g^{-1}$.
      Dann gibt es ein Element~$h' \in G_x$ mit~$h = g h' g^{-1}$.
      Deshalb gilt
      \[
        h.g.x
        =
        g h' g^{-1} . g.x
        =
        g h ' . x
        =
        g . x \,.
      \]
      Das zeigt, dass~$h \in G_{g.x}$ gilt.

      Für die umgekehrte Inklusion~$G_{g.x} \subseteq g G_x g^{-1}$ geben wir zwei Beweise an.
      \begin{enumerate}
        \item
          Es sei~$h \in G_{g.x}$.
          Wir wollen zeigen, dass~$h \in g G_x g^{-1}$ gilt.
          Wir müssen also ein Element~$h' \in G_x$ finden, so dass~$h = g h' g^{-1}$ gilt.
          Durch Umstellen dieser gewollten Gleichung erhalten wir, dass wir~$h'$ als~$h' \defined g^{-1} h g$ definieren müssen.
          Dieses Element~$h'$ hat die gewünschten Eigenschaften, denn
          \begin{gather*}
            g h' g^{-1}
            =
            g g^{-1} h g g^{-1}
            =
            h \,,
          \shortintertext{sowie}
            h' . x
            =
            g^{-1} h g . x
            =
            g^{-1} h. (g.x)
            =
            g^{-1} . (g.x)
            =
            x
          \end{gather*}
          und somit~$h' \in G_{g.x}$.
        \item
          Es gilt~$x = g^{-1}.(g.x)$.
          Durch Awenden der bereits gezeigten Inklusion erhalten wir somit, dass
          \[
            G_x
            =
            G_{g^{-1}.(g.x)}
            \supseteq
            g^{-1} G_{g.x} g \,,
          \]
          gilt und somit~$g G_x g^{-1} \supseteq G_{g.x}$, also~$G_{g.x} \subseteq g G_x g^{-1}$.
      \end{enumerate}
  \end{enumerate}
\end{solution}

\begin{exercise}
  Es sei~$G$ eine endliche Gruppe, die aus genau zwei Konjugationsklassen besteht.
  Zeigen Sie, dass bereits~$G \cong \Integer_2$ gilt.

  (\emph{Tipp}: Bestimmen Sie die Größen der beiden Konjugationsklassen.)
\end{exercise}

\begin{solution}
  Es seien~$x_1$,~$x_2$ ein Repräsentantensystem für die Konjugationsklassen von~$G$.
  Es ist~$G$ die disjunkte Vereinigung der beiden Konjugationsklassen~$x_1^G$ und~$x_2^G$.
  Für~$n \defined \card{G}$ gilt somit
  \[
    n
    =
    \card{x_1^G} + \card{x_2^G}
  \]
  Die Konjugationsklasse des neutralen Elements ist gegeben durch~$1^G = \{ 1 \}$.
  Somit ist eine der beiden Konjugationsklassen~$x_i^G$ einelementig (und dann gilt~$x_i = 1$).
  Wir können davon ausgehen, dass~$x_1^G = \{ 1 \}$ (also~$x_1 = 1$).
  Damit gilt
  \[
    \card{ x_2^G }
    =
    \card{G} - \card{ x_1^G }
    =
    n - 1 \,.
  \]
  Dabei gilt
  \[
    \card{ x_2^G }
    =
    [G : \zentralizer_G(x_2)] \,,
  \]
  wobei nach dem Satz von Lagrange der Index~$[G : \zentralizer_G(x_2)]$ ein Teiler der Ordnung~$\card{G} = n$ ist.
  Wir erhalten somet, dass
  \[
    (n-1) \divides n \,.
  \]
  Dies gilt nur für~$n = 2$.

  Wir haben nun gezeugt, dass~$\card{G} = 2$ gilt.
  Da~$2$ eine Primzahl ist, folgt ferner, dass~$G \cong \Integer_2$.
\end{solution}

\begin{exercise}
  Es sei~$G$ eine Gruppe und~$N$ eine normale Untergruppe von~$G$, so dass die Faktorgruppe~$G/N$ abelsch ist.
  Zeigen Sie, dass jede Untergruppe von~$G$, die~$N$ enthält, ebenfalls normal in~$G$ ist.
\end{exercise}

\begin{solution}
  Es sei~$H$ eine Untergruppe von~$G$, welche den Normalteiler~$N$ enthält.
  Die Untergruppe~$H$ entspricht der Untergruppe~$H/N$ von~$G/N$, und wir wissen, dass die Untergruppe~$H$ genau dann ein Normalteiler in~$G$ ist, wenn~$H/N$ ein Normalteiler in~$G/N$ ist.
  Da die Faktorgruppe~$G/N$ abelsch ist, ist jede Untergruppe von~$G/N$ ein Normalteiler, also insbesonder auch~$H/N$.
  Also ist~$H$ ein Normalteiler in~$G$.
\end{solution}

\begin{exercise}
  Es sei~$G$ eine Gruppe und~$H$ eine Untergruppe von~$G$ vom endlichen Index.
  Zeigen Sie, dass~$H$ eine Untergruppe~$N$ enthält, die normal in~$G$ ist und ebenfalls endlichen Index in~$G$ hat.

  (\emph{Tipp}: Konstruieren Sie für~$n \defined [G : H]$ einen Gruppenhomomorphismus~$G \to \Sym_n$, dessen Kern in~$H$ enthalten ist.)
\end{exercise}

\begin{solution}
  Es sei~$X \defined G/H$ die Menge der Linksnebenklassen von~$G$ bzgl.~$H$.
  Die Gruppe~$G$ wirkt auf der Menge~$X$ vermöge
  \[
    g . (g' H)
    \defined
    (g g') H
    \qquad
    \text{für alle~$g \in G$,~$g'H \in X$.}
  \]
  Diese Gruppenaktion von~$G$ auf~$X$ entspricht einem Gruppenhomomorphismus
  \[
    \varphi
    \colon
    G \to \Sym_X \,.
  \]
  Dieser Gruppenhomomorphismus ist dadurch gegeben, dass
  \[
    g.x = \varphi(g)(x)
    \qquad
    \text{für alle~$g \in G$,~$x \in X$.}
  \]
  Wir behaupten, dass~$N \defined \ker(\varphi)$ die gewünschten Eigenschaften hat.

  Es ist~$N$ ein Normalteiler in~$G$, da es sich um den Kern eines Gruppenhomomorphismus handelt.

  Es sei~$g \in \ker(\varphi)$.
  Für die Nebenklasse~$x \in X$ gegeben durch~$x = 1H$ gilt zum einen
  \[
    g.x
    =
    g . (1H)
    =
    (g \cdot 1) H
    =
    gH \,,
  \]
  und zum anderen
  \[
    g.x
    =
    \varphi(g)(x)
    =
    \id_X(x)
    =
    x
    =
    1H \,.
  \]
  Es gilt also~$gH = 1H$.
  Dies ist äquivalent dazu, dass~$1^{-1} g \in H$ gilt, also~$g \in H$.
  Wir haben somit gezeigt, dass~$\ker(\varphi) \subseteq H$ gilt.

  Es gilt~$\card{X} = \card{G/H} = [G : H]$.
  Nach Annahmen ist der Index~$[G : H]$ endlich, weshalb die Menge~$X$ endlich ist.
  Somit ist auch auch die Gruppe~$\Sym_X$ endlich.
  (Für~$n \defined \card{X}$ gilt~$\Sym_X \cong \Sym_n$ und somit~$\card{ \Sym_X } = n!$.)
  Es gilt~$G / {\ker(\varphi)} \cong \im(\varphi)$ und somit
  \[
    [G : \ker(\varphi)]
    =
    \card{ G / {\ker(\varphi)} }
    =
    \card{ \im(\varphi) }
    \leq
    \card{ \Sym_X } \,.
  \]
  Wir haben somit gezeigt, dass~$\ker(\varphi)$ endlichen Index in~$G$ hat.
\end{solution}

%\begin{exercise}
%  Es sei~$G$ eine Gruppen.
%  Für jedes Element~$g \in G$ sei~$c_g \colon G \to G$ die Konjugation mit~$g$, d.h. es sel~$c_g(x) = g x g^{-1}$ für jedes~$x \in G$.
%  \begin{enumerate}
%    \item
%      Zeigen Sie, dass~$c_g$ für jedes~$g \in G$ ein Gruppenautomorphismus von~$G$ ist.
%    \item
%      Zeigen Sie, dass die Abbildung~$c \colon G \to \Aut(G)$ gegeben durch~$g \mapsto c_g$ ein Gruppenhomomorphismus ist.
%    \item
%      Zeigen Sie, dass~$\ker(c) = \zenter(G)$.
%    \item
%      Zeigen Sie, dass~$\im(c)$ eine normale Untergruppe von~$\Aut(G)$ ist.
%  \end{enumerate}
%  Man bezeichnet die Elemente von~$\Inn(G) \defined \im(c)$ als \emph{innere Automorphismen} von~$G$.
%  Die Faktorgruppe~$G/{\Inn(G)}$ ist die \emph{Gruppe der äußeren Automorphismen} von~$G$.
%\end{exercise}





\printsolutions





\end{document}
