\documentclass{scrartcl}

\usepackage{style_general}
\usepackage{style_sheets}



\titlehead{
  \textbf{Repetitorium: Einführung in die Algebra}%
%  \footnote{
%    Online verfügbar unter \url{https://gitlab.com/cionx/einfuehrung-in-die-algebra-review-ws-19-20}.
%  }
  \hfill
  2. März 2020
}
\title{Tag 1}
\author{}
\date{}



\begin{document}

\maketitle
\vspace{-6em}

\begin{exercise}
  \leavevmode
  \begin{enumerate}
    \item
      Zeigen Sie, dass jede echte Untergruppe der symmetrischen Gruppe~$\Sym_3$ zyklisch ist.
    \item
      Bestimmen Sie alle Untergruppen von~$\Sym_3$.
  \end{enumerate}
\end{exercise}

\begin{exercise}
  Es seien~$G$,~$H$ zwei endliche Gruppen deren Ordnungen~$\card{G}$,~$\card{H}$ teilerfremd sind.
  Zeigen Sie, dass der jeder Gruppenhomomorphismus~$\varphi \colon G \to H$ bereits trivial ist.
\end{exercise}

\begin{exercise}
  Es sei~$G$ eine endliche Gruppe mit~$\card{G} = 77$, und es sei~$X$ eine endliche Menge mit~$\card{X} = 17$.
  Zeigen Sie, dass jede Gruppenaktion von~$G$ auf~$X$ mindestens drei Fixpunkte besitzt.
\end{exercise}

\begin{exercise}
  Es sei~$G$ eine endliche Gruppe mit~$\card{G} = 55$ und es sei~$X$ eine endliche Menge mit~$\card{X} = 20$.
  Zeigen Sie, dass jede Gruppenaktion von~$G$ auf~$X$ mindestens vier Bahnen hat.
\end{exercise}

\begin{exercise}
  Es sei~$G$ eine Gruppe und~$X$ eine~\Menge{$G$}.
  \begin{enumerate}
    \item
      Zeigen Sie, dass~$G_{g.x} = g G_x g^{-1}$ für jedes Element~$x \in X$ und jedes Gruppenelement~$g \in G$.
    \item
      Die obige Gleichheit zeigt, dass je zwei Elemente in der gleichen~\Bahn{$G$} von~$X$ konjugierte Stabilisatoren haben.
      Entscheiden Sie ob auch die Umkehrung dieser Aussage gilt, d.h. ob Element von~$X$ mit konjugierten Stabilisatoren bereits in der gleichen~\Bahn{$G$} liegen müssen.
  \end{enumerate}
\end{exercise}

\begin{exercise}
  Es sei~$G$ eine endliche Gruppe, die as genau zwei Konjugationsklassen besteht.
  Zeigen Sie, dass bereits~$G \cong \Integer/2$ gilt.

  (\emph{Tipp}: Bestimmen Sie die Größen der beiden Konjugationsklassen.)
\end{exercise}

\begin{exercise}
  Es sei~$G$ eine Gruppe und~$N$ eine normale Untergruppe von~$G$, so dass die Faktorgruppe~$G/N$ abelsch ist.
  Zeigen Sie, dass jede Untergruppe von~$G$, die~$N$ enthält, ebenfalls normal in~$G$ ist.
\end{exercise}

\begin{exercise}
  Es sei~$G$ eine Gruppe und~$H$ eine Untergruppe von~$G$ vom endlichen Index.
  Zeigen Sie, dass~$H$ eine Untergruppe~$N$ enthält, die normal in~$G$ ist und ebenfalls endlichen Index in~$G$ hat.

  (\emph{Tipp}: Konstruieren Sie für~$n \defined [G : H]$ einen Gruppenhomomorphismus~$G \to \Sym_n$ dessen Kern in~$H$ enthalten ist.)
\end{exercise}

\begin{exercise}
  Es sei~$G$ eine Gruppen.
  Für jedes Element~$g \in G$ sei~$c_g \colon G \to G$ die Konjugation mit~$g$, d.h. es sel~$c_g(x) = g x g^{-1}$ für jedes~$x \in G$.
  \begin{enumerate}
    \item
      Zeigen Sie, dass~$c_g$ für jedes~$g \in G$ ein Gruppenautomorphismus von~$G$ ist.
    \item
      Zeigen Sie, dass die Abbildung~$c \colon G \to \Aut(G)$ gegeben durch~$g \mapsto c_g$ ein Gruppenhomomorphismus ist.
    \item
      Zeigen Sie, dass~$\ker(c) = \zenter(G)$.
    \item
      Zeigen Sie, dass~$\im(c)$ eine normale Untergruppe von~$\Aut(G)$ ist.
  \end{enumerate}
  Man bezeichnet die Elemente von~$\Inn(G) \defined \im(c)$ als \emph{innere Automorphismen} von~$G$.
  Die Faktorgruppe~$G/{\Inn(G)}$ ist die \emph{Gruppe der äußeren Automorphismen} von~$G$.
\end{exercise}

\end{document}
