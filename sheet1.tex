\documentclass{scrartcl}

\usepackage{style_general}
\usepackage{style_sheets}

\titlehead{Repetitorium: Einführung in die Algebra \hfill 2. März 2020}
\title{Tag 1}
\author{}
\date{}

\begin{document}

\maketitle
\vspace{-6em}

\begin{exercise}
  \leavevmode
  \begin{enumerate}
    \item
      Zeigen Sie, dass jede echte Untergruppe der symmetrischen Gruppe~$\Sym_3$ zyklisch ist.
    \item
      Bestimmen Sie alle Untergruppen von~$\Sym_3$.
  \end{enumerate}
\end{exercise}

\begin{exercise}
  Es sei~$G$ eine endliche Gruppe mit~$\card{G} = 77$, und es sei~$X$ eine endliche Menge mit~$\card{X} = 17$.
  Zeigen Sie, dass jede Gruppenaktion von~$G$ auf~$X$ mindestens drei Fixpunkte besitzt.
\end{exercise}

\begin{exercise}
  Es seien~$G$,~$H$ zwei endliche Gruppen deren Ordnungen~$\card{G}$,~$\card{H}$ teilerfremd sind.
  Zeigen Sie, dass der jeder Gruppenhomomorphismus~$\varphi \colon G \to H$ bereits trivial ist.
\end{exercise}

\begin{exercise}
  Es sei~$G$ eine endliche Gruppe, die as genau zwei Konjugationsklassen besteht.
  Zeigen Sie, dass bereits~$G \cong \Integer/2$ gilt.

  (\emph{Tipp}: Bestimmen Sie die Größen der beiden Konjugationsklassen.)
\end{exercise}

\begin{exercise}
  Es sei~$G$ eine Gruppe und~$N$ eine normale Untergruppe von~$G$, so dass die Faktorgruppe~$G/N$ abelsch ist.
  Zeigen Sie, dass jede Untergruppe von~$G$, die~$N$ enthält, ebenfalls normal in~$G$ ist.
\end{exercise}

\begin{exercise}
  Es sei~$G$ eine Gruppe und~$H$ eine Untergruppe von~$G$ vom endlichen Index.
  Zeigen Sie, dass~$H$ eine Untergruppe~$N$ enthält, die normal in~$G$ ist und ebenfalls endlichen Index in~$G$ hat.

  (\emph{Tipp}: Konstruieren Sie für~$n \defined [G : H]$ einen Gruppenhomomorphismus~$G \to \Sym_n$ der~$N$ im Kern enthält.)
\end{exercise}


\end{document}
