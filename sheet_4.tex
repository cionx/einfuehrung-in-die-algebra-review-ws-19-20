\documentclass{scrartcl}

\usepackage{style_general}
\usepackage{style_sheets}



\titlehead{
  \textbf{Repetitorium: Einführung in die Algebra}%
%  \footnote{
%    Online verfügbar unter \url{https://gitlab.com/cionx/einfuehrung-in-die-algebra-review-ws-19-20}.
%  }
  \hfill
  5. März 2020
}
\title{\vspace{-1em}Tag 4}
\author{}
\date{}



\begin{document}

\maketitle
\vspace{-7em}

\begin{exercise}
  \begin{enumerate}
    \item
      Es sei~$L/K$ eine algebraische Körpererweiterung und~$\closure{L}$ ein algebraischer Abschluss von~$L$.
      Zeigen Sie, dass~$\closure{L}$ auch ein algebraischer Abschluss von~$L$ ist.
    \item
      Geben Sie ein Beispiel für eine Körpererweiterung~$L/K$ an, so dass~$L$ zwar algebraisch abgeschlossen ist, aber kein algebraischer Abschluss von~$K$ ist.
    \item
      Es sei~$L/K$ eine Körpererweiterung, so dass~$L$ algebraisch abgeschlossen ist.
      Zeigen Sie, dass dann~$\Alg_{L/K}$ ein algebraischer Abschluss von~$K$ ist.
  \end{enumerate}
\end{exercise}

\begin{solution}
  \begin{enumerate}
    \item
      Da die Erweiterungen~$\closure{L} / L$ und~$L / K$ beide algebraisch sind, ist auch die Erweiterung~$\closure{L} / K$ algebraisch.
      Außerdem ist~$\closure{L}$ algebraisch abgeschlossen.
      Somit ist~$\closure{L}$ ein algebraischer Abschluss von~$K$.
    \item
      Die Körpererweiterung~$\Complex / \Rational$ ist nicht algebraisch, also ist~$\Complex$ kein algebraischer Abschluss von~$\Rational$.
      Aber~$\Complex$ ist nach dem Fundamentalsatz der Algebra algebraisch abgeschlossen.
    \item
      Wir setzen abkürzend~$A \defined \Alg_{L/K}$.
      Wir wissen (aus der Vorlesung), dass~$A$ ein Zwischenkörper der Erweiterung~$L/K$ ist.
      Per Definition von~$A$ ist diese Erweiterung algebraisch.
      Es bleibt also nur noch zu zeigen, dass~$A$ algebraisch abgeschlossen ist.

      Sei hierfür~$f \in A[X]$ ein nicht-konstantes Polynom.
      Da~$A$ ein Unterkörper von~$L$ ist, können wir~$f$ als ein Polynom in~$L[X]$ auffasen.
      Da~$L$ algebraisch abgeschlossen ist, hat~$f$ eine Nullstelle~$\alpha$ in~$L$.
      Dieses Element~$\alpha$ ist nun algebraisch über~$A$, und da die Erweiterung~$A / K$ algebraisch ist, somit auch algebraisch über~$K$.
      Dies bedeutet aber, dass bereits~$\alpha \in A$ gilt.
      Also hat~$f$ eine Nullstelle in~$A$.

      Dies zeigt, dass jedes nicht-konstante Polynom in~$A[X]$ eine Nullstelle in~$A$ hat.
      Also ist~$A$ tatsächlich algebraisch abgeschlossen.
  \end{enumerate}
\end{solution}

\begin{exercise}
  Es sei~$L/K$ eine algebraische Körpererweiterung und~$\varphi \colon L \to L$ ein~\Homomorphismus{$K$}.
  Zeigen Sie, dass~$\varphi$ bereits bijektiv ist.

  (\emph{Tipp:} Betrachten Sie für~$f \in K[X]$ die Wirkung von~$\varphi$ auf den Nullstellen von~$f$.)
\end{exercise}

\begin{solution}
  Die Abbildung~$\varphi$ ist injektiv, da sie ein Körperhomomorphismus ist.
  Es gilt zu zeigen, dass~$\varphi$ bereits surjektiv ist.
  Hierzu sei~$\alpha \in L$.
  Es sei~$\mu_\alpha \in K[X]$ das Minimalpolynom von~$\alpha$ über~$K$.
  Da~$\varphi$ ein~\Homomorphismus{$K$} ist, muss auch~$\varphi(\alpha)$ wieder eine Nullstelle von~$\mu_\alpha$ sein.
  Es gilt also für die Menge~$N$ der Nullstellen von~$\mu_\alpha$ in~$L$, also
  \[
    N
    \defined
    \{
      \beta \in L
    \suchthat
      \mu_\alpha( \beta ) = 0
    \} \,,
  \]
  dass~$\varphi(N) \subseteq N$.
  Deshalb schränkt sich~$\varphi$ zu einer Abbildung~$\restrict{\varphi}{N} \colon N \to N$ ein.
  Diese Einschränkung ist injektiv, da~$\varphi$ injektiv ist.
  Die Menge~$N$ ist allerdings endlich, weshalb die Einschränkung~$\restrict{\varphi}{N}$ bereits bijektiv ist.
  Inbesondere gibt es, da~$\alpha \in N$ gilt, ein weiteres Element~$\beta \in N$ mit~$\restrict{\varphi}{N}(\beta) = \alpha$, also~$\varphi(\beta) = \alpha$.
  Das zeigt die Surjektivität von~$\varphi$.
\end{solution}

\begin{exercise}[subtitle = {Erstklausur~18/19, Erstklausur~19/20}]
  Eine Körpererweiterung~$L/K$ ist eine Galois-Erweiterung, falls sie endlich, normal und separabel ist.
  Entscheiden Sie, ob die folgenden Körpererweiterungen Galois-Erweiterungen sin,.
  \begin{enumerate}
    \item
      $\Rational( \sqrt{2}, \sqrt{3}, i ) / \Rational$.
    \item
      $\Rational(\sqrt[4]{3}, i) / \Rational$.
    \item
      $\Rational(\sqrt[5]{2}) / \Rational$.
    \item
      $L / \Finite_2$, wobei~$L \defined \Finite_2[X] / (X^2 - X - 1)$.
  \end{enumerate}
\end{exercise}

\begin{solution}
  \begin{enumerate}
    \item
      Die gegebene Erweiterung ist endlich, da sie von endlich vielen algebraischen Elementen erzeugt ist.
      Die Erweiterung ist separabel, da~$\ringchar(\Rational) = 0$ gilt, und der Körper~$\Rational$ somit perfekt ist.
      Es gilt
      \[
        \Rational( \sqrt{2}, \sqrt{3}, i )
        =
        \Rational( \sqrt{2}, -\sqrt{2}, \sqrt{3}, -\sqrt{3}, i, -i ) \,,
      \]
      weshalb~$\Rational( \sqrt{2}, \sqrt{3}, i )$ der Zerfällungskörper der Polynome~$X^2 - 2, X^2 - 3, X^2 + 1 \in \Rational[X]$ ist.
      Deshalb ist die gegeben Erweiterung auch normal.
      Insgesamt erhalten wir, diess die gegebene Körpererweiterung eine Galois-Erweiterung ist.
    \item
      Wir erhalten wie im vorherigen Beispiel, dass die gegebene Erweiterung endlich und separabel ist.
      Wir bemerken nun, dass
      \[
        \Rational( \sqrt[4]{3}, i )
        =
        \Rational( \, \sqrt[4]{3} \,, \; i \sqrt[4]{3} \,, \; -\sqrt[4]{3} \,, \; -i \sqrt[4]{3} \, )
      \]
      gilt, da~$i = ( i \sqrt[4]{3} ) / \sqrt[4]{3}$.
      Also ist~$\Rational( \sqrt[4]{3}, i )$ der Zerfällungskörper des Polynoms~$X^4 - 3 \in \Rational[X]$.
      Die gegebene Körpererweiterung ist somit auch normal, und deshalb insgesamt eine Galois-Erweiterung.
    \item
      Das Polynom~$f \defined X^5 - 2 \in \Rational[X]$ ist nach dem Eisenstein-Kriterium mit~$p = 2$ irreduzibel.
      Es hat zwar eine Nullstelle in~$\Rational( \sqrt[5]{3})$, zerfällt aber in~$\Rational( \sqrt[5]{3})[X]$ noch nicht in Linearfaktoren, denn alle anderen Nullstellen von~$f$ sind nicht-reell, während~$\Rational( \sqrt[5]{3})$ ein Unterkörper von~$\Real$ ist.
      Das zeigt, dass die gegebene Erweiterung nicht normal ist, und somit auch keine Galois-Erweiterung ist.
      (Es ergibt sich allerdings wie in den vorherigen Beispielen, dass die Erweiterung endlich und separabel ist.)
    \item
      Das gegebene Polynom~$f \defined X^2 - X - 1 \in \Finite_2[X]$ ist quadratisch und hat keine Nullstellen in~$\Finite_2$, und ist somit irreduzibel in~$\Finite_2[X]$.
      Da~$\Finite_2[X]$ ein Hauptidealring ist, ist somit der Faktorring~$L = \Finite_2[X] / \genideal{f}$ tatsächlich ein Körper.

      Für das Element~$\alpha \in L$ gegeben durch~$\alpha \defined \class{X}$ gilt~$L = \Finite_2[\alpha]$, und somit auch~$L = \Finite_2(\alpha)$.
      Per Konstruktion von~$L$ ist~$\alpha$ eine Nullstelle des Polynoms~$f$.
      Da~$f$ irreduzibel und normiert ist, ist~$f$ bereits das Minimalpolynom von~$f$.
      Inbesondere ist der Grad
      \[
        [L : \Finite_2]
        =
        [\Finite_2(\alpha) : \Finite_2]
        =
        \deg(f)
        =
        2
      \]
      endlich, die Erweiterung~$L/\Finite_2$ also endlich.
      Da das quadratische~(!) Polynom~$f \in \Finite_2[X]$ eine Nullstelle in~$L$ hat, muss es in~$L[X]$ bereits in zwei Linearfaktoren zerfallen.
      Somit ist~$L$ der Zerfällungskörper von~$f$, und die Erweiterung~$L/\Finite_2$ somit normal.

      Die Erweiterung ist außerdem separabel, da der Körper~$\Finite_2$ endlich ist, und somit perfekt.
      Insgesamt ist die Erweiterung~$L/\Finite_2$ also eine Galois-Erweiterung.
  \end{enumerate}
\end{solution}

\begin{exercise}[subtitle = {Erstklausur~18/19}]
  Es sei~$p$ eine Primzahl und~$\alpha \defined \sqrt[p]{p} \in \Real$.
  \begin{enumerate}
    \item
      Bestimmen Sie das Minimalpolynom von~$\alpha$ über~$\Rational$.
    \item
      Bestimmen Sie in Abhängigkeit von~$p$, ob die Erweiterung~$\Rational(\alpha) / \Rational$ normal ist.
    \item
      Bestimmen Sie in Abhängigkeit von~$p$, ob die Erweiterung~$\Rational(\alpha) / \Rational$ eine Galois-Erweiterung ist.
  \end{enumerate}
\end{exercise}

\begin{solution}
  \begin{enumerate}
    \item
      Das Element~$\alpha$ ist eine Nullstelle des Polynoms~$X^p - p \in \Rational[X]$.
      Dieses Polynom ist irreduzibel nach dem Eisenstein-Kriterium bzgl. der Primzahl~$p$.
      Es ist außerdem normiert.
      Also ist es bereits das Minimalpolynom von~$\alpha$.
    \item
      Für~$p = 2$ gilt~$\alpha = 1$ und somit~$\Rational(\alpha) = \Rational$.
      Die Erweiterung~$\Rational / \Rational$ ist normal, also ist~$\Rational(\alpha) / \Rational$ für~$p = 2$ normal.

      Für~$p \geq 3$ hingegen hat das irreduzible Polynom~$f \defined X^p - p \in \Rational[X]$ zwar eine Nullstelle in~$\Rational(\alpha)$, nämlich~$\alpha$, zerfällt in~$\Rational(\alpha)[X]$ allerdings noch nicht in Linearfaktoren, da~$f$ in diesem Fall auch auch nicht-reelle Nullstellen hat,~$\Rational(\alpha)$ aber ein Unterkörper von~$\Real$ ist.
      Da~$f$ irreduzibel ist, zeigt dies, dass die Erweiterung~$\Rational(\alpha) / \Rational$ für~$p \geq 3$ nicht normal ist.
    \item
      Für~$p \geq 3$ ist die Erweiterung nicht normal, und somit auch keine Galois-Erweiterung.
      Für~$p = 2$ erhalten wir die Erweiterung~$\Rational/\Rational$, welche eine Galois-Erweiterung ist.
  \end{enumerate}
\end{solution}

\begin{exercise}[subtitle = {Erstklausur~18/19}]
  \begin{enumerate}
    \item
      Zeigen Sie, dass die Körpererweiterung~$\Rational(\sqrt[3]{2}) / \Rational$ nicht normal ist.
    \item
      Zeigen Sie, dass~$\Gal( \Rational(\sqrt[3]{2}) / \Rational ) = 1$.
  \end{enumerate}
\end{exercise}

\begin{solution}
  \begin{enumerate}
    \item
      Wir betrachten das Polynom~$f \defined X^3 - 2 \in \Rational[X]$.
      Dieses Polynom ist irreduzibel nach dem Eisenstein-Kriterium mit~$p = 3$.
      (Alternativ hat~$f$ keine Nullstelle in~$\Rational$, und ist somit irreduzibel in~$\Rational[X]$, da~$f$ kubisch ist.)
      Das irreduzible Polynom~$f$ hat eine Nullstelle in~$\Rational(\sqrt[3]{2})$, nämlich~$\sqrt[3]{2}$.
      Aber es zerfällt über~$\Rational(\sqrt[3]{2})$ noch nicht in Linearfaktoren, denn~$f$ hat zwei nicht-reelle Nullstellen, während~$\Rational(\sqrt[3]{2})$ ein Unterkörper von~$\Real$ ist.
      Das zeigt, dass die Erweiterung~$\Rational(\sqrt[3]{2}) / \Rational$ nicht normal ist.
    \item
      Das Polynom~$f = X^3 - 2$ ist nicht nur irreduzibel, sondern ist auch normiert und hat~$\sqrt[3]{2}$ als Nullstelle.
      Es ist also somit bereits das Minimalpolynom von~$\sqrt[3]{2}$ über~$\Rational$.
      Wir erhalten hiermit eine Bijektion
      \[
        \Gal\bigl( \Rational(\sqrt[3]{2}) / \Rational \bigr)
        \to
        \bigl\{
          \text{Nullstellen von~$f$ in~$\Rational( \sqrt[3]{2} )$}
        \bigr\} \,,
        \quad
        \varphi
        \mapsto
        \varphi(\alpha) \,.
      \]
      Da das Polynom~$f$ nur eine Nullstelle in~$\Rational(\sqrt[3]{2})$ hat, nämlich~$\sqrt[3]{2}$, erhalten wir aus dieser Bijektion, dass~$\card{ \Gal( \Rational(\sqrt[3]{2}) / \Rational ) } = 1$ gilt.
      Also gilt~$\Gal( \Rational( \sqrt[3]{2} ) / \Rational ) = 1$.
  \end{enumerate}
\end{solution}

\begin{exercise}
  \begin{enumerate}
    \item
      Es sei~$L/K$ eine Körpererweiterung vom Grad~$[L : K] = 2$.
      Zeigen Sie, dass~$L$ der Zerfällungskörper eines quadratischen Polynoms aus~$K[X]$ ist.
      Folgern Sie, dass die Erweiterung~$L/K$ normal ist.
    \item
      Zeigen Sie, dass die Erweiterungen~$\Rational(\sqrt[4]{2}) / \Rational(\sqrt{2})$ und~$\Rational(\sqrt{2}) / \Rational$ beide normal sind.
    \item
      Zeigen Sie, dass die Erweiterung~$\Rational(\sqrt[4]{2}) / \Rational$ jedoch nicht normal ist.
  \end{enumerate}
  Dies zeigt, dass Normalität von Körpererweiterungen nicht transitiv ist.
\end{exercise}

\begin{solution}
  \begin{enumerate}
    \item
      Es sei~$\alpha \in L$ mit~$\alpha \notin K$.
      Es ist~$K(\alpha)$ ein Zwischenkörper der Erweiterung~$L / K$ mit~$K(\alpha) \neq K$, also mit~$[K(\alpha) : K] \geq 2$.
      Aus~$[L : K] = 2$ folgt, das bereits~$[K(\alpha) : K] = 2$ gilt, und somit~$K(\alpha) = L$.

      Die Erweiterung~$L/K$ ist algebraisch, da sie algebraisch ist.
      Es sei~$\mu_\alpha$ das Minimalpolynom von~$\alpha$ über~$K$.
      Es gilt
      \[
        \deg(\mu_\alpha)
        =
        [K(\alpha) : K]
        =
        [L : K]
        =
        2 \,,
      \]
      weshalb das Polynom~$\mu_\alpha$ quadratisch ist.
      Da~$\mu_\alpha$ eine Nullstelle in~$L$ hat, nämlich~$\alpha$, und quadratisch ist, zerfällt~$\mu_\alpha$ in~$K[X]$ bereits in zwei Linearfaktoren,
      \[
        \mu_\alpha
        =
        (X - \alpha) (X - \beta) \,.
      \]
      Für die beiden Nullstellen~$\alpha$,~$\beta$ gilt
      \[
        L
        \supseteq
        K(\alpha, \beta)
        \supseteq
        K(\alpha)
        =
        L
      \]
      und somit~$L = K(\alpha, \beta)$.
      Es ist also~$L$ der Zerfällungskörper von~$\mu_\alpha$.
      Somit ist~$L/K$ normal.
    \item
      Das Minimalpolynom von~$\sqrt{2}$ über~$\Rational$ ist~$f \defined X^2 - 2 \in \Rational[X]$.
      Denn dieses Polynom ist normiert, hat~$\sqrt{2}$ als Nullstelle, und ist nach dem Eisenstein-Kriterium mit~$p = 2$ irreduzibel.
      (Alternativ ist~$f$ quadratisch und hat keine Nullstelle in~$\Rational$, und ist somit irreduzibel.)
      
      Das Minimalpolynom von~$\sqrt[4]{2}$ über~$\Rational$ ist~$g \defined X^4 - 2 \in \Rational[X]$.
      Denn dieses Polynom ist normiert, hat~$\sqrt[4]{2}$ als Nullstelle, und ist nach dem Eisenstein-Kriterium mit~$p = 2$ irreduzibel.

      Wir erhalten nun, dass
      \[
        [\Rational(\sqrt{2}) : \Rational]
        =
        \deg(f)
        =
        2 \,,
        \quad
        [\Rational(\sqrt[4]{2}) : \Rational]
        =
        \deg(g)
        =
        4 \,,
      \]
      und somit nach der Gradformel auch, dass
      \[
        [\Rational(\sqrt[4]{2}) : \Rational(\sqrt{2})]
        =
        \frac{ [\Rational(\sqrt[4]{2}) : \Rational] }{ [\Rational(\sqrt{2}) : \Rational] }
        =
        \frac{4}{2}
        =
        2 \,.
      \]
      Die Erweiterungen~$\Rational(\sqrt{2}) / \Rational$ und~$\Rational(\sqrt[4]{2}) / \Rational(\sqrt{2})$ sind also beide vom Grad~$2$, und somit nach dem vorherigen Aufgabenteil normal.
    \item
      Das Polynom~$g = X^4 - 2 \in \Rational[X]$ ist irreduzibel und hat eine Nullstelle in~$\Rational(\sqrt[4]{2})$.
      Es zerfällt aber über~$\Rational(\sqrt[4]{2})$ noch nicht in Linearfaktoren, denn~$g$ hat nicht-reelle Nullstellen, aber~$\Rational(\sqrt[4]{2})$ ist ein Unterkörper von~$\Real$.
      Das zeigt, dass die Erweiterung~$\Rational(\sqrt[4]{2}) / \Rational$ nicht normal ist.
  \end{enumerate}
\end{solution}

\begin{exercise}
  Es sei~$K$ ein perfekter Körper und~$L/K$ eine endliche Körpererweiterung.
  Zeigen Sie, dass auch~$L$ perfekt ist, indem Sie zeigen, dass jede algebraische Körpererweiterung~$M/L$ separabel ist.
\end{exercise}

\begin{solution}
  Es gilt zu zeigen, dass jede algebraische Körpererweiterung~$M/L$ bereits separabel ist.
  Hierfür beobachten wir, dass die Erweiterung~$L/K$ algebraisch ist, da sie endlich ist.
  Also ist auch die Erweiterung~$M/K$ algebraisch.
  Es folgt aus der Perfektheit von~$K$, dass die Erweiterung~$M/K$ seperabel ist.
  Damit ist auch die Erweiterung~$M/L$ separabel.
\end{solution}





\clearpage





\printsolutions





\end{document}
