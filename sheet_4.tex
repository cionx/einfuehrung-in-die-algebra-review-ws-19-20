\documentclass{scrartcl}

\usepackage{style_general}
\usepackage{style_sheets}



\titlehead{
  \textbf{Repetitorium: Einführung in die Algebra}%
%  \footnote{
%    Online verfügbar unter \url{https://gitlab.com/cionx/einfuehrung-in-die-algebra-review-ws-19-20}.
%  }
  \hfill
  5. März 2020
}
\title{Tag 4}
\author{}
\date{}



\begin{document}

\maketitle
\vspace{-6em}

\begin{exercise}
  \begin{enumerate}
    \item
      Es sei~$L/K$ eine algebraische Körpererweiterung und~$\closure{L}$ ein algebraischer Abschluss von~$L$.
      Zeigen Sie, dass~$\closure{L}$ auch ein algebraischer Abschluss von~$L$ ist.
    \item
      Geben Sie ein Beispiel für eine Körpererweiterung~$L/K$ an, so dass~$L$ zwar algebraisch abgeschlossen ist, aber kein algebraischer Abschluss von~$K$ ist.
    \item
      Es sei~$L/K$ eine Körpererweiterung, so dass~$L$ algebraisch abgeschlossen ist.
      Zeigen Sie, dass dann~$\Alg_{L/K}$ ein algebraischer Abschluss von~$K$ ist.
%    \item
%      Zeigen Sie, dass~$[\closure{\Rational} : \Rational] = \infty$.
%      (\emph{Tipp:} Betrachten Sie die Polynome~$X^n - 2 \in \Rational[X]$.)
  \end{enumerate}
\end{exercise}

%\begin{exercise}
%  Es sei~$K$ ein endlicher Körper.
%  \begin{enumerate}
%    \item
%      Geben Sie ein Beispiel für ein Polynom~$f \in K[X]$ an, so dass zwar~$f \neq 0$ gilt, aber dennoch~$f(x) = 0$ für alle~$x \in K$.
%    \item
%      Folgern Sie, dass~$K$ nicht algebraisch abgeschlossen ist.
%  \end{enumerate}
%\end{exercise}
%
%\begin{exercise}
%  \begin{enumerate}
%    \item
%      Es sei~$L/K$ eine algebraische Körpererweiterung, so dass jedes nicht-konstante Polynom~$f \in K[X]$ über~$L$ in Linearfaktoren zerfällt.
%      Zeigen Sie, dass~$L$ ein algebraischer Abschluss von~$K$ ist.
%  \end{enumerate}
%  Wir wollen im Folgenden einen algebraischen Abschluss von~$\Finite_p$ konstruieren.
%  \begin{enumerate}
%    \item
%      Es sei~$f \in \Finite_p[X]$ ein nicht-konstantes Polynom.
%      Zeigen Sie, dass~$f$ in~$\Finite_{p^n}$ 
%  \end{enumerate}
%\end{exercise}

\begin{exercise}
  Es sei~$L/K$ eine algebraische Körpererweiterung und~$\varphi \colon L \to L$ ein~\Homomorphismus{$K$}.
  Zeigen Sie, dass~$\varphi$ bereits bijektiv ist.

  (\emph{Tipp:} Betrachten Sie für~$f \in K[X]$ die Wirkung von~$\varphi$ auf den Nullstellen von~$f$.)
\end{exercise}

\begin{exercise}[subtitle = {Erstklausur~18/19, Erstklausur~19/20}]
  Eine Körpererweiterung~$L/K$ ist eine Galois-Erweiterung, falls sie endlich, normal und separabel ist.
  Entscheiden Sie, ob die folgenden Körpererweiterungen Galois-Erweiterungen sin,.
  \begin{enumerate}
    \item
      $\Rational( \sqrt{2}, \sqrt{3}, i ) / \Rational$.
    \item
      $\Rational(\sqrt[4]{3}, i) / \Rational$
    \item
      $\Rational(\sqrt[5]{2}) / \Rational$
    \item
      $L / \Finite_2$, wobei~$L \defined \Finite_2[X] / (X^2 - X - 1)$.
  \end{enumerate}
\end{exercise}

\begin{exercise}[subtitle = {Erstklausur~18/19}]
  Es sei~$p$ eine Primzahl und~$\alpha \defined \sqrt[p]{p} \in \Real$.
  \begin{enumerate}
    \item
      Bestimmen Sie das Minimalpolynom von~$\alpha$ über~$\Rational$.
    \item
      Bestimmen Sie in Abhängigkeit von~$p$, ob die Erweiterung~$\Rational(\alpha) / \Rational$ normal ist.
    \item
      Bestimmen Sie in Abhängigkeit von~$p$, ob die Erweiterung~$\Rational(\alpha) / \Rational$ eine Galois-Erweiterung ist.
  \end{enumerate}
\end{exercise}

\begin{exercise}[subtitle = {Erstklausur~18/19}]
  \begin{enumerate}
    \item
      Zeigen Sie, dass die Körpererweiterung~$\Rational(\sqrt[3]{2}) / \Rational$ nicht normal ist.
    \item
      Zeigen Sie, dass~$\Gal( \Rational(\sqrt[3]{2}) / \Rational ) = 1$.
  \end{enumerate}
\end{exercise}

\begin{exercise}
  \begin{enumerate}
    \item
      Es sei~$L/K$ eine Körpererweiterung vom Grad~$[L : K] = 2$.
      Zeigen Sie, dass~$L$ der Zerfällungskörper eines quadratischen Polynoms aus~$K[X]$ ist.
      Folgern Sie, dass die Erweiterung~$L/K$ normal ist.
    \item
      Zeigen Sie, dass die Erweiterungen~$\Rational(\sqrt[4]{2}) / \Rational(\sqrt{2})$ und~$\Rational(\sqrt{2}) / \Rational$ beide normal sind.
    \item
      Zeigen Sie, dass die Erweiterung~$\Rational(\sqrt[4]{2}) / \Rational$ jedoch nicht normal ist.
  \end{enumerate}
  Dies zeigt, dass Normalität von Körpererweiterungen nicht transitiv ist.
\end{exercise}

\begin{exercise}
  Es sei~$K$ ein perfekter Körper und~$L/K$ eine endliche Körpererweiterung.
  Zeigen Sie, dass auch~$L$ perfekt ist, indem Sie zeigen, dass jede algebraische Körpererweiterung~$M/L$ separabel ist.
\end{exercise}

\end{document}
